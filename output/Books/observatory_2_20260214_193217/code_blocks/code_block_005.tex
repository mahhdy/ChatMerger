% ═══════════════════════════════════════════════════════════════════════════════
% ادامه فصل ۸ — از جمع‌بندی فصل
% ═══════════════════════════════════════════════════════════════════════════════

\begin{chaptersummary}
یافته‌های کلیدی این فصل:

\begin{enumerate}
    \item \textbf{نیروی انسانی عظیم اما تدریجی}: اوج نیاز در فاز ۲ (۲۷,۰۰۰-۶۴,۰۰۰ نفر) است، اما استقرار باید تدریجی و با آموزش کافی باشد. اتکای بیش‌ازحد به نیروی بین‌المللی نه ممکن است و نه مطلوب.
    
    \item \textbf{ظرفیت ایرانی ستون فقرات است}: ۸۰-۹۰٪ نیروی انسانی باید ایرانی باشد. دیاسپورا منبع ارزشمند اما باید با دقت مدیریت شود (سقف ۲۰-۳۰٪).
    
    \item \textbf{هشت نهاد کلیدی باید ایجاد شوند}: دفتر \lr{SRSG}، کمیسیون انتخابات، کمیسیون حقیقت، دادگاه ویژه، صندوق امانی، کمیسیون رسانه، کمیسیون ضدفساد، و نهاد حقوق اقوام.
    
    \item \textbf{زیرساخت فنی حیاتی‌تر از آنچه به نظر می‌رسد}: رفع فیلترینگ اینترنت در روز اول، سامانه ثبت رأی‌دهندگان، و امنیت سایبری از اولویت‌های بالا هستند.
    
    \item \textbf{انتخاب فناوری انتخابات}: رأی کاغذی با شمارش علنی در مرحله اول، امن‌ترین و قابل‌اعتمادترین گزینه است.
    
    \item \textbf{چارچوب حقوقی سه‌لایه}: قطعنامه شورای امنیت (بین‌المللی) + \lr{SOMA} (دوجانبه) + قوانین موقت (داخلی) باید همزمان آماده شوند.
    
    \item \textbf{از خلأ قانونی اجتناب شود}: قوانین موجود که با حقوق بشر سازگارند، تا تصویب قانون جدید حفظ شوند (درس عراق).
    
    \item \textbf{الحاق به معاهدات بین‌المللی}: اساسنامه رم، \lr{CEDAW}، پروتکل شکنجه و سایر معاهدات حقوق بشری باید در اولویت باشند.
\end{enumerate}

\vspace{0.5cm}
\textit{در فصل بعد (\ref{ch:timeline})، زمان‌بندی تفصیلی، ساختار تیم‌ها و مکانیزم هماهنگی بررسی خواهد شد.}
\end{chaptersummary}

\chapterend