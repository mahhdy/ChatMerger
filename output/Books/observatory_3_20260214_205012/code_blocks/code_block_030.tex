%══════════════════════════════════════════════════════════════
% ادامهٔ پیوست د — نمودار شبکهٔ نهادها
%══════════════════════════════════════════════════════════════

\begin{tikzpicture}[
  node distance=2cm,
  core/.style={
    draw=MainPurple, fill=MainPurple!15, rounded corners=5pt,
    minimum width=2.5cm, minimum height=1cm, font=\small\bfseries,
    align=center, thick
  },
  un/.style={
    draw=MainBlue, fill=MainBlue!10, rounded corners=3pt,
    minimum width=2cm, minimum height=0.7cm, font=\tiny,
    align=center
  },
  regional/.style={
    draw=MainGreen, fill=MainGreen!10, rounded corners=3pt,
    minimum width=2cm, minimum height=0.7cm, font=\tiny,
    align=center
  },
  ngo/.style={
    draw=MainOrange, fill=MainOrange!10, rounded corners=3pt,
    minimum width=2cm, minimum height=0.7cm, font=\tiny,
    align=center
  },
  iranian/.style={
    draw=MainRed, fill=MainRed!10, rounded corners=3pt,
    minimum width=2cm, minimum height=0.7cm, font=\tiny,
    align=center
  },
  finance/.style={
    draw=MainYellow!80!black, fill=MainYellow!10, rounded corners=3pt,
    minimum width=2cm, minimum height=0.7cm, font=\tiny,
    align=center
  },
  link/.style={-, gray!50, thin},
  stronglink/.style={-, MainPurple!60, thick}
]

% هستهٔ مرکزی
\node[core] (iran) at (0,0) {گذار\\ایران};

% سازمان ملل (بالا)
\node[un] (unsc) at (-3,3.5) {\lr{UNSC}\\شورای امنیت};
\node[un] (srsg) at (0,3.5) {\lr{SRSG/UNMOIT}\\مأموریت ایران};
\node[un] (iaea) at (3,3.5) {\lr{IAEA}\\هسته‌ای};
\node[un] (ohchr) at (-1.5,2.3) {\lr{OHCHR}\\حقوق بشر};
\node[un] (undp) at (1.5,2.3) {\lr{UNDP}\\توسعه};

% منطقه‌ای (چپ)
\node[regional] (eu) at (-5,1) {\lr{EU}\\اتحادیهٔ اروپا};
\node[regional] (osce) at (-5,-0.5) {\lr{OSCE/ODIHR}\\ناظران};
\node[regional] (oic) at (-5,-2) {\lr{OIC}\\همکاری اسلامی};

% دولت‌ها (راست)
\node[ngo] (us) at (5,1.5) {آمریکا\\تحریم/مشوق};
\node[ngo] (china) at (5,0) {چین + روسیه\\وتو/حمایت؟};
\node[ngo] (turkey) at (5,-1.5) {ترکیه + عربستان\\همسایگان};

% مالی (پایین-راست)
\node[finance] (wb) at (3,-3) {\lr{WB/IMF}\\بازسازی};
\node[finance] (fund) at (0,-3.5) {صندوق امانی\\بازسازی ایران};

% ایرانی (پایین-چپ)
\node[iranian] (const) at (-3,-3) {مجلس مؤسسان\\ایران};
\node[iranian] (trc) at (-1.5,-2.3) {کمیسیون حقیقت\\ایران};
\node[iranian] (civil) at (1.5,-2.3) {جامعهٔ مدنی\\ایران};

% NGOها (بالا-راست)
\node[ngo] (hrw) at (4.5,3) {\lr{HRW/AI}\\حقوق بشر};
\node[ngo] (ictj) at (3,2.3) {\lr{ICTJ}\\عدالت انتقالی};

% اتصالات قوی
\draw[stronglink] (iran) -- (srsg);
\draw[stronglink] (iran) -- (const);
\draw[stronglink] (iran) -- (trc);
\draw[stronglink] (iran) -- (civil);
\draw[stronglink] (iran) -- (fund);
\draw[stronglink] (srsg) -- (unsc);

% اتصالات عادی
\draw[link] (iran) -- (eu);
\draw[link] (iran) -- (osce);
\draw[link] (iran) -- (oic);
\draw[link] (iran) -- (us);
\draw[link] (iran) -- (china);
\draw[link] (iran) -- (turkey);
\draw[link] (iran) -- (wb);
\draw[link] (srsg) -- (ohchr);
\draw[link] (srsg) -- (undp);
\draw[link] (srsg) -- (iaea);
\draw[link] (srsg) -- (hrw);
\draw[link] (srsg) -- (ictj);
\draw[link] (fund) -- (wb);
\draw[link] (fund) -- (eu);
\draw[link] (trc) -- (ictj);
\draw[link] (const) -- (eu);

% راهنما
\node[font=\tiny, MainBlue] at (-5.5,3.5) {$\blacksquare$ سازمان ملل};
\node[font=\tiny, MainGreen] at (-5.5,3) {$\blacksquare$ منطقه‌ای};
\node[font=\tiny, MainOrange] at (-5.5,2.5) {$\blacksquare$ دولت‌ها/\lr{NGO}};
\node[font=\tiny, MainRed] at (-5.5,2) {$\blacksquare$ ایرانی};
\node[font=\tiny, MainYellow!80!black] at (-5.5,1.5) {$\blacksquare$ مالی};

\end{tikzpicture}
\caption{شبکهٔ نهادهای کلیدی مرتبط با نظارت بر گذار ایران}
\label{fig:app-org-network}
\end{figure}

\sectiondivider

%═══════════════════════════════════════════════════════════
\section{جمع‌بندی پیوست}
\label{app:org:conclusion}
%═══════════════════════════════════════════════════════════

\begin{chaptersummary}
جمع‌بندی پیوست د — فهرست نهادها و سازمان‌ها:

\begin{enumerate}[nosep]
\item بیش از \textbf{۸۰ نهاد} در ۹ دسته معرفی شدند: سازمان ملل (۱۴)، منطقه‌ای (۸)، دولت‌ها (۱۴)، \lr{NGO}ها (۱۴)، مالی (۶)، ایرانی (۱۴)، آکادمیک (۱۰)، رسانه‌ای (۷).
\item \textbf{شورای امنیت سازمان ملل} مرجع تصمیم‌گیری اصلی است — ریسک وتوی چین و روسیه باید مدیریت شود.
\item \textbf{\lr{SRSG} و مأموریت \lr{UNMOIT}} هستهٔ مرکزی نظارت بین‌المللی خواهد بود.
\item \textbf{نهادهای ایرانی پیشنهادی} (مجلس مؤسسان، کمیسیون حقیقت، ارتش ملی واحد) \textbf{اصلی‌ترین بازیگران} هستند — مالکیت ملی.
\item \textbf{چهارگانهٔ ایرانی} (الگوی تونس) و \textbf{تشکل‌های صنفی} نقش میانجی‌گری خواهند داشت.
\item \textbf{رسانه‌های مستقل ایرانی} باید تقویت شوند — رسانه‌های دولتی خارجی بی‌اعتمادی ایجاد می‌کنند.
\item \lr{IAEA} نقش ویژه‌ای در بُعد هسته‌ای دارد — مکمل فرآیند سیاسی.
\item صندوق امانی بازسازی ایران (\$۲.۵-۵B) با مشارکت بانک جهانی، \lr{EU}، و کشورهای کمک‌دهنده تأمین خواهد شد.
\end{enumerate}

\vspace{0.3cm}
\textit{ارجاعات:}
\begin{itemize}[nosep]
\item نهادها و بازیگران (تفصیلی): \seeChapter{ch:actors}
\item بودجه‌بندی و تأمین مالی: \seeChapter{ch:budget}
\item زمان‌بندی و تیم‌سازی: \seeChapter{ch:timeline}
\item واژه‌نامهٔ تخصصی: \seeChapter{app:glossary}
\end{itemize}
\end{chaptersummary}

\chapterend

%══════════════════════════════════════════════════════════════
% پایان پیوست د
%══════════════════════════════════════════════════════════════