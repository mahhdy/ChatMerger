% =============================================================================
% فصل: مشارکت مستقیم شهروندان در درآمدهای منابع طبیعی
% بر اساس ایده‌های دکتر پویا ناظران و تجربیات بین‌المللی
% =============================================================================

\chapter{نقشه‌ راه مشارکت شهروندان در درآمدهای منابع طبیعی}
\label{chap:oil-revenue-distribution}

\begin{abstract}
این فصل به بررسی تفصیلی طرح توزیع مستقیم درآمدهای حاصل از منابع طبیعی به شهروندان می‌پردازد.
ایدهٔ محوری، الهام‌گرفته از مدل پیشنهادی دکتر پویا ناظران (تحلیلگر مؤسسهٔ مودیز و اقتصاددان دانش‌آموختهٔ دانشگاه ایالتی اوهایو)،
بر آن است که مالکیت حقیقی منابع طبیعی از آنِ شهروندان است و دولت صرفاً نقش نمایندگی و امانتداری دارد.
بر اساس این دیدگاه، عواید حاصل از فروش نفت، گاز و سایر منابع طبیعی باید مستقیماً و به‌صورت نقدی بین تمامی شهروندان توزیع شود،
نه آن‌که در اختیار دستگاه دولتی قرار گیرد.
\end{abstract}

% ---------------------------------------------------------------------------
\section{مقدمه: نفرین منابع و دولت رانتیر}
\label{sec:resource-curse-intro}

پدیدهٔ «نفرین منابع» (\lr{Resource Curse}) یکی از مباحث شناخته‌شده در اقتصاد توسعه است.
کشورهایی که از ثروت‌های طبیعی فراوان برخوردارند، غالباً به‌جای رشد اقتصادی پایدار و توسعهٔ انسانی،
با فساد ساختاری، اقتصاد رانتی، ناکارآمدی نهادها و سرکوب نهادهای دموکراتیک مواجه می‌شوند.

در ایران، درآمدهای نفتی به‌طور تاریخی مستقیماً به حساب دولت واریز شده و دولت نقش تخصیص‌دهندهٔ اصلی این منابع را ایفا کرده است.
این ساختار نتایج زیر را در پی داشته است:

\begin{enumerate}
\item \textbf{اقتصاد رانتی:}
بنگاه‌ها به‌جای جلب رضایت مشتریان، برای دسترسی به رانت‌های دولتی (ارز ترجیحی، تسهیلات ارزان، یارانهٔ نهاده‌ها) رقابت می‌کنند.

\item \textbf{فساد سیستماتیک:}
اختیار تخصیص منابع عظیم، زمینهٔ فساد گسترده را فراهم می‌کند.
نمونهٔ بارز آن رانت ارز ترجیحی ۴۲۰۰ تومانی بود که میلیاردها دلار ثروت ملی را به جیب واسطه‌گران منتقل کرد.

\item \textbf{عدم پاسخگویی دولت:}
دولتی که درآمدش را از فروش نفت کسب می‌کند، نه به مالیات شهروندان وابسته است و نه الزامی به پاسخگویی به آنان دارد.

\item \textbf{بی‌ثباتی اقتصاد کلان:}
نوسانات قیمت نفت مستقیماً بر بودجهٔ دولت، نقدینگی و تورم اثر می‌گذارد.

\item \textbf{ناکارآمدی صندوق توسعهٔ ملی:}
صندوق توسعهٔ ملی که با هدف ذخیره‌سازی بین‌نسلی تأسیس شد، در عمل به «قلک دولت» تبدیل شده
و تحت فشار بودجه‌ای دولت‌های مختلف خالی شده است.
\end{enumerate}


% ---------------------------------------------------------------------------
\section{چارچوب نظری: حق مالکانهٔ شهروندان}
\label{sec:theoretical-framework}

\subsection{تعریف حق مالکانه}

دکتر پویا ناظران تأکید دارد که توزیع نقدی عواید نفتی واقعاً «یارانه» نیست.
مفهوم دقیق‌تر، \textbf{حق مالکانه} (\lr{Royalty Right}) مردم است.
منابع طبیعی زیرزمینی ثروت مشترک ملی هستند و هر شهروند ایرانی سهمی ذاتی در این ثروت دارد.
دولت صرفاً امانتدار و نماینده است، نه مالک.

\subsection{سه اصل بنیادین}

ناظران سه اصل کلیدی را مطرح می‌کند:

\begin{enumerate}
\item \textbf{مردم امین‌تر از دولت هستند:}
مردم بر ثروت خدادادی سرزمین خود امین‌تر از دولت‌اند.
پدر و مادرها دلسوزتر بر سرنوشت فرزندان، نوه‌ها و نسل‌های آینده‌اند تا بوروکراسی دولتی.
تجربهٔ صندوق توسعهٔ ملی ایران مؤید این مدعاست.

\item \textbf{بنگاه‌ها باید از طریق جلب رضایت مردم کسب درآمد کنند، نه از طریق رابطه با دولت:}
هنگامی که منابع نفتی مستقیماً بین مردم توزیع شود، بنگاه‌های اقتصادی برای جذب این منابع باید
محصول بهتر و خدمات بهتری ارائه دهند. اما اگر دولت تخصیص‌دهنده باشد، رقابت برای دسترسی
به «رانت» جایگزین رقابت برای خدمت به مشتری می‌شود.

\item \textbf{یارانهٔ نقدی که به پشتوانهٔ پول بانک مرکزی نباشد، تورم‌زا نیست:}
فروش ارز حاصل از نفت صادراتی به این معناست که ریال از دست خریداران ارز گرفته شده
و بین مردم توزیع می‌شود. در این فرایند حجم پول افزایش نمی‌یابد و لذا تورم‌زا نیست.
\end{enumerate}


% ---------------------------------------------------------------------------
\section{مکانیسم پیشنهادی: صندوق سهام ملی منابع طبیعی}
\label{sec:national-equity-fund}

\subsection{ساختار کلی صندوق}

طرح پیشنهادی، تأسیس \textbf{صندوق سهام ملی منابع طبیعی} (\lr{National Natural Resource Equity Fund}) است
که بر اساس اصول زیر عمل می‌کند:

\begin{enumerate}
\item تمامی درآمدهای حاصل از فروش نفت خام، گاز طبیعی، میعانات گازی و سایر منابع معدنی
مستقیماً به این صندوق واریز می‌شود.

\item هر شهروند ایرانی (اعم از زن و مرد، کودک و بزرگسال) دارای یک «حساب شهروندی» در این صندوق است.

\item عواید صندوق به‌صورت دوره‌ای (ماهانه یا فصلی) به‌طور مساوی بین تمامی حساب‌های شهروندی توزیع می‌شود.

\item دولت هیچ دسترسی مستقیمی به منابع صندوق ندارد و هزینه‌های عمومی خود را
صرفاً از طریق \textbf{مالیات‌ستانی} از شهروندان و بنگاه‌ها تأمین می‌کند.
\end{enumerate}

\subsection{فرایند عملیاتی گام‌به‌گام}

\begin{table}[htbp]
\centering
\caption{فرایند عملیاتی صندوق سهام ملی منابع طبیعی}
\label{tab:fund-operations}
\begin{tabular}{|c|p{10cm}|}
\hline
\textbf{گام} & \textbf{شرح} \\
\hline
۱ & نفت خام و گاز توسط شرکت ملی نفت استخراج و در بازارهای بین‌المللی به فروش می‌رسد. \\
\hline
۲ & ارز حاصل از صادرات وارد صندوق سهام ملی می‌شود. \\
\hline
۳ & صندوق ارز را در بازار ارز داخلی عرضه می‌کند (یا مستقیماً به ریال تبدیل می‌کند). \\
\hline
۴ & ریال حاصل (که از خریداران ارز دریافت شده) به‌طور مساوی بین حساب‌های شهروندی واریز می‌شود. \\
\hline
۵ & شهروندان آزادند مبلغ دریافتی را مصرف یا سرمایه‌گذاری کنند. \\
\hline
۶ & دولت هزینه‌های عمومی را از طریق مالیات‌ستانی از شهروندان و بنگاه‌ها تأمین می‌کند. \\
\hline
\end{tabular}
\end{table}


\subsection{چرا تمام عواید نفت به مردم؟}

یک پرسش رایج این است: چرا همهٔ عواید نفت را به مردم بدهیم؟ بهتر نیست بخشی به مردم و بخشی به بنگاه‌های اقتصادی داده شود؟

پاسخ ناظران شفاف است:

\begin{quote}
وقتی عواید نفتی را بین مردم توزیع می‌کنیم، بخشی را مصرف می‌کنند و بخشی را سرمایه‌گذاری می‌کنند.
\textbf{پولی که مصرف می‌کنند، می‌شود درآمد بنگاه‌ها.}
پولی که در سهام سرمایه‌گذاری می‌کنند، می‌شود سرمایهٔ بیشتر برای بنگاه‌ها.
پولی هم که در بانک می‌گذارند، می‌شود منابع بیشتر برای وام بنگاه‌ها.
پس در نهایت، پولی که بین مردم توزیع می‌شود، به بنگاه‌های اقتصادی می‌رسد ---
منتها آن بنگاه‌هایی که مردم به‌خاطر محصول بهتر، یا فرصت سرمایه‌گذاری مناسب‌تر، انتخاب کرده‌اند.
برعکس، اگر قرار باشد دولت بنگاه‌های دریافت‌کننده را انتخاب کند، لاجرم عرصهٔ فساد باز می‌شود
و ثروت نفتی حیف‌ومیل می‌شود.
\end{quote}


% ---------------------------------------------------------------------------
\section{تحلیل اثرات اقتصادی}
\label{sec:economic-effects}

\subsection{اثر بر تورم}

یکی از نگرانی‌های اصلی، تورم‌زا بودن توزیع نقدی است. ناظران استدلال می‌کند:

در بلندمدت، تورم صرفاً ناشی از افزایش حجم پول است.
در مدل پیشنهادی:
\begin{itemize}
\item ارز حاصل از نفت صادراتی فروخته می‌شود.
\item ریال از دست خریداران ارز گرفته می‌شود.
\item همان ریال بین مردم توزیع می‌شود.
\item حجم پول افزایش نمی‌یابد $\Rightarrow$ تورم‌زا نیست.
\end{itemize}

\noindent
نکتهٔ کلیدی آن است که در سیستم فعلی نیز ارز نفتی به ریال تبدیل می‌شود.
تفاوت اصلی در آن است که \emph{چه کسی} ریال دریافت می‌کند: دولت یا شهروندان.
اگر دولت ریال دریافت کند و آن را هزینه کند، اثر تورمی آن کمتر از حالتی نیست
که شهروندان مستقیماً دریافت‌کننده باشند.
بلکه دقیقاً برعکس، هزینه‌کرد دولتی به‌دلیل ناکارآمدی و فساد، معمولاً تورم‌زاتر است.

\subsection{اثر بر نسل‌های آینده}

پرسش دیگر آن است که اگر همهٔ عواید نفتی به نسل حاضر داده شود، نسل‌های آینده محروم نمی‌شوند؟

\begin{itemize}
\item بخشی از عواید توزیع‌شده توسط شهروندان پس‌انداز و سرمایه‌گذاری می‌شود
(خرید سهام، مسکن، تأسیس کسب‌وکار، سرمایه‌گذاری در تحصیل فرزندان).
\item تجربهٔ صندوق توسعهٔ ملی ایران نشان داد که ادعای ذخیره‌سازی بین‌نسلی توسط دولت
ممکن نیست: صندوق عملاً به «قلک دولت» تبدیل شد.
\item پدر و مادرها طبیعتاً دلسوزتر از بوروکرات‌ها نسبت به سرنوشت فرزندانشان هستند.
\end{itemize}

\subsection{اثر بر ساختار حکمرانی}

\begin{itemize}
\item \textbf{تقویت پاسخگویی دولت:}
وقتی دولت مجبور شود هزینه‌های خود را از محل مالیات شهروندان تأمین کند،
انگیزهٔ شهروندان برای مطالبهٔ شفافیت و پاسخگویی به‌شدت افزایش می‌یابد.

\item \textbf{کاهش فساد ساختاری:}
حذف اختیار تخصیص منابع نفتی از دولت، بزرگ‌ترین منبع رانت و فساد را خشکانده
و فضای رقابت سالم اقتصادی را فراهم می‌کند.

\item \textbf{تقویت نهادهای مالیاتی:}
نیاز به مالیات‌ستانی کارآمد، الزامی برای ایجاد نظام مالیاتی مدرن، شفاف و عادلانه ایجاد می‌کند.
\end{itemize}


% ---------------------------------------------------------------------------
\section{تجربیات بین‌المللی}
\label{sec:international-experiences}

\subsection{مدل آلاسکا: صندوق دائمی}

ایالت آلاسکای آمریکا از سال ۱۹۸۲ بخشی از درآمدهای نفتی خود را از طریق
\lr{Alaska Permanent Fund Dividend} (PFD) مستقیماً بین ساکنان توزیع می‌کند.

\begin{itemize}
\item هر ساکن آلاسکا (اعم از بزرگسال و کودک) سالانه مبلغی بین ۱۰۰۰ تا ۳۰۰۰ دلار دریافت می‌کند.
\item منابع صندوق به‌صورت متنوع در بازارهای مالی جهانی سرمایه‌گذاری می‌شود.
\item این مدل به کاهش نابرابری در آلاسکا و افزایش اعتماد عمومی به مدیریت منابع طبیعی کمک کرده است.
\end{itemize}

\subsection{مدل نروژ: صندوق بازنشستگی دولتی جهانی}

صندوق ثروت ملی نروژ (\lr{Government Pension Fund Global}) با بیش از ۱.۵ تریلیون دلار دارایی،
بزرگ‌ترین صندوق ثروت ملی جهان است.

تفاوت‌های کلیدی:
\begin{itemize}
\item نروژ عواید نفتی را در بازارهای جهانی سرمایه‌گذاری می‌کند و تنها بازدهٔ سرمایه‌گذاری
(حدود ۳ درصد سالانه) را هزینه می‌کند.
\item این مدل به‌دلیل وجود نهادهای قوی دموکراتیک و حاکمیت خوب قابل اجراست.
\item ناظران تأکید دارد که در ایران، به‌دلیل ضعف نهادهای حکمرانی، مدل نروژ عملی نیست
و مدل توزیع مستقیم (نزدیک‌تر به آلاسکا) مناسب‌تر است.
\end{itemize}

\subsection{پیشنهادهای مشابه در عراق و سایر کشورها}

\begin{itemize}
\item \textbf{عراق:}
اقتصاددانانی مانند تامس پلی (\lr{Thomas Palley}) طرح «صندوق توزیع درآمد شهروندی»
(\lr{Citizen Revenue Distribution Fund}) را برای عراق پیشنهاد کرده‌اند.

\item \textbf{مغولستان و تیمور شرقی:}
هر دو کشور تجربهٔ محدودی در توزیع مستقیم درآمدهای منابع طبیعی به شهروندان داشته‌اند.
\end{itemize}


% ---------------------------------------------------------------------------
\section{نقشهٔ راه پیاده‌سازی}
\label{sec:implementation-roadmap}

\subsection{فاز اول: مقدمات قانونی و نهادی (سال اول)}

\begin{enumerate}
\item تصویب قانون \textbf{«مالکیت شهروندی منابع طبیعی»} که مالکیت ذاتی هر شهروند ایرانی
بر سهمی مساوی از منابع طبیعی را به رسمیت بشناسد.

\item تأسیس \textbf{صندوق سهام ملی منابع طبیعی} به‌عنوان نهادی مستقل از دولت،
با هیئت‌امنای منتخب و ساختار نظارتی شفاف.

\item ایجاد \textbf{سامانهٔ ملی حساب‌های شهروندی} با استفاده از زیرساخت‌های بانکی و فناوری اطلاعات.

\item اصلاح قوانین مالیاتی برای تأمین هزینه‌های دولت از محل مالیات
(مالیات بر درآمد، مالیات بر ارزش افزوده، مالیات بر دارایی).
\end{enumerate}

\subsection{فاز دوم: اجرای آزمایشی (سال‌های دوم و سوم)}

\begin{enumerate}
\item آغاز توزیع با درصد محدود (مثلاً ۲۰ تا ۳۰ درصد از عواید نفتی).
\item ارزیابی اثرات اقتصادی و اجتماعی.
\item اصلاح مکانیسم‌ها بر اساس بازخورد.
\item تقویت ظرفیت مالیات‌ستانی دولت.
\end{enumerate}

\subsection{فاز سوم: اجرای کامل (سال‌های چهارم به بعد)}

\begin{enumerate}
\item افزایش تدریجی سهم توزیع مستقیم تا ۱۰۰ درصد عواید منابع طبیعی.
\item حذف کامل وابستگی بودجهٔ دولت به درآمد نفت.
\item استقرار کامل نظام مالیاتی مدرن.
\item گسترش دامنه به سایر منابع طبیعی (معادن، جنگل‌ها، منابع دریایی).
\end{enumerate}


% ---------------------------------------------------------------------------
\section{برآورد اولیه سهم هر شهروند}
\label{sec:per-capita-estimation}

\begin{table}[htbp]
\centering
\caption{برآورد سناریوهای توزیع درآمد نفتی بین شهروندان (ارقام تقریبی)}
\label{tab:revenue-scenarios}
\begin{tabular}{|l|c|c|c|}
\hline
\textbf{سناریو} & \textbf{درآمد نفتی سالانه} & \textbf{جمعیت} & \textbf{سهم سالانه هر نفر} \\
\hline
محافظه‌کارانه & ۳۰ میلیارد دلار & ۸۸ میلیون & حدود ۳۴۰ دلار \\
\hline
متوسط & ۵۰ میلیارد دلار & ۸۸ میلیون & حدود ۵۷۰ دلار \\
\hline
خوش‌بینانه & ۸۰ میلیارد دلار & ۸۸ میلیون & حدود ۹۱۰ دلار \\
\hline
\end{tabular}
\end{table}

\noindent
توجه: ارقام فوق شامل تمامی منابع طبیعی (نفت، گاز، میعانات، معادن) می‌شود
و بر اساس سناریوهای مختلف قیمت نفت و حجم صادرات محاسبه شده است.
برای یک خانوادهٔ ۴ نفره در سناریوی متوسط، این رقم حدود ۲۳۰۰ دلار در سال خواهد بود.


% ---------------------------------------------------------------------------
\section{پیش‌نیازها و چالش‌های اجرایی}
\label{sec:challenges}

\subsection{پیش‌نیازهای ضروری}

\begin{enumerate}
\item \textbf{نظام مالیاتی مدرن و کارآمد:}
بزرگ‌ترین پیش‌نیاز، اصلاح بنیادین نظام مالیاتی است.
نظام مالیاتی فعلی ایران عمدتاً کارمندمحور است
و بخش‌های بزرگی از اقتصاد (بازار، نهادهای شبه‌دولتی، اقتصاد غیررسمی) از پرداخت مالیات معاف‌اند.

\item \textbf{شفافیت مالی دولت:}
بودجه و هزینه‌کرد دولت باید کاملاً شفاف و قابل نظارت باشد.

\item \textbf{استقلال بانک مرکزی:}
بانک مرکزی باید از فشار دولت برای خلق پول مستقل شود.

\item \textbf{زیرساخت‌های فناوری اطلاعات:}
سامانه‌های دقیق شناسایی شهروندان و حساب‌های بانکی.

\item \textbf{ارادهٔ سیاسی:}
حذف دسترسی دولت به درآمدهای نفتی، قدرت و اختیار عظیمی را از نهادهای حکومتی سلب می‌کند
و طبعاً با مقاومت سیاسی مواجه خواهد شد.
\end{enumerate}

\subsection{چالش‌های بالقوه}

\begin{enumerate}
\item \textbf{مقاومت نهادهای ذی‌نفع:}
نهادها و گروه‌هایی که از رانت نفتی بهره می‌برند (از جمله نهادهای شبه‌دولتی، واسطه‌گران ارزی،
و بنگاه‌هایی که از یارانهٔ نهاده‌ها استفاده می‌کنند) به‌شدت مقاومت خواهند کرد.

\item \textbf{دورهٔ گذار:}
در دورهٔ گذار ممکن است دولت با کسری بودجه مواجه شود تا نظام مالیاتی جدید تثبیت شود.

\item \textbf{اثرات توزیعی:}
بنگاه‌هایی که به یارانهٔ انرژی و ارز ترجیحی وابسته‌اند ممکن است در کوتاه‌مدت آسیب ببینند.

\item \textbf{تحریم‌ها:}
تحریم‌های بین‌المللی، درآمد نفتی ایران را کاهش داده و مکانیسم‌های مالی را پیچیده کرده است.
\end{enumerate}


% ---------------------------------------------------------------------------
\section{ارتباط با سایر اصلاحات اقتصادی}
\label{sec:related-reforms}

طرح توزیع مستقیم درآمد نفتی، یک اصلاح منفرد نیست بلکه بخشی از بستهٔ جامع اصلاحات اقتصادی است:

\begin{description}
\item[حذف ارز ترجیحی:]
با توزیع مستقیم عواید ارزی بین مردم، دیگر نیازی به ارز ترجیحی نیست.
مردم خود قادر خواهند بود کالاهای مورد نیاز خود را با قیمت واقعی بخرند.

\item[اصلاح قیمت حامل‌های انرژی:]
آزادسازی قیمت سوخت و انرژی در کنار توزیع نقدی مستقیم،
اثرات رفاهی منفی بر اقشار کم‌درآمد را خنثی می‌کند.

\item[اصلاح نظام بانکی:]
حذف تسهیلات تکلیفی و ارزان‌قیمت دولتی، نظام بانکی را به سمت
واسطه‌گری مالی سالم سوق می‌دهد.

\item[اصلاح نظام مالیاتی:]
\begin{itemize}
\item گسترش پایهٔ مالیاتی به تمامی فعالیت‌های اقتصادی.
\item مالیات بر مجموع درآمد (نه فقط حقوق کارمندان).
\item مالیات بر دارایی و ثروت.
\item تقویت مالیات بر ارزش افزوده.
\end{itemize}

\item[خصوصی‌سازی واقعی:]
بنگاه‌های دولتی و شبه‌دولتی باید واقعاً به بخش خصوصی واگذار شوند
(نه به نهادهای شبه‌دولتی دیگر).
\end{description}


% ---------------------------------------------------------------------------
\section{جمع‌بندی و توصیه‌های سیاستی}
\label{sec:conclusion-oil-chapter}

طرح توزیع مستقیم درآمدهای منابع طبیعی بین شهروندان، یک تغییر پارادایمی
در رابطهٔ دولت--ملت--نفت در ایران است.
این طرح هم‌زمان چندین هدف را محقق می‌کند:

\begin{enumerate}
\item \textbf{عدالت اقتصادی:}
هر شهروند سهم مساوی از ثروت ملی دریافت می‌کند.

\item \textbf{کاهش فساد:}
بزرگ‌ترین منبع رانت اقتصادی حذف می‌شود.

\item \textbf{تقویت دموکراسی:}
دولت مالیات‌محور ذاتاً پاسخگوتر از دولت نفت‌محور است.

\item \textbf{کارآمدی اقتصادی:}
تخصیص منابع بر اساس تصمیم‌های میلیون‌ها شهروند، کارآمدتر از تخصیص بوروکراتیک است.

\item \textbf{ثبات اقتصاد کلان:}
قطع وابستگی بودجهٔ دولت به نوسانات قیمت نفت.
\end{enumerate}


% ---------------------------------------------------------------------------
\section{مراجع و منابع}
\label{sec:references-oil-chapter}

\begin{thebibliography}{99}

\bibitem{nazaran-daraian}
ناظران، پویا.
«مزایا و معایب ملی کردن نفت».
\textit{دارایان -- داشته‌های اقتصادی ایران}،
\lr{daraian.com/fa/paper/25431}، ۲۰۲۱.

\bibitem{ross2012}
\lr{Ross, Michael L.}
\textit{\lr{The Oil Curse: How Petroleum Wealth Shapes the Development of Nations}}.
\lr{Princeton University Press}, 2012.

\bibitem{palley2003}
\lr{Palley, Thomas I.}
``\lr{Combating the Natural Resource Curse with Citizen Revenue Distribution Funds: Oil and the Case of Iraq}.''
\textit{\lr{Foreign Policy in Focus}}, 2003.

\bibitem{bagherzadeh2002}
\lr{Bagherzadeh, H.}
``\lr{Tarhe Democratizeh Kardan e Sanat Naft dar Iran / Proposal for Democratization of Oil Industry in Iran}.''
\textit{\lr{Iran-e Emruz}}, 2002.

\bibitem{alaska-pfd}
\lr{Alaska Permanent Fund Corporation.}
\textit{\lr{An Alaskan's Guide to the Permanent Fund}}.
\url{https://apfc.org}.

\bibitem{norway-gpfg}
\lr{Norges Bank Investment Management.}
\textit{\lr{Government Pension Fund Global}}.
\url{https://www.nbim.no}.

\bibitem{cambridge-iran-oil}
``\lr{Oil, Economic Diversification and the Democratic Process in Iran}.''
\textit{\lr{Iranian Studies}}, \lr{Cambridge University Press}, 2022.

\bibitem{nazaran-eghtesadnews}
ناظران، پویا.
مقالات متعدد در \textit{دنیای اقتصاد}.
\url{https://donya-e-eqtesad.com}.

\end{thebibliography}
