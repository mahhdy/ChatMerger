\chapter{تحریم‌های بین‌المللی و تحول ساختار فساد در ایران}
\label{chap:sanctions-corruption}

\begin{abstract}
این فصل به بررسی نظام‌مند تحریم‌های بین‌المللی علیه ایران و اثرات آن بر 
شاخص‌های اقتصادی، ساختار فساد و ظهور بازیگران جدید اقتصادی می‌پردازد. 
با استفاده از داده‌های کمّی و تحلیل‌های کیفی، نشان داده می‌شود که 
تحریم‌ها نه‌تنها به اهداف اعلام‌شده نرسیده‌اند، بلکه به نهادینه شدن 
فساد، ظهور شبکه‌های رانتی جدید و تضعیف بخش خصوصی واقعی منجر شده‌اند.
\end{abstract}

%═══════════════════════════════════════════════════════════════════════════════
\section{مقدمه}
\label{sec:sanctions-intro}
%═══════════════════════════════════════════════════════════════════════════════

تحریم‌های اقتصادی به عنوان ابزار سیاست خارجی، پدیده‌ای است که در قرن 
بیستم گسترش یافت و در قرن بیست‌ویکم به اوج خود رسید. ایران یکی از 
تحریم‌شده‌ترین کشورهای جهان است که تجربه منحصربه‌فردی از تحریم‌های 
چندلایه، طولانی‌مدت و فراگیر دارد.

\begin{tcolorbox}[colback=red!5!white,colframe=red!75!black,
    title=پرسش‌های پژوهشی این فصل]
\begin{enumerate}
    \item تحریم‌ها چه تأثیری بر شاخص‌های کلان اقتصادی ایران داشته‌اند؟
    \item آیا تحریم‌ها به کاهش یا افزایش فساد منجر شده‌اند؟
    \item چه بازیگران جدیدی در نتیجه تحریم‌ها ظهور کرده‌اند؟
    \item مکانیزم‌های دور زدن تحریم چگونه به ساختار فساد دامن زده‌اند؟
    \item چه گروه‌هایی از تحریم‌ها منتفع و چه گروه‌هایی متضرر شده‌اند؟
\end{enumerate}
\end{tcolorbox}

\subsection{چارچوب نظری}

\subsubsection{نظریه‌های اثربخشی تحریم}

ادبیات آکادمیک درباره اثربخشی تحریم‌ها به دو دسته کلی تقسیم می‌شود:

\paragraph{دیدگاه خوش‌بینانه (Liberal Institutionalism)}
\begin{itemize}
    \item تحریم‌ها از طریق فشار اقتصادی، رفتار دولت هدف را تغییر می‌دهند
    \item هزینه‌های اقتصادی منجر به نارضایتی عمومی و فشار برای تغییر می‌شود
    \item تحریم‌های «هوشمند» می‌توانند نخبگان را هدف قرار دهند
\end{itemize}

\paragraph{دیدگاه بدبینانه (Realism \& Critical Studies)}
\begin{itemize}
    \item تحریم‌ها به ندرت به اهداف خود می‌رسند \parencite{pape1997sanctions}
    \item هزینه‌ها عمدتاً بر مردم عادی تحمیل می‌شود
    \item تحریم‌ها می‌توانند رژیم را تقویت کنند («اثر تجمع حول پرچم»)
    \item ایجاد اقتصاد موازی و تقویت فساد
\end{itemize}

\subsubsection{نظریه دولت رانتیر و تحریم}

\begin{quote}
\textit{«در دولت‌های رانتیر تحریم‌شده، منابع رانتی محدود شده به کانال‌های 
غیررسمی منتقل می‌شود. این امر به جای تضعیف نخبگان حاکم، آن‌ها را 
قدرتمندتر می‌کند زیرا کنترل کانال‌های جایگزین را در دست می‌گیرند.»}
\parencite{dizaji2019sanctions}
\end{quote}

%═══════════════════════════════════════════════════════════════════════════════
\section{تاریخچه و طبقه‌بندی تحریم‌های ایران}
\label{sec:sanctions-history}
%═══════════════════════════════════════════════════════════════════════════════

\subsection{سیر تاریخی تحریم‌ها}

\begin{table}[htbp]
\centering
\caption{گاه‌شمار تحریم‌های کلیدی علیه ایران (۱۹۷۹-۲۰۲۴)}
\label{tab:sanctions-timeline}
\small
\begin{tabular}{|c|p{3cm}|p{4cm}|p{4.5cm}|}
\hline
\textbf{سال} & \textbf{رویداد/مرجع} & \textbf{نوع تحریم} & \textbf{دلیل اعلام‌شده} \\
\hline
\multicolumn{4}{|c|}{\cellcolor{gray!20}\textbf{دوره اول: بحران گروگان‌گیری (۱۹۷۹-۱۹۸۱)}} \\
\hline
۱۹۷۹ & فرمان اجرایی ۱۲۱۷۰ & مسدودسازی دارایی‌ها & بحران گروگان‌گیری \\
\hline
۱۹۸۰ & فرمان ۱۲۲۰۵ & تحریم تجاری کامل & ادامه بحران \\
\hline
۱۹۸۱ & توافق الجزایر & رفع بخشی & آزادی گروگان‌ها \\
\hline
\multicolumn{4}{|c|}{\cellcolor{gray!20}\textbf{دوره دوم: تحریم‌های تروریسم (۱۹۸۴-۱۹۹۵)}} \\
\hline
۱۹۸۴ & وزارت خارجه آمریکا & لیست حامیان تروریسم & حمایت از حزب‌الله \\
\hline
۱۹۸۷ & فرمان ۱۲۶۱۳ & ممنوعیت واردات کالاهای ایرانی & --- \\
\hline
\multicolumn{4}{|c|}{\cellcolor{gray!20}\textbf{دوره سوم: تحریم‌های فراسرزمینی (۱۹۹۵-۲۰۰۶)}} \\
\hline
۱۹۹۵ & فرمان ۱۲۹۵۷ & ممنوعیت سرمایه‌گذاری نفتی & تروریسم + برنامه هسته‌ای \\
\hline
۱۹۹۶ & قانون داماتو (ILSA) & \textbf{تحریم ثانویه} شرکت‌های خارجی & --- \\
\hline
\multicolumn{4}{|c|}{\cellcolor{gray!20}\textbf{دوره چهارم: تحریم‌های هسته‌ای (۲۰۰۶-۲۰۱۵)}} \\
\hline
۲۰۰۶ & قطعنامه ۱۷۳۷ UNSC & اولین تحریم سازمان ملل & برنامه هسته‌ای \\
\hline
۲۰۰۷ & قطعنامه ۱۷۴۷ & گسترش تحریم‌های UN & --- \\
\hline
۲۰۰۸ & قطعنامه ۱۸۰۳ & محدودیت‌های مالی & --- \\
\hline
۲۰۱۰ & قطعنامه ۱۹۲۹ & \textbf{شدیدترین تحریم UN} & --- \\
\hline
۲۰۱۰ & CISADA (آمریکا) & تحریم بانک مرکزی & --- \\
\hline
۲۰۱۲ & NDAA Sec. 1245 & تحریم نفتی گسترده & --- \\
\hline
۲۰۱۲ & تحریم‌های اتحادیه اروپا & \textbf{تحریم نفت + SWIFT} & --- \\
\hline
\multicolumn{4}{|c|}{\cellcolor{gray!20}\textbf{دوره پنجم: برجام و پسابرجام (۲۰۱۵-۲۰۲۴)}} \\
\hline
۲۰۱۶ & اجرای برجام & رفع تحریم‌های هسته‌ای & توافق هسته‌ای \\
\hline
۲۰۱۸ & خروج ترامپ از برجام & \textbf{بازگشت همه تحریم‌ها} & --- \\
\hline
۲۰۱۹ & تحریم آیت‌الله خامنه‌ای & تحریم رهبری & --- \\
\hline
۲۰۲۰ & Snapback (ناموفق) & تلاش برای بازگشت تحریم‌های UN & --- \\
\hline
۲۰۲۲ & تحریم‌های حقوق بشری & تحریم اشخاص جدید & اعتراضات ۱۴۰۱ \\
\hline
۲۰۲۴ & تحریم‌های پهپادی & تحریم صنایع دفاعی & همکاری با روسیه \\
\hline
\end{tabular}
\end{table}

\subsection{طبقه‌بندی تحریم‌ها بر اساس مرجع صادرکننده}

\begin{figure}[htbp]
\centering
\begin{tikzpicture}[
    level 1/.style={sibling distance=5cm, level distance=2.5cm},
    level 2/.style={sibling distance=2.5cm, level distance=2cm},
    every node/.style={rectangle, draw, align=center, font=\small, 
        minimum width=2cm}
]
\node[fill=red!30] {تحریم‌های ایران}
    child {node[fill=blue!20] {چندجانبه}
        child {node[fill=blue!10] {سازمان ملل\\(۶ قطعنامه)}}
        child {node[fill=blue!10] {اتحادیه اروپا\\(متعدد)}}
    }
    child {node[fill=orange!20] {یک‌جانبه آمریکا}
        child {node[fill=orange!10] {خزانه‌داری\\(OFAC)}}
        child {node[fill=orange!10] {وزارت خارجه}}
        child {node[fill=orange!10] {کنگره}}
    }
    child {node[fill=green!20] {ثانویه}
        child {node[fill=green!10] {شرکت‌های\\خارجی}}
        child {node[fill=green!10] {بانک‌های\\خارجی}}
    }
;
\end{tikzpicture}
\caption{ساختار مراجع تحریم‌کننده ایران}
\label{fig:sanctions-sources}
\end{figure}

\subsection{طبقه‌بندی بر اساس نوع و هدف}

\begin{table}[htbp]
\centering
\caption{انواع تحریم‌ها بر اساس حوزه هدف}
\label{tab:sanctions-types}
\begin{tabular}{|l|p{4cm}|p{4cm}|c|}
\hline
\textbf{نوع} & \textbf{شرح} & \textbf{مثال} & \textbf{شدت} \\
\hline
\multicolumn{4}{|c|}{\cellcolor{red!10}\textbf{تحریم‌های مالی}} \\
\hline
بانک مرکزی & مسدودسازی دارایی‌ها & تحریم CBI & ●●●●● \\
\hline
قطع SWIFT & قطع ارتباط بانکی بین‌المللی & ۲۰۱۲ و ۲۰۱۸ & ●●●●● \\
\hline
بانک‌های تجاری & تحریم بانک‌های خاص & بانک ملت، سپه & ●●●●○ \\
\hline
\multicolumn{4}{|c|}{\cellcolor{orange!10}\textbf{تحریم‌های نفتی}} \\
\hline
صادرات نفت & محدودیت فروش & EU Oil Ban 2012 & ●●●●● \\
\hline
بیمه نفتکش & ممنوعیت بیمه حمل & تحریم P\&I Clubs & ●●●●○ \\
\hline
سرمایه‌گذاری & ممنوعیت سرمایه‌گذاری & ILSA/ISA & ●●●○○ \\
\hline
\multicolumn{4}{|c|}{\cellcolor{yellow!10}\textbf{تحریم‌های تجاری}} \\
\hline
تکنولوژی دوگانه & ممنوعیت صادرات & لیست EAR & ●●●●○ \\
\hline
فلزات & آهن، فولاد، آلومینیوم & ۲۰۱۹ & ●●●○○ \\
\hline
پتروشیمی & محصولات پتروشیمی & ۲۰۱۹ & ●●●○○ \\
\hline
\multicolumn{4}{|c|}{\cellcolor{green!10}\textbf{تحریم‌های هدفمند (اشخاص)}} \\
\hline
افراد & مسدودسازی دارایی، ممنوعیت سفر & لیست SDN & ●●○○○ \\
\hline
نهادها & سپاه، بنیادها & تحریم IRGC & ●●●●○ \\
\hline
\end{tabular}
\end{table}

\subsection{مقایسه شدت تحریم‌ها در ادوار مختلف}

\begin{figure}[htbp]
\centering
\begin{tikzpicture}
\begin{axis}[
    xlabel={سال},
    ylabel={شاخص شدت تحریم (۰-۱۰۰)},
    xmin=1980, xmax=2024,
    ymin=0, ymax=100,
    xtick={1980,1985,1990,1995,2000,2005,2010,2015,2020,2024},
    ytick={0,20,40,60,80,100},
    legend pos=north west,
    ymajorgrids=true,
    grid style=dashed,
    width=14cm,
    height=8cm,
    area style,
]

% خط شدت تحریم
\addplot[thick, color=red, mark=*, mark size=1.5pt] coordinates {
    (1979,45) (1980,50) (1981,20) (1984,25) (1987,30) 
    (1995,40) (1996,45) (2000,45) (2005,50) (2006,55)
        (2007,60) (2008,65) (2010,75) (2011,80) (2012,95)
    (2013,95) (2014,90) (2015,85) (2016,40) (2017,35)
    (2018,70) (2019,95) (2020,98) (2021,98) (2022,95) (2023,92) (2024,90)
};
\addlegendentry{شاخص شدت تحریم}

% رویدادهای کلیدی
\draw[dashed, gray] (axis cs:2012,0) -- (axis cs:2012,100);
\draw[dashed, gray] (axis cs:2016,0) -- (axis cs:2016,100);
\draw[dashed, gray] (axis cs:2018,0) -- (axis cs:2018,100);

\node[rotate=90, font=\tiny] at (axis cs:2012,50) {تحریم نفتی EU};
\node[rotate=90, font=\tiny] at (axis cs:2016,50) {اجرای برجام};
\node[rotate=90, font=\tiny] at (axis cs:2018,50) {خروج ترامپ};

\end{axis}
\end{tikzpicture}
\caption{روند شدت تحریم‌های بین‌المللی علیه ایران (۱۹۷۹-۲۰۲۴)}
\label{fig:sanctions-intensity}
\end{figure}

%═══════════════════════════════════════════════════════════════════════════════
\section{اثرات اقتصاد کلان تحریم‌ها}
\label{sec:macro-effects}
%═══════════════════════════════════════════════════════════════════════════════

\subsection{تولید ناخالص داخلی}

\begin{figure}[htbp]
\centering
\begin{tikzpicture}
\begin{axis}[
    xlabel={سال},
    ylabel={تولید ناخالص داخلی (میلیارد دلار)},
    xmin=2005, xmax=2023,
    ymin=0, ymax=700,
    xtick={2005,2008,2011,2014,2017,2020,2023},
    legend pos=north east,
    ymajorgrids=true,
    grid style=dashed,
    width=14cm,
    height=7cm,
]

% GDP واقعی
\addplot[thick, color=blue, mark=square*, mark size=2pt] coordinates {
    (2005,192) (2006,222) (2007,286) (2008,356) (2009,366)
    (2010,422) (2011,576) (2012,502) (2013,493) (2014,425)
    (2015,385) (2016,412) (2017,445) (2018,454) (2019,350)
    (2020,250) (2021,280) (2022,390) (2023,402)
};
\addlegendentry{GDP اسمی (دلاری)}

% GDP بدون تحریم (تخمین)
\addplot[thick, color=green!60!black, dashed, mark=triangle*, mark size=2pt] coordinates {
    (2005,192) (2006,230) (2007,300) (2008,380) (2009,400)
    (2010,480) (2011,620) (2012,680) (2013,740) (2014,800)
    (2015,850) (2016,900) (2017,960) (2018,1020) (2019,1080)
    (2020,1100) (2021,1150) (2022,1200) (2023,1250)
};
\addlegendentry{GDP تخمینی بدون تحریم}

% ناحیه زیان
\addplot[fill=red, fill opacity=0.2] coordinates {
    (2012,502) (2013,493) (2014,425) (2015,385) (2016,412)
    (2017,445) (2018,454) (2019,350) (2020,250) (2021,280)
    (2022,390) (2023,402) (2023,1250) (2022,1200) (2021,1150)
    (2020,1100) (2019,1080) (2018,1020) (2017,960) (2016,900)
    (2015,850) (2014,800) (2013,740) (2012,680)
} \closedcycle;

\end{axis}
\end{tikzpicture}
\caption{مقایسه GDP واقعی و تخمینی بدون تحریم (ناحیه قرمز: زیان تحریم)}
\label{fig:gdp-comparison}
\end{figure}

\begin{table}[htbp]
\centering
\caption{برآورد زیان اقتصادی تحریم‌ها (میلیارد دلار)}
\label{tab:economic-loss}
\begin{tabular}{|l|r|r|r|p{4cm}|}
\hline
\textbf{دوره} & \textbf{GDP تجمعی واقعی} & \textbf{GDP تخمینی} & \textbf{زیان} & \textbf{منبع} \\
\hline
۲۰۱۲-۲۰۱۵ & ۱,۸۰۵ & ۳,۰۷۰ & ۱,۲۶۵ & \cite{ifo2016iran} \\
\hline
۲۰۱۸-۲۰۲۳ & ۲,۱۲۶ & ۶,۷۳۰ & ۴,۶۰۴ & برآورد نویسنده \\
\hline
\textbf{کل (۲۰۱۲-۲۰۲۳)} & ۴,۵۸۱ & ۱۰,۴۵۰ & \textbf{۵,۸۶۹} & --- \\
\hline
\end{tabular}
\end{table}

\subsection{صادرات نفت}

\begin{figure}[htbp]
\centering
\begin{tikzpicture}
\begin{axis}[
    ybar,
    xlabel={سال},
    ylabel={صادرات نفت (میلیون بشکه در روز)},
    symbolic x coords={2011,2012,2013,2014,2015,2016,2017,2018,2019,2020,2021,2022,2023},
    xtick=data,
    x tick label style={rotate=45, anchor=east},
    nodes near coords,
    nodes near coords align={vertical},
    nodes near coords style={font=\tiny},
    width=15cm,
    height=7cm,
    bar width=0.5cm,
    ymin=0,
    ymax=3,
    legend pos=north east,
]
\addplot[fill=blue!60] coordinates {
    (2011,2.5) (2012,2.2) (2013,1.1) (2014,1.2) (2015,1.1)
    (2016,2.1) (2017,2.3) (2018,1.9) (2019,0.5) (2020,0.4)
    (2021,0.6) (2022,1.0) (2023,1.4)
};
\addlegendentry{صادرات نفت خام}
\end{axis}
\end{tikzpicture}
\caption{روند صادرات نفت خام ایران (میلیون بشکه در روز)}
\label{fig:oil-exports}
\end{figure}

\subsection{نرخ ارز و تورم}

\begin{figure}[htbp]
\centering
\begin{tikzpicture}
\begin{axis}[
    xlabel={سال},
    ylabel={نرخ دلار (تومان) - مقیاس لگاریتمی},
    xmin=2010, xmax=2024,
    ymode=log,
    log basis y={10},
    ymin=1000, ymax=100000,
    xtick={2010,2012,2014,2016,2018,2020,2022,2024},
    legend pos=north west,
    ymajorgrids=true,
    grid style=dashed,
    width=14cm,
    height=7cm,
]

\addplot[thick, color=red, mark=*, mark size=2pt] coordinates {
    (2010,1050) (2011,1200) (2012,2500) (2013,3000) (2014,3200)
    (2015,3400) (2016,3600) (2017,4000) (2018,14000) (2019,13000)
    (2020,25000) (2021,28000) (2022,32000) (2023,50000) (2024,60000)
};
\addlegendentry{نرخ بازار آزاد}

\addplot[thick, color=blue, dashed, mark=square*, mark size=2pt] coordinates {
    (2010,1050) (2011,1200) (2012,1226) (2013,2500) (2014,2700)
    (2015,3000) (2016,3200) (2017,3600) (2018,4200) (2019,4200)
    (2020,4200) (2021,4200) (2022,4200) (2023,28500) (2024,28500)
};
\addlegendentry{نرخ رسمی/نیمایی}

% شوک‌ها
\draw[dashed, gray, thick] (axis cs:2012,1000) -- (axis cs:2012,100000);
\draw[dashed, gray, thick] (axis cs:2018,1000) -- (axis cs:2018,100000);

\node[font=\tiny, rotate=90] at (axis cs:2012.3,5000) {شوک ۲۰۱۲};
\node[font=\tiny, rotate=90] at (axis cs:2018.3,30000) {شوک ۲۰۱۸};

\end{axis}
\end{tikzpicture}
\caption{روند نرخ ارز رسمی و آزاد (مقیاس لگاریتمی)}
\label{fig:exchange-rate}
\end{figure}

\begin{table}[htbp]
\centering
\caption{شاخص‌های کلیدی اقتصاد ایران در دوره‌های مختلف تحریم}
\label{tab:macro-indicators}
\begin{tabular}{|l|c|c|c|c|c|}
\hline
\textbf{شاخص} & \textbf{۲۰۱۰} & \textbf{۲۰۱۲} & \textbf{۲۰۱۶} & \textbf{۲۰۱۹} & \textbf{۲۰۲۳} \\
\hline
\multicolumn{6}{|c|}{\cellcolor{gray!20}\textbf{شاخص‌های کلان}} \\
\hline
رشد GDP (٪) & ۶.۶ & $-$۶.۶ & ۱۳.۴ & $-$۶.۸ & ۴.۷ \\
\hline
تورم (٪) & ۱۲.۴ & ۳۰.۵ & ۹.۰ & ۴۱.۱ & ۴۲.۵ \\
\hline
بیکاری (٪) & ۱۳.۵ & ۱۲.۱ & ۱۲.۴ & ۱۰.۷ & ۸.۴ \\
\hline
\multicolumn{6}{|c|}{\cellcolor{gray!20}\textbf{بخش خارجی}} \\
\hline
صادرات نفت (M bpd) & ۲.۴ & ۱.۵ & ۲.۱ & ۰.۵ & ۱.۴ \\
\hline
درآمد نفتی (B\$) & ۷۳ & ۶۹ & ۴۱ & ۱۶ & ۵۳ \\
\hline
ذخایر ارزی (B\$) & ۸۰ & ۶۰ & ۱۲۸ & ۸۵ & ۴۰؟ \\
\hline
\multicolumn{6}{|c|}{\cellcolor{gray!20}\textbf{شاخص‌های رفاهی}} \\
\hline
نرخ فقر (٪) & ۱۲ & ۱۵ & ۱۱ & ۱۸ & ۳۰؟ \\
\hline
ضریب جینی & ۰.۳۸ & ۰.۳۷ & ۰.۳۹ & ۰.۴۱ & ۰.۴۲ \\
\hline
\end{tabular}
\end{table}

%═══════════════════════════════════════════════════════════════════════════════
\section{تحریم‌ها و نهادینه شدن فساد}
\label{sec:sanctions-corruption}
%═══════════════════════════════════════════════════════════════════════════════

\subsection{چارچوب مفهومی: مکانیزم‌های ارتباط تحریم و فساد}

\begin{figure}[htbp]
\centering
\begin{tikzpicture}[
    node distance=1.5cm,
    box/.style={rectangle, draw, fill=#1, minimum width=3cm, minimum height=1cm, 
        text centered, font=\small, rounded corners},
    arrow/.style={->, thick, >=stealth}
]

% تحریم
\node[box=red!30] (sanctions) at (0,0) {تحریم‌های اقتصادی};

% مکانیزم‌ها
\node[box=orange!30] (scarcity) at (-5,-2) {کمیابی مصنوعی};
\node[box=orange!30] (informal) at (0,-2) {اقتصاد غیررسمی};
\node[box=orange!30] (opacity) at (5,-2) {عدم شفافیت};

% واسطه‌ها
\node[box=yellow!30] (rent) at (-5,-4) {رانت کمیابی};
\node[box=yellow!30] (smuggling) at (0,-4) {قاچاق و دور زدن};
\node[box=yellow!30] (discretion) at (5,-4) {صلاحدید مقامات};

% نتایج
\node[box=purple!30] (corruption) at (0,-6) {\textbf{نهادینه شدن فساد}};

% فلش‌ها
\draw[arrow] (sanctions) -- (scarcity);
\draw[arrow] (sanctions) -- (informal);
\draw[arrow] (sanctions) -- (opacity);
\draw[arrow] (scarcity) -- (rent);
\draw[arrow] (informal) -- (smuggling);
\draw[arrow] (opacity) -- (discretion);
\draw[arrow] (rent) -- (corruption);
\draw[arrow] (smuggling) -- (corruption);
\draw[arrow] (discretion) -- (corruption);

\end{tikzpicture}
\caption{مکانیزم‌های ارتباط تحریم و فساد}
\label{fig:sanctions-corruption-mechanism}
\end{figure}

\subsection{کانال‌های اصلی فسادزایی تحریم‌ها}

\subsubsection{کانال اول: رانت ارزی}

\begin{tcolorbox}[colback=red!5!white,colframe=red!75!black,
    title=مکانیزم فساد ارزی]
\textbf{مرحله ۱:} ایجاد نظام چندنرخی ارز به بهانه «مدیریت تحریم»

\textbf{مرحله ۲:} تخصیص ارز ارزان (۴۲۰۰ تومان) به «کالاهای اساسی»

\textbf{مرحله ۳:} واردکننده با ارز ۴۲۰۰ تومانی کالا وارد می‌کند

\textbf{مرحله ۴:} کالا با قیمت ارز آزاد (۵۰,۰۰۰ تومان) فروخته می‌شود
\textbf{مرحله ۵:} سود رانتی = (۵۰,۰۰۰ - ۴,۲۰۰) × مقدار ارز = \textbf{۱۰۹۰٪ سود}

\textbf{نتیجه:} انتقال ثروت از خزانه عمومی به واردکنندگان رانتی
\end{tcolorbox}

\begin{table}[htbp]
\centering
\caption{برآورد رانت ارزی سالانه در ایران (میلیارد دلار)}
\label{tab:currency-rent}
\begin{tabular}{|c|r|r|r|r|r|}
\hline
\textbf{سال} & \textbf{ارز ۴۲۰۰ تخصیصی} & \textbf{نرخ بازار} & \textbf{نرخ رسمی} & \textbf{شکاف (٪)} & \textbf{رانت کل} \\
\hline
۱۳۹۷ & ۱۵.۰ & ۱۴,۰۰۰ & ۴,۲۰۰ & ۲۳۳٪ & ۱۴.۷ \\
\hline
۱۳۹۸ & ۱۲.۰ & ۱۳,۰۰۰ & ۴,۲۰۰ & ۲۱۰٪ & ۱۰.۶ \\
\hline
۱۳۹۹ & ۱۰.۰ & ۲۵,۰۰۰ & ۴,۲۰۰ & ۴۹۵٪ & ۲۰.۸ \\
\hline
۱۴۰۰ & ۸.۵ & ۲۸,۰۰۰ & ۴,۲۰۰ & ۵۶۷٪ & ۲۰.۲ \\
\hline
۱۴۰۱ & ۷.۰ & ۳۲,۰۰۰ & ۴,۲۰۰ & ۶۶۲٪ & ۱۹.۵ \\
\hline
\textbf{مجموع ۵ ساله} & \textbf{۵۲.۵} & --- & --- & --- & \textbf{۸۵.۸} \\
\hline
\end{tabular}
\end{table}

\begin{figure}[htbp]
\centering
\begin{tikzpicture}
\begin{axis}[
    xlabel={سال},
    ylabel={میلیارد دلار},
    xmin=2018, xmax=2023,
    ymin=0, ymax=25,
    xtick={2018,2019,2020,2021,2022,2023},
    legend pos=north east,
    ymajorgrids=true,
    grid style=dashed,
    width=12cm,
    height=7cm,
    ybar,
    bar width=0.8cm,
]

\addplot[fill=red!60] coordinates {
    (2018,14.7) (2019,10.6) (2020,20.8) (2021,20.2) (2022,19.5) (2023,12.0)
};
\addlegendentry{رانت ارزی تخمینی}

\end{axis}
\end{tikzpicture}
\caption{روند رانت ارزی ناشی از نظام چندنرخی (۱۳۹۷-۱۴۰۲)}
\label{fig:currency-rent}
\end{figure}

\subsubsection{کانال دوم: قاچاق و دور زدن تحریم}

تحریم‌ها شبکه‌های گسترده قاچاق و دور زدن ایجاد کرده‌اند:

\begin{enumerate}
    \item \textbf{قاچاق سوخت:} برآورد ۱۵-۲۰ میلیون لیتر روزانه
    \item \textbf{صادرات غیررسمی نفت:} انتقال کشتی به کشتی، خاموش کردن AIS
    \item \textbf{واردات کالا از مسیرهای غیررسمی:} امارات، ترکیه، عراق
    \item \textbf{تبادلات مالی غیررسمی:} صرافی‌ها، حواله، رمزارز
\end{enumerate}

\begin{tcolorbox}[colback=orange!5!white,colframe=orange!75!black,
    title=مطالعه موردی: قاچاق سوخت]
\textbf{ساختار:}
\begin{itemize}
    \item قیمت بنزین داخلی: ۳,۰۰۰ تومان (لیتر)
    \item قیمت بنزین در پاکستان: معادل ۳۰,۰۰۰ تومان
    \item قیمت گازوئیل داخلی: ۸۰۰ تومان
    \item قیمت گازوئیل در افغانستان: معادل ۱۵,۰۰۰ تومان
\end{itemize}

\textbf{سود قاچاق:} ۹۰۰٪ تا ۱,۸۰۰٪

\textbf{حجم تخمینی:} ۱۰-۲۰ میلیارد دلار سالانه

\textbf{بازیگران:} شبکه‌های محلی مرزی، برخی نهادهای رسمی، گروه‌های مسلح
\end{tcolorbox}

\subsubsection{کانال سوم: صلاحدید و فساد اداری}

\begin{figure}[htbp]
\centering
\begin{tikzpicture}[
    node distance=1.2cm,
    box/.style={rectangle, draw, fill=#1, minimum width=4cm, minimum height=0.8cm, 
        text centered, font=\small},
    arrow/.style={->, thick, >=stealth}
]

\node[box=gray!20] (normal) at (0,6) {شرایط عادی: قواعد شفاف و قابل پیش‌بینی};
\node[box=red!20] (sanctions) at (0,4.5) {شرایط تحریمی: «ضرورت» و «امنیت ملی»};
\node[box=orange!20] (discretion) at (0,3) {افزایش صلاحدید مقامات};
\node[box=yellow!20] (opaque) at (-4,1.5) {معاملات محرمانه};
\node[box=yellow!20] (exempt) at (0,1.5) {معافیت از نظارت};
\node[box=yellow!20] (monopoly) at (4,1.5) {انحصارات ویژه};
\node[box=purple!30] (corruption) at (0,0) {\textbf{فساد سیستماتیک}};

\draw[arrow] (normal) -- (sanctions) node[midway, right, font=\tiny] {تحریم};
\draw[arrow] (sanctions) -- (discretion);
\draw[arrow] (discretion) -- (opaque);
\draw[arrow] (discretion) -- (exempt);
\draw[arrow] (discretion) -- (monopoly);
\draw[arrow] (opaque) -- (corruption);
\draw[arrow] (exempt) -- (corruption);
\draw[arrow] (monopoly) -- (corruption);

\end{tikzpicture}
\caption{مکانیزم فسادزایی صلاحدید در شرایط تحریمی}
\label{fig:discretion-corruption}
\end{figure}

\subsection{شاخص‌های فساد در ایران}

\begin{table}[htbp]
\centering
\caption{رتبه ایران در شاخص ادراک فساد (CPI) سازمان شفافیت بین‌الملل}
\label{tab:cpi-ranking}
\begin{tabular}{|c|c|c|c|l|}
\hline
\textbf{سال} & \textbf{امتیاز (از ۱۰۰)} & \textbf{رتبه جهانی} & \textbf{تعداد کشورها} & \textbf{وضعیت تحریم} \\
\hline
۲۰۰۸ & ۲۳ & ۱۴۱ & ۱۸۰ & قبل از تشدید \\
\hline
۲۰۱۰ & ۲۵ & ۱۴۶ & ۱۷۸ & شروع تشدید \\
\hline
۲۰۱۲ & ۲۸ & ۱۳۳ & ۱۷۴ & اوج تحریم اول \\
\hline
۲۰۱۴ & ۲۷ & ۱۳۶ & ۱۷۵ & --- \\
\hline
۲۰۱۶ & ۲۹ & ۱۳۱ & ۱۷۶ & برجام \\
\hline
۲۰۱۸ & ۲۸ & ۱۳۸ & ۱۸۰ & خروج ترامپ \\
\hline
۲۰۲۰ & ۲۵ & ۱۴۹ & ۱۸۰ & فشار حداکثری \\
\hline
۲۰۲۲ & ۲۵ & ۱۴۷ & ۱۸۰ & تداوم فشار \\
\hline
۲۰۲۳ & ۲۴ & ۱۴۹ & ۱۸۰ & ---
\hline
\end{tabular}
\end{table}

\begin{figure}[htbp]
\centering
\begin{tikzpicture}
\begin{axis}[
    xlabel={سال},
    ylabel={امتیاز CPI (از ۱۰۰)},
    xmin=2006, xmax=2024,
    ymin=0, ymax=50,
    xtick={2006,2008,2010,2012,2014,2016,2018,2020,2022,2024},
    ytick={0,10,20,30,40,50},
    legend pos=north east,
    ymajorgrids=true,
    grid style=dashed,
    width=14cm,
    height=7cm,
]

% ایران
\addplot[thick, color=red, mark=*, mark size=2pt] coordinates {
    (2006,27) (2007,25) (2008,23) (2009,18) (2010,25)
    (2011,27) (2012,28) (2013,25) (2014,27) (2015,27)
    (2016,29) (2017,30) (2018,28) (2019,26) (2020,25)
    (2021,25) (2022,25) (2023,24)
};
\addlegendentry{ایران}

% ترکیه (برای مقایسه)
\addplot[thick, color=blue, mark=square*, mark size=2pt, dashed] coordinates {
    (2006,38) (2007,41) (2008,46) (2009,44) (2010,44)
    (2011,42) (2012,49) (2013,50) (2014,45) (2015,42)
    (2016,41) (2017,40) (2018,41) (2019,39) (2020,40)
    (2021,38) (2022,36) (2023,34)
};
\addlegendentry{ترکیه}

% امارات (برای مقایسه)
\addplot[thick, color=green!60!black, mark=triangle*, mark size=2pt, dotted] coordinates {
    (2006,62) (2007,57) (2008,59) (2009,63) (2010,63)
    (2011,68) (2012,68) (2013,69) (2014,70) (2015,70)
    (2016,66) (2017,71) (2018,70) (2019,71) (2020,71)
    (2021,69) (2022,67) (2023,68)
};
\addlegendentry{امارات}

% خط میانگین جهانی
\addplot[thick, color=gray, dashed] coordinates {
    (2006,43) (2023,43)
};
\addlegendentry{میانگین جهانی}

% مناطق تحریمی
\fill[red, opacity=0.1] (axis cs:2010,0) rectangle (axis cs:2016,50);
\fill[red, opacity=0.2] (axis cs:2018,0) rectangle (axis cs:2024,50);

\node[font=\tiny] at (axis cs:2013,45) {دوره تحریم اول};
\node[font=\tiny] at (axis cs:2021,45) {دوره فشار حداکثری};

\end{axis}
\end{tikzpicture}
\caption{مقایسه شاخص ادراک فساد ایران با کشورهای منطقه (۲۰۰۶-۲۰۲۳)}
\label{fig:cpi-comparison}
\end{figure}

\subsection{انواع فساد تشدیدشده در دوره تحریم}

\begin{table}[htbp]
\centering
\caption{گونه‌شناسی فساد در دوره تحریم}
\label{tab:corruption-typology}
\small
\begin{tabular}{|p{2.5cm}|p{4cm}|p{4cm}|p{3cm}|}
\hline
\textbf{نوع فساد} & \textbf{شرح} & \textbf{مثال} & \textbf{مقیاس تخمینی} \\
\hline
\multicolumn{4}{|c|}{\cellcolor{red!15}\textbf{فساد بزرگ (Grand Corruption)}} \\
\hline
رانت ارزی & سوءاستفاده از ارز ترجیحی & پرونده ۳ میلیارد دلاری & ۸۰+ میلیارد \$ \\
\hline
فساد نفتی & فروش غیررسمی نفت، کم‌اظهاری & صادرات بدون ثبت & ۲۰+ میلیارد \$ \\
\hline
فساد بانکی & تسهیلات کلان بدون بازگشت & پرونده‌های بانک سرمایه، ملت & ۵۰+ میلیارد \$ \\
\hline
قراردادهای نفتی & کمیسیون و رشوه در قراردادها & پرونده بابک زنجانی & ۱۰+ میلیارد \$ \\
\hline
\multicolumn{4}{|c|}{\cellcolor{orange!15}\textbf{فساد سازمان‌یافته}} \\
\hline
قاچاق کالا & واردات غیرقانونی از مبادی غیررسمی & کالاهای مصرفی، لوازم خانگی & ۱۵-۲۵ میلیارد \$ \\
\hline
قاچاق سوخت & صادرات بنزین و گازوئیل & مرزهای شرقی و غربی & ۵-۱۰ میلیارد \$ \\
\hline
پولشویی & تبدیل پول نامشروع به مشروع & صرافی‌ها، رمزارز، املاک & نامشخص \\
\hline
\multicolumn{4}{|c|}{\cellcolor{yellow!15}\textbf{فساد خُرد (Petty Corruption)}} \\
\hline
رشوه اداری & تسریع در کارها، اخذ مجوز & گمرک، شهرداری، دادگستری & ۲-۵ میلیارد \$ \\
\hline
باج‌گیری & اخاذی از کسب‌وکارها & بازرسی‌ها، مجوزها & نامشخص \\
\hline
پارتی‌بازی & استخدام و انتصاب غیرشایسته & نهادهای دولتی و شبه‌دولتی & غیرمالی \\
\hline
\end{tabular}
\end{table}

%═══════════════════════════════════════════════════════════════════════════════
\section{ظهور بازیگران جدید اقتصادی}
\label{sec:new-actors}
%═══════════════════════════════════════════════════════════════════════════════

تحریم‌ها منجر به بازآرایی کامل ساختار قدرت اقتصادی در ایران شده‌اند. 
بازیگران سنتی (بخش خصوصی رسمی، شرکت‌های دولتی کلاسیک) تضعیف شده و 
بازیگران جدید با ویژگی‌های متفاوت ظهور کرده‌اند.

\subsection{طبقه‌بندی بازیگران اقتصادی}

\begin{figure}[htbp]
\centering
\begin{tikzpicture}[
    actor/.style={rectangle, draw, fill=#1, minimum width=3.5cm, 
        minimum height=1.2cm, text centered, font=\small, rounded corners},
    arrow/.style={->, thick, >=stealth}
]

% قبل از تحریم
\node[font=\bfseries] at (-6,5) {قبل از تشدید تحریم (۲۰۱۰)};

\node[actor=blue!30] (gov1) at (-8,3.5) {شرکت‌های دولتی\\(۴۵٪)};
\node[actor=green!30] (private1) at (-6,3.5) {بخش خصوصی واقعی\\(۳۵٪)};
\node[actor=orange!30] (semi1) at (-4,3.5) {شبه‌دولتی‌ها\\(۱۵٪)};
\node[actor=red!30] (informal1) at (-6,2) {بخش غیررسمی\\(۵٪)};

% بعد از تحریم
\node[font=\bfseries] at (4,5) {بعد از فشار حداکثری (۲۰۲۳)};

\node[actor=blue!20] (gov2) at (2,3.5) {شرکت‌های دولتی\\(۲۵٪)};
\node[actor=green!15] (private2) at (4,3.5) {بخش خصوصی واقعی\\(۱۵٪)};
\node[actor=orange!40] (semi2) at (6,3.5) {شبه‌دولتی‌ها\\(۳۵٪)};
\node[actor=red!40] (informal2) at (4,2) {بخش غیررسمی\\(۲۵٪)};

% فلش تحول
\draw[->, ultra thick, gray] (-2,3) -- (0,3) node[midway, above, font=\small] {تحریم};

\end{tikzpicture}
\caption{تحول ساختار بازیگران اقتصادی قبل و بعد از تحریم}
\label{fig:actors-transformation}
\end{figure}

\subsection{بازیگران جدید: مشخصات و عملکرد}

\subsubsection{گروه اول: نهادهای شبه‌دولتی نظامی-امنیتی}

\begin{tcolorbox}[colback=red!5!white,colframe=red!75!black,
    title=مشخصات نهادهای شبه‌دولتی]

\textbf{ویژگی‌های کلیدی:}
\begin{itemize}
    \item وابستگی به نهادهای نظامی و امنیتی (سپاه، بسیج، وزارت دفاع)
    \item معافیت از مالیات و عوارض گمرکی
    \item دسترسی ترجیحی به ارز، اعتبارات بانکی و قراردادهای دولتی
    \item عدم شفافیت مالی و پاسخگویی
    \item توانایی فعالیت در شرایط تحریمی به دلیل ساختار موازی
\end{itemize}

\textbf{نمونه‌ها:}
\begin{itemize}
    \item قرارگاه خاتم‌الانبیا (سپاه): زیرساخت، نفت و گاز، پتروشیمی
    \item بنیاد تعاون سپاه: بانکداری، بیمه، واردات
    \item ستاد اجرایی فرمان امام: سرمایه‌گذاری متنوع
    \item بنیاد مستضعفان: صنعت، کشاورزی، گردشگری
\end{itemize}

\textbf{سهم تخمینی از GDP:} ۲۵-۴۰٪
\end{tcolorbox}

\begin{table}[htbp]
\centering
\caption{برآورد فعالیت‌های اقتصادی نهادهای شبه‌دولتی}
\label{tab:quasi-governmental}
\small
\begin{tabular}{|l|p{3cm}|r|p{4cm}|}
\hline
\textbf{نهاد} & \textbf{حوزه‌های فعالیت} & \textbf{برآورد دارایی (B\$)} & \textbf{منابع مزیت} \\
\hline
قرارگاه خاتم‌الانبیا & نفت، گاز، زیرساخت، معدن & ۵۰+ & قراردادهای انحصاری، معافیت \\
\hline
بنیاد مستضعفان & صنعت، کشاورزی، گردشگری & ۲۰-۳۰ & معافیت مالیاتی، زمین \\
\hline
ستاد اجرایی فرمان امام & سرمایه‌گذاری متنوع & ۱۵-۲۰ & عدم نظارت، تحریم‌ناپذیری \\
\hline
آستان قدس رضوی & کشاورزی، صنعت، خدمات & ۲۰+ & معافیت کامل، موقوفات \\
\hline
بنیاد تعاون سپاه & بانکداری، واردات & ۱۰-۱۵ & دسترسی ارزی \\
\hline
\textbf{مجموع} & --- & \textbf{۱۲۰-۱۵۰} & --- \\
\hline
\end{tabular}
\end{table}

\subsubsection{گروه دوم: شبکه‌های دور زدن تحریم}

\begin{figure}[htbp]
\centering
\begin{tikzpicture}[
    node distance=1.5cm,
    box/.style={rectangle, draw, fill=#1, minimum width=2.8cm, 
        minimum height=0.9cm, text centered, font=\footnotesize, rounded corners},
    arrow/.style={->, thick, >=stealth}
]

% مرکز
\node[box=red!40] (iran) at (0,0) {اقتصاد ایران};

% لایه اول - شرکت‌های پوسته
\node[box=orange!30] (uae) at (4,2) {شرکت پوسته\\امارات};
\node[box=orange!30] (turkey) at (4,0) {شرکت پوسته\\ترکیه};
\node[box=orange!30] (iraq) at (4,-2) {شرکت پوسته\\عراق};
\node[box=orange!30] (china) at (-4,1) {شرکت پوسته\\چین};
\node[box=orange!30] (russia) at (-4,-1) {شرکت پوسته\\روسیه};

% لایه دوم - واسطه‌ها
\node[box=yellow!30] (malaysia) at (7,1) {مالزی\\سنگاپور};
\node[box=yellow!30] (oman) at (7,-1) {عمان\\قطر};
\node[box=yellow!30] (hk) at (-7,0) {هنگ‌کنگ\\ماکائو};

% بازار جهانی
\node[box=green!30] (global) at (0,-4) {بازار جهانی};

% فلش‌ها
\draw[arrow, red] (iran) -- (uae);
\draw[arrow, red] (iran) -- (turkey);
\draw[arrow, red] (iran) -- (iraq);
\draw[arrow, red] (iran) -- (china);
\draw[arrow, red] (iran) -- (russia);

\draw[arrow, orange] (uae) -- (malaysia);
\draw[arrow, orange] (turkey) -- (malaysia);
\draw[arrow, orange] (iraq) -- (oman);
\draw[arrow, orange] (china) -- (hk);
\draw[arrow, orange] (russia) -- (hk);

\draw[arrow, green!60!black] (malaysia) -- (global);
\draw[arrow, green!60!black] (oman) -- (global);
\draw[arrow, green!60!black] (hk) -- (global);

% توضیحات
\node[font=\tiny, gray] at (2,1.5) {نفت، پتروشیمی};
\node[font=\tiny, gray] at (2,-1.5) {طلا، کالا};

\end{tikzpicture}
\caption{ساختار شبکه‌های دور زدن تحریم}
\label{fig:sanctions-evasion-network}
\end{figure}

\begin{tcolorbox}[colback=orange!5!white,colframe=orange!75!black,
    title=مطالعه موردی: پرونده بابک زنجانی]

\textbf{زمینه:} بابک زنجانی یکی از بزرگ‌ترین واسطه‌های فروش نفت ایران 
در دوره تحریم بود.

\textbf{مکانیزم:}
\begin{enumerate}
    \item دریافت نفت از وزارت نفت با قیمت ترجیحی
    \item فروش از طریق شبکه شرکت‌های پوسته در امارات، ترکیه، مالزی
    \item انتقال وجوه از طریق صرافی‌ها و بانک‌های ثالث
    \item عدم بازگشت بخش عمده ارز به خزانه
\end{enumerate}

\textbf{مقیاس:} بالغ بر ۲.۷ میلیارد دلار بدهی اثبات‌شده

\textbf{نتیجه:} حکم اعدام (اجرا نشده)

\textbf{درس:} نظام دور زدن تحریم ذاتاً فسادآور است زیرا:
\begin{itemize}
    \item عدم شفافیت ساختاری
    \item وابستگی به روابط شخصی و رشوه
    \item نبود نظارت و حسابرسی
    \item منافع مشترک فسادکننده و «دور زننده»
\end{itemize}
\end{tcolorbox}

\subsubsection{گروه سوم: بخش غیررسمی و قاچاق}

\begin{table}[htbp]
\centering
\caption{برآورد حجم اقتصاد غیررسمی و قاچاق}
\label{tab:informal-economy}
\begin{tabular}{|l|r|r|p{4cm}|}
\hline
\textbf{نوع فعالیت} & \textbf{۲۰۱۰ (B\$)} & \textbf{۲۰۲۳ (B\$)} & \textbf{تغییر} \\
\hline
\multicolumn{4}{|c|}{\cellcolor{gray!20}\textbf{قاچاق کالا}} \\
\hline
واردات غیرقانونی & ۸ & ۲۰ & ۱۵۰٪+ \\
\hline
قاچاق سوخت به خارج & ۲ & ۸ & ۳۰۰٪+ \\
\hline
قاچاق مواد مخدر (ترانزیت) & ۳ & ۵ & ۶۷٪+ \\
\hline
\multicolumn{4}{|c|}{\cellcolor{gray!20}\textbf{اقتصاد غیررسمی داخلی}} \\
\hline
کسب‌وکارهای ثبت‌نشده & ۲۵ & ۵۵ & ۱۲۰٪+ \\
\hline
اشتغال غیررسمی & ۲۰ & ۴۵ & ۱۲۵٪+ \\
\hline
فرار مالیاتی & ۱۵ & ۳۵ & ۱۳۳٪+ \\
\hline
\textbf{مجموع اقتصاد غیررسمی} & \textbf{۷۳} & \textbf{۱۶۸} & \textbf{۱۳۰٪+} \\
\hline
\textbf{نسبت به GDP} & \textbf{۱۷٪} & \textbf{۴۲٪} & --- \\
\hline
\end{tabular}
\end{table}

\subsubsection{گروه چهارم: الیگارشی جدید}

تحریم‌ها منجر به ظهور طبقه جدیدی از ثروتمندان شده که ویژگی‌های متفاوتی 
با نسل قبلی سرمایه‌داران دارند:

\begin{table}[htbp]
\centering
\caption{مقایسه نسل‌های سرمایه‌داران ایرانی}
\label{tab:capitalist-generations}
\begin{tabular}{|p{3cm}|p{5cm}|p{5cm}|}
\hline
\textbf{معیار} & \textbf{سرمایه‌داران سنتی (قبل ۲۰۱۰)} & \textbf{الیگارشی جدید (پس از ۲۰۱۰)} \\
\hline
منشأ ثروت & تولید، تجارت، ارث & رانت، دور زدن تحریم، قراردادهای دولتی \\
\hline
ارتباط با قدرت & غیرمستقیم، لابی‌گری & مستقیم، شراکت \\
\hline
شفافیت & نسبتاً بالا & بسیار پایین \\
\hline
محل دارایی‌ها & عمدتاً داخلی & عمدتاً خارجی \\
\hline
نوع فعالیت & تولیدی، خدماتی & واسطه‌گری، واردات، مالی \\
\hline
مشروعیت اجتماعی & نسبتاً بالا & پایین \\
\hline
وابستگی به تحریم & آسیب‌دیده & منتفع \\
\hline
\end{tabular}
\end{table}

%═══════════════════════════════════════════════════════════════════════════════
\section{برندگان و بازندگان تحریم}
\label{sec:winners-losers}
%═══════════════════════════════════════════════════════════════════════════════

\subsection{ماتریس برد-باخت تحریم‌ها}

\begin{figure}[htbp]
\centering
\begin{tikzpicture}
    % محورها
    \draw[thick, ->] (-6,0) -- (6,0) node[right] {قدرت سیاسی};
    \draw[thick, ->] (0,-4) -- (0,5) node[above] {تغییر رفاه اقتصادی};
    
    % خطوط مرجع
    \draw[dashed, gray] (-6,0) -- (6,0);
    \draw[dashed, gray] (0,-4) -- (0,5);
    
    % بازیگران
    % برندگان
    \node[circle, fill=green!60, minimum size=1.2cm, font=\tiny] at (4,4) {نهادهای\\شبه‌دولتی};
    \node[circle, fill=green!40, minimum size=1cm, font=\tiny] at (3,3) {شبکه‌های\\قاچاق};
    \node[circle, fill=green!40, minimum size=0.9cm, font=\tiny] at (2,2.5) {واسطه‌های\\تحریم};
    \node[circle, fill=green!30, minimum size=0.8cm, font=\tiny] at (-1,1.5) {رانت‌بگیران\\ارزی};
    
    % بازندگان
    \node[circle, fill=red!60, minimum size=1.5cm, font=\tiny] at (-2,-3) {طبقه\\متوسط};
    \node[circle, fill=red!50, minimum size=1.2cm, font=\tiny] at (0,-2.5) {بخش\\خصوصی واقعی};
    \node[circle, fill=red!40, minimum size=1cm, font=\tiny] at (-3,-2) {کارگران\\و حقوق‌بگیران};
    \node[circle, fill=red!30, minimum size=0.8cm, font=\tiny] at (2,-1.5) {صادرکنندگان\\غیرنفتی};
    
    % راهنما
    \node[font=\small] at (4,-3.5) {● برندگان (سبز)};
    \node[font=\small] at (4,-4) {● بازندگان (قرمز)};
    
\end{tikzpicture}
\caption{ماتریس برندگان و بازندگان تحریم در ایران}
\label{fig:winners-losers-matrix}
\end{figure}

\subsection{تحلیل گروه‌های برنده}

\begin{table}[htbp]
\centering
\caption{تحلیل تفصیلی گروه‌های برنده از تحریم}
\label{tab:winners-analysis}
\small
\begin{tabular}{|p{2.5cm}|p{3.5cm}|p{4cm}|p{3.5cm}|}
\hline
\textbf{گروه} & \textbf{منبع انتفاع} & \textbf{مکانیزم} & \textbf{برآورد منفعت سالانه} \\
\hline
\multicolumn{4}{|c|}{\cellcolor{green!15}\textbf{برندگان درجه اول}} \\
\hline
نهادهای شبه‌دولتی نظامی & انحصار قراردادها، معافیت‌ها & جایگزینی شرکت‌های خارجی، واردات انحصاری & ۲۰-۳۰ میلیارد \$ \\
\hline
شبکه‌های دور زدن تحریم & کمیسیون و کارمزد & واسطه‌گری در فروش نفت و واردات & ۵-۱۰ میلیارد \$ \\
\hline
وارد‌کنندگان رانتی & تفاوت نرخ ارز & خرید ارز ۴۲۰۰، فروش به قیمت آزاد & ۱۵-۲۵ میلیارد \$ \\
\hline
\multicolumn{4}{|c|}{\cellcolor{green!10}\textbf{برندگان درجه دوم}} \\
\hline
قاچاقچیان سازمان‌یافته & کمیابی مصنوعی & واردات غیرقانونی با سود بالا & ۸-۱۲ میلیارد \$ \\
\hline
صرافان و دلالان ارز & نوسان و شکاف ارزی & آربیتراژ، حواله، پولشویی & ۳-۵ میلیارد \$ \\
\hline
تولیدکنندگان انحصاری & حذف رقابت خارجی & فروش محصولات بی‌کیفیت با قیمت بالا & ۵-۸ میلیارد \$ \\
\hline
\multicolumn{4}{|c|}{\cellcolor{yellow!15}\textbf{برندگان موقعیتی}} \\
\hline
برخی کشاورزان & خودکفایی اجباری & افزایش قیمت محصولات داخلی & ۲-۳ میلیارد \$ \\
\hline
فعالان رمزارز & دور زدن محدودیت‌های ارزی & استخراج و معامله بیت‌کوین & ۱-۲ میلیارد \$ \\
\hline
\end{tabular}
\end{table}

\subsection{تحلیل گروه‌های بازنده}

\begin{table}[htbp]
\centering
\caption{تحلیل تفصیلی گروه‌های بازنده از تحریم}
\label{tab:losers-analysis}
\small
\begin{tabular}{|p{2.5cm}|p{3.5cm}|p{4cm}|p{3.5cm}|}
\hline
\textbf{گروه} & \textbf{منبع آسیب} & \textbf{مکانیزم} & \textbf{برآورد زیان} \\
\hline
\multicolumn{4}{|c|}{\cellcolor{red!15}\textbf{بازندگان درجه اول}} \\
\hline
طبقه متوسط شهری & کاهش قدرت خرید & تورم، کاهش ارزش پس‌اندازها & ۶۰-۷۰٪ کاهش رفاه \\
\hline
کارگران و مزدبگیران & عقب‌ماندگی دستمزد از تورم & رشد دستمزد کمتر از تورم & ۵۰-۶۰٪ کاهش رفاه \\
\hline
بخش خصوصی مولد & محدودیت واردات، تأمین مالی & عدم دسترسی به تکنولوژی، قطعات & ۴۰٪ کاهش تولید \\
\hline
\multicolumn{4}{|c|}{\cellcolor{red!10}\textbf{بازندگان درجه دوم}} \\
\hline
صادرکنندگان غیرنفتی & مشکلات انتقال پول، بیمه & عدم امکان گشایش LC، محدودیت بانکی & ۳۰٪ کاهش صادرات \\
\hline
بیماران خاص & کمبود دارو و تجهیزات & محدودیت واردات دارو & جانی (غیرقابل برآورد) \\
\hline
دانشجویان و محققان & انزوای علمی & محدودیت دسترسی به نشریات، همکاری & کاهش کیفیت آموزش \\
\hline
\multicolumn{4}{|c|}{\cellcolor{orange!15}\textbf{بازندگان ساختاری}} \\
\hline
نسل جوان & بیکاری، مهاجرت & محدودیت فرصت‌های شغلی & فرار مغزها \\
\hline
محیط زیست & کاهش استانداردها & واردات خودروهای بی‌کیفیت، سوخت مازوت & آلودگی مزمن \\
\hline
نهادهای مدنی & فشار امنیتی بیشتر & بهانه تحریم برای سرکوب & تضعیف جامعه مدنی \\
\hline
\end{tabular}
\end{table}

\subsection{اثر توزیعی تحریم‌ها}

\begin{figure}[htbp]
\centering
\begin{tikzpicture}
\begin{axis}[
    xlabel={دهک درآمدی},
    ylabel={تغییر درآمد واقعی (٪)},
    xmin=0.5, xmax=10.5,
    ymin=-60, ymax=40,
    xtick={1,2,3,4,5,6,7,8,9,10},
    xticklabels={دهک ۱,دهک ۲,دهک ۳,دهک ۴,دهک ۵,دهک ۶,دهک ۷,دهک ۸,دهک ۹,دهک ۱۰},
    x tick label style={rotate=45, anchor=east, font=\tiny},
    ytick={-60,-40,-20,0,20,40},
    legend pos=north east,
    ymajorgrids=true,
    grid style=dashed,
    width=14cm,
    height=8cm,
    ybar,
    bar width=0.6cm,
]

% تغییر درآمد واقعی ۲۰۱۸-۲۰۲۳
\addplot[fill=red!60] coordinates {
    (1,-45) (2,-42) (3,-40) (4,-38) (5,-35)
    (6,-30) (7,-25) (8,-15) (9,5) (10,30)
};
\addlegendentry{تغییر درآمد واقعی (۲۰۱۸-۲۰۲۳)}

% خط صفر
\draw[thick, dashed, gray] (axis cs:0.5,0) -- (axis cs:10.5,0);

\end{axis}
\end{tikzpicture}
\caption{اثر توزیعی تحریم‌ها بر دهک‌های درآمدی (برآورد)}
\label{fig:distributional-effect}
\end{figure}

\begin{tcolorbox}[colback=red!5!white,colframe=red!75!black,
    title=یافته کلیدی: تحریم به مثابه بازتوزیع معکوس]
تحریم‌ها عملاً منجر به انتقال ثروت از:
\begin{itemize}
    \item دهک‌های پایین و متوسط ← دهک بالا
    \item بخش خصوصی مولد ← بخش شبه‌دولتی و غیررسمی
    \item اقتصاد شفاف ← اقتصاد سایه
    \item شهروندان عادی ← رانت‌بگیران و فسادکنندگان
\end{itemize}
شده‌اند. این پدیده را می‌توان «بازتوزیع معکوس» نامید: ثروت از پایین به بالا منتقل می‌شود.
\end{tcolorbox}

%═══════════════════════════════════════════════════════════════════════════════
\section{پیامدهای نهادی و بلندمدت}
\label{sec:institutional-consequences}
%═══════════════════════════════════════════════════════════════════════════════

\subsection{تغییرات ساختار نهادی}

\begin{figure}[htbp]
\centering
\begin{tikzpicture}[
    node distance=1.5cm,
    box/.style={rectangle, draw, fill=#1, minimum width=4.5cm, 
        minimum height=1cm, text centered, font=\small, rounded corners},
    arrow/.style={->, thick, >=stealth}
]

% ستون قبل
\node[font=\bfseries] at (-4,6) {قبل از تحریم شدید};
\node[box=blue!20] (rule1) at (-4,4.5) {حاکمیت نسبی قانون};
\node[box=green!20] (market1) at (-4,3) {بازار نسبتاً رقابتی};
\node[box=yellow!20] (trans1) at (-4,1.5) {شفافیت نسبی};
\node[box=orange!20] (prop1) at (-4,0) {حقوق مالکیت نسبی};

% ستون بعد
\node[font=\bfseries] at (4,6) {بعد از تحریم شدید};
\node[box=red!30] (rule2) at (4,4.5) {حاکمیت صلاحدید و استثنا};
\node[box=red!30] (market2) at (4,3) {انحصارات و رانت};
\node[box=red!30] (trans2) at (4,1.5) {عدم شفافیت سیستماتیک};
\node[box=red!30] (prop2) at (4,0) {مصادره و سلب مالکیت};

% فلش‌ها
\draw[arrow, red, thick] (rule1) -- (rule2) node[midway, above, font=\tiny] {تحریم};
\draw[arrow, red, thick] (market1) -- (market2);
\draw[arrow, red, thick] (trans1) -- (trans2);
\draw[arrow, red, thick] (prop1) -- (prop2);

\end{tikzpicture}
\caption{تحول نهادی ناشی از تحریم‌ها}
\label{fig:institutional-transformation}
\end{figure}

\subsection{شاخص‌های نهادی}

\begin{table}[htbp]
\centering
\caption{تغییرات شاخص‌های نهادی ایران در دوره تحریم}
\label{tab:institutional-indicators}
\begin{tabular}{|l|c|c|c|c|l|}
\hline
\textbf{شاخص} & \textbf{۲۰۱۰} & \textbf{۲۰۱۵} & \textbf{۲۰۲۰} & \textbf{۲۰۲۳} & \textbf{منبع} \\
\hline
\multicolumn{6}{|c|}{\cellcolor{gray!20}\textbf{شاخص‌های حکمرانی بانک جهانی}} \\
\hline
کیفیت قوانین & $-$۰.۸۲ & $-$۱.۰۳ & $-$۱.۲۵ & $-$۱.۳۵ & WGI \\
\hline
حاکمیت قانون & $-$۰.۷۵ & $-$۰.۸۵ & $-$۰.۹۵ & $-$۱.۰۵ & WGI \\
\hline
کنترل فساد & $-$۰.۶۵ & $-$۰.۷۵ & $-$۰.۹۰ & $-$۰.۹۸ & WGI \\
\hline
پاسخگویی & $-$۱.۵۰ & $-$۱.۵۵ & $-$۱.۶۵ & $-$۱.۷۵ & WGI \\
\hline
\multicolumn{6}{|c|}{\cellcolor{gray!20}\textbf{شاخص‌های آزادی اقتصادی}} \\
\hline
آزادی اقتصادی & ۴۳.۴ & ۴۱.۸ & ۴۷.۲ & ۴۲.۱ & Heritage \\
\hline
سهولت کسب‌وکار & ۱۲۹ & ۱۱۸ & ۱۲۷ & --- & WB (توقف شد) \\
\hline
\multicolumn{6}{|c|}{\cellcolor{gray!20}\textbf{شاخص‌های شفافیت}} \\
\hline
آزادی مطبوعات & ۱۷۵ & ۱۷۳ & ۱۷۳ & ۱۷۶ & RSF \\
\hline
شاخص CPI & ۲۵ & ۲۷ & ۲۵ & ۲۴ & TI \\
\hline
\end{tabular}
\end{table}

\subsection{اثر چرخه‌ای تحریم و فساد}

\begin{figure}[htbp]
\centering
\begin{tikzpicture}[
    node distance=2.5cm,
    box/.style={rectangle, draw, fill=#1, minimum width=3.5cm, 
        minimum height=1.2cm, text centered, font=\small, rounded corners},
    arrow/.style={->, very thick, >=stealth}
]

\node[box=red!30] (sanctions) at (0,4) {تحریم‌های خارجی};
\node[box=orange!30] (informal) at (5,2) {گسترش اقتصاد غیررسمی};
\node[box=yellow!30] (corruption) at (5,-2) {افزایش فساد};
\node[box=purple!30] (institution) at (0,-4) {تضعیف نهادها};
\node[box=blue!30] (resistance) at (-5,-2) {مقاومت در برابر اصلاحات};
\node[box=green!30] (justify) at (-5,2) {توجیه تداوم تحریم};

% چرخه
\draw[arrow, red] (sanctions) -- (informal);
\draw[arrow, orange] (informal) -- (corruption);
\draw[arrow, yellow] (corruption) -- (institution);
\draw[arrow, purple] (institution) -- (resistance);
\draw[arrow, blue] (resistance) -- (justify);
\draw[arrow, green!60!black] (justify) -- (sanctions);

% توضیح مرکزی
\node[font=\bfseries, text=red] at (0,0) {چرخه معیوب};
\node[font=\small] at (0,-0.5) {تحریم-فساد-تداوم};

\end{tikzpicture}
\caption{چرخه معیوب تحریم و فساد}
\label{fig:vicious-cycle}
\end{figure}

\begin{tcolorbox}[colback=blue!5!white,colframe=blue!75!black,
    title=نظریه: تحریم به مثابه «قفل‌شدگی نهادی»]
تحریم‌های بلندمدت منجر به پدیده‌ای می‌شوند که می‌توان آن را «قفل‌شدگی 
نهادی» \LTRfootnote{Institutional Lock-in} نامید:

\begin{enumerate}
    \item \textbf{ذی‌نفعان جدید:} گروه‌هایی که از تحریم سود می‌برند، 
    قدرت سیاسی کسب می‌کنند.
    
    \item \textbf{مقاومت در برابر تغییر:} این گروه‌ها در برابر هرگونه 
    اصلاح (اعم از رفع تحریم یا اصلاحات داخلی) مقاومت می‌کنند.
    
    \item \textbf{تضعیف ظرفیت اصلاح:} تضعیف نهادها و جامعه مدنی، ظرفیت 
    اصلاحات را کاهش می‌دهد.
    
    \item \textbf{تعادل سطح پایین:} اقتصاد در یک تعادل سطح پایین 
    با فساد بالا، رشد پایین و نابرابری بالا گیر می‌کند.
\end{enumerate}
\end{tcolorbox}

%═══════════════════════════════════════════════════════════════════════════════
\section{مطالعات موردی}
\label{sec:case-studies}
%═══════════════════════════════════════════════════════════════════════════════

\subsection{مطالعه موردی ۱: ارز ۴۲۰۰ تومانی}

\begin{table}[htbp]
\centering
\caption{خلاصه پرونده ارز ۴۲۰۰ تومانی (۱۳۹۷-۱۴۰۱)}
\label{tab:case-4200}
\begin{tabular}{|l|p{10cm}|}
\hline
\textbf{هدف اعلام‌شده} & کنترل قیمت کالاهای اساسی، جلوگیری از تورم \\
\hline
\textbf{مکانیزم} & تخصیص ارز با نرخ ثابت ۴۲۰۰ تومان برای واردات کالاهای اساسی \\
\hline
\textbf{حجم ارز تخصیصی} & حدود ۵۰ میلیارد دلار در ۴ سال \\
\hline
\textbf{مشکلات اجرایی} & 
\begin{itemize}[noitemsep,topsep=0pt]
    \item واردات کالاهای غیراساسی با ارز ترجیحی
    \item کم‌اظهاری و اظهار خلاف واقع
    \item عدم عرضه کالا به قیمت مصوب
    \item ایجاد شرکت‌های کاغذی
\end{itemize} \\
\hline
\textbf{برآورد رانت} & ۶۰-۸۰ میلیارد دلار \\
\hline
\textbf{ذی‌نفعان} & واردکنندگان بزرگ، برخی مقامات، شبکه‌های توزیع \\
\hline
\textbf{نتیجه} & حذف تدریجی از ۱۴۰۱، بدون محاکمه جدی متخلفان \\
\hline
\end{tabular}
\end{table}

\subsection{مطالعه موردی ۲: فساد بانکی}

\begin{table}[htbp]
\centering
\caption{پرونده‌های بزرگ فساد بانکی در دوره تحریم}
\label{tab:banking-corruption}
\small
\begin{tabular}{|l|c|p{4cm}|p{4cm}|}
\hline
\textbf{پرونده} & \textbf{مبلغ (هزار میلیارد ریال)} & \textbf{شرح} & \textbf{نتیجه قضایی} \\
\hline
بانک سرمایه & ۴,۶۰۰ & تسهیلات بدون وثیقه کافی & چند حکم اعدام و حبس \\
\hline
ثامن‌الحجج & ۳,۰۰۰+ & موسسه غیرمجاز & ورشکستگی، برخی احکام \\
\hline
کاسپین & ۲,۰۰۰+ & موسسه غیرمجاز & ادغام، احکام متعدد \\
\hline
پرونده کلاهبرداری ۳ میلیارد دلاری & ۱,۲۰۰ & سوءاستفاده از ارز دولتی & اعدام محکومان اصلی \\
\hline
پرونده بابک زنجانی & ۲,۷۰۰ & عدم بازگشت ارز نفتی & حکم اعدام (اجرا نشده) \\
\hline
\textbf{مجموع برآوردی} & \textbf{۱۵,۰۰۰+} & --- & --- \\
\hline
\end{tabular}
\end{table}

\subsection{مطالعه موردی ۳: قاچاق سوخت}

\begin{figure}[htbp]
\centering
\begin{tikzpicture}[
    node distance=2cm,
    box/.style={rectangle, draw, fill=#1, minimum width=2.5cm, 
        minimum height=0.8cm, text centered, font=\footnotesize, rounded corners},
    arrow/.style={->, thick, >=stealth}
]

% زنجیره قاچاق سوخت
\node[box=blue!20] (refinery) at (0,0) {پالایشگاه\\(قیمت یارانه‌ای)};
\node[box=green!20] (station) at (3,0) {جایگاه سوخت\\(۳۰۰۰ ت/لیتر)};
\node[box=orange!20] (tanker) at (6,0) {تانکر قاچاق};
\node[box=red!20] (border) at (9,0) {مرز\\(پاکستان/عراق)};
\node[box=purple!20] (foreign) at (12,0) {فروش خارجی\\(۳۰۰۰۰ ت/لیتر)};

\draw[arrow] (refinery) -- (station);
\draw[arrow] (station) -- (tanker) node[midway, above, font=\tiny] {انحراف};
\draw[arrow] (tanker) -- (border);
\draw[arrow] (border) -- (foreign);

% سود
\node[font=\small, red] at (6,-1.5) {سود: ۹۰۰٪ تا ۱۸۰۰٪};
\node[font=\tiny, gray] at (6,-2) {حجم: ۱۰-۲۰ میلیون لیتر/روز};

\end{tikzpicture}
\caption{زنجیره قاچاق سوخت}
\label{fig:fuel-smuggling-chain}
\end{figure}

%═══════════════════════════════════════════════════════════════════════════════
\section{تحلیل تطبیقی: ایران و سایر کشورهای تحریم‌شده}
\label{sec:comparative-analysis}
%═══════════════════════════════════════════════════════════════════════════════

\begin{table}[htbp]
\centering
\caption{مقایسه اثر تحریم بر فساد در کشورهای مختلف}
\label{tab:comparative-sanctions}
\small
\begin{tabular}{|l|c|c|c|c|p{3cm}|}
\hline
\textbf{کشور} & \textbf{مدت تحریم} & \textbf{CPI قبل} & \textbf{CPI بعد} & \textbf{تغییر} & \textbf{توضیح} \\
\hline
ایران & ۴۵+ سال & ۲۹ & ۲۴ & $-$۵ & تشدید فساد ساختاری \\
\hline
روسیه & ۱۰ سال & ۲۸ & ۲۶ & $-$۲ & افزایش اقتصاد سایه \\
\hline
کره شمالی & ۷۰+ سال & --- & --- & --- & فساد کامل، داده ندارد \\
\hline
کوبا & ۶۰+ سال & ۴۷ & ۴۲ & $-$۵ & فساد در اقتصاد دوگانه \\
\hline
ونزوئلا & ۱۵ سال & ۲۳ & ۱۳ & $-$۱۰ & فروپاشی نهادی \\
\hline
میانمار & ۲۵ سال & ۲۱ & ۲۳ & $+$۲ & بهبود نسبی پس از اصلاحات \\
\hline
سودان & ۲۰ سال & ۱۲ & ۲۲ & $+$۱۰ & بهبود پس از رفع تحریم \\
\hline
\end{tabular}
\end{table}

\begin{tcolorbox}[colback=yellow!5!white,colframe=yellow!75!black,
    title=یافته تطبیقی]
شواهد تطبیقی نشان می‌دهد:
\begin{enumerate}
    \item تحریم‌های بلندمدت در اکثر موارد با افزایش فساد همراه است
    \item کشورهایی که پس از رفع تحریم اصلاحات کردند (سودان) بهبود یافتند
    \item ترکیب تحریم + دولت رانتیر + اقتدارگرایی بدترین نتایج را دارد
    \item ایران از نظر ترکیب عوامل منفی، وضعیت نسبتاً نامساعدی دارد
\end{enumerate}
\end{tcolorbox}

%═══════════════════════════════════════════════════════════════════════════════
\section{راهکارها و پیشنهادها}
\label{sec:recommendations}
%═══════════════════════════════════════════════════════════════════════════════

\subsection{اصلاحات فوری (کوتاه‌مدت)}

\begin{enumerate}
    \item \textbf{یکسان‌سازی نرخ ارز:} حذف هرگونه ارز ترجیحی و یکسان‌سازی 
    نرخ ارز برای از بین بردن رانت ارزی.
    
    \item \textbf{شفافیت تجارت خارجی:} انتشار عمومی تمام آمار واردات و 
    صادرات به تفکیک کالا و واردکننده.
    
    \item \textbf{اصلاح قیمت حامل‌های انرژی:} کاهش فاصله قیمت داخلی و 
    خارجی سوخت برای کاهش انگیزه قاچاق.
    
    \item \textbf{محاکمه علنی مفسدان بزرگ:} پیگیری جدی پرونده‌های 
    کلان فساد بدون ملاحظات سیاسی.
\end{enumerate}

\subsection{اصلاحات ساختاری (میان‌مدت)}

\begin{enumerate}
    \item \textbf{استقلال قوه قضائیه:} تضمین استقلال واقعی دستگاه 
    قضایی برای مبارزه با فساد.
    
    \item \textbf{قانون دسترسی به اطلاعات:} تصویب و اجرای قانون جامع 
    آزادی اطلاعات.
    
    \item \textbf{اصلاح نظام بانکی:} بازسازی نظام بانکی، تعیین تکلیف 
    مطالبات معوق، نظارت مستقل.
    
    \item \textbf{محدودسازی فعالیت اقتصادی نهادهای شبه‌دولتی:} الزام 
    به شفافیت، پرداخت مالیات، یا واگذاری.
\end{enumerate}

\subsection{تغییرات بنیادین (بلندمدت)}

\begin{enumerate}
    \item \textbf{تغییر رابطه دولت-ملت:} گذار از دولت توزیع‌کننده رانت 
    به دولت خدمت‌رسان متکی بر مالیات.
    
    \item \textbf{صندوق ثروت ملی شفاف:} ایجاد صندوق ثروت ملی با 
    مدیریت مستقل و توزیع مستقیم سود (فصل قبل).
    
    \item \textbf{فدرالیسم مالی:} توزیع درآمدها بین مرکز و مناطق برای 
    کاهش تمرکز و فساد.
    
    \item \textbf{عادی‌سازی روابط خارجی:} تلاش برای رفع تحریم‌ها از 
    طریق مذاکره و تعامل سازنده.
\end{enumerate}

%═══════════════════════════════════════════════════════════════════════════════
\section{جمع‌بندی}
\label{sec:conclusion}
%═══════════════════════════════════════════════════════════════════════════════

این فصل نشان داد که تحریم‌های بین‌المللی علیه ایران:

\begin{enumerate}
    \item \textbf{به اهداف اعلام‌شده نرسیده‌اند:} تغییر رفتار حکومت 
    محقق نشده و برنامه‌های مورد اعتراض ادامه یافته است.
    
    \item \textbf{هزینه‌های سنگینی بر مردم عادی تحمیل کرده‌اند:} کاهش 
    رفاه، افزایش فقر، مشکلات بهداشتی.
    
    \item \textbf{فساد را نهادینه کرده‌اند:} ایجاد رانت‌های جدید، تقویت 
    اقتصاد غیررسمی، تضعیف شفافیت.
    
    \item \textbf{بازیگران جدیدی را قدرتمند کرده‌اند:} شبه‌دولتی‌های 
    نظامی، شبکه‌های دور زدن تحریم، قاچاقچیان.
    
    \item \textbf{چرخه معیوبی ایجاد کرده‌اند:} تحریم ← فساد ← تضعیف 
    نهادها ← مقاومت در برابر اصلاح ← تداوم تحریم.
\end{enumerate}

\begin{tcolorbox}[colback=blue!5!white,colframe=blue!75!black,
    title=نتیجه‌گیری نهایی]
تحریم‌ها نه راه‌حل هستند و نه بهانه. نه می‌توان همه مشکلات را به تحریم 
نسبت داد و نه می‌توان منتظر رفع تحریم برای اصلاحات ماند. اصلاحات ساختاری 
داخلی برای مبارزه با فساد، افزایش شفافیت و توزیع عادلانه ثروت، هم ممکن 
است و هم ضروری.

رفع تحریم‌ها شرط لازم اما نه کافی برای بهبود است. بدون اصلاحات نهادی، 
حتی رفع تحریم‌ها نیز به جای توزیع رفاه، به افزایش رانت منجر خواهد شد 
(مانند تجربه ۱۳۹۵-۱۳۹۷).
\end{tcolorbox}

%═══════════════════════════════════════════════════════════════════════════════
% منابع
%═══════════════════════════════════════════════════════════════════════════════
\section*{منابع و مآخذ}
\addcontentsline{toc}{section}{منابع و مآخذ}

\begin{thebibliography}{99}

\bibitem{pape1997sanctions}
Pape, R. A. (1997).
Why Economic Sanctions Do Not Work.
\textit{International Security}, 22(2), 90-136.

\bibitem{dizaji2019sanctions}
Dizaji, S. F., \& van Bergeijk, P. A. (2013).
Potential early phase success and ultimate failure of economic sanctions:
A VAR approach with an application to Iran.
\textit{Journal of Peace Research}, 50(6), 721-736.

\bibitem{ifo2016iran}
Haidar, J. I. (2017).
Sanctions and export deflection: Evidence from Iran.
\textit{Economic Policy}, 32(90), 319-355.

\bibitem{transparency2023}
Transparency International (2023).
\textit{Corruption Perceptions Index 2023}.
Berlin: Transparency International.

\bibitem{wgi2023}
World Bank (2023).
\textit{Worldwide Governance Indicators}.
Washington, DC: World Bank.

\bibitem{mazarei2019}
Mazarei, A. (2019).
Iran Has a Slow Motion Banking Crisis.
\textit{Peterson Institute for International Economics Policy Brief}.

\bibitem{fathollah2015}
Fathollah-Nejad, A. (2015).
Why Sanctions Against Iran Are Counterproductive.
\textit{International Journal}, 70(1), 72-91.

\bibitem{gordon2019}
Gordon, J. (2019).
The Not So Targeted Instrument: Sanctions and the Humanitarian Toll.
\textit{Ethics \& International Affairs}, 33(3), 291-302.

\bibitem{nephew2017}
Nephew, R. (2017).
\textit{The Art of Sanctions: A View from the Field}.
New York: Columbia University Press.

\bibitem{roshandel2020}
Roshandel, J. (2020).
Iran, Sanctions, and Regional Security.
\textit{Middle East Policy}, 27(2), 77-91.

\bibitem{motamedi2022}
Motamedi, M. (2022).
The 4200-Rial Dollar: A Case Study of Policy Failure.
\textit{Iran Economic Monitor}, World Bank.

\bibitem{khajehpour2020}
Khajehpour, B. (2020).
The Iranian Economy: Challenges and Opportunities.
\textit{Middle East Institute Report}.

\end{thebibliography}

%═══════════════════════════════════════════════════════════════════════════════
% پیوست: داده‌ها و منابع آماری
%═══════════════════════════════════════════════════════════════════════════════
\appendix
\section{پیوست: منابع داده و روش‌شناسی}
\label{app:data-sources}

\begin{table}[htbp]
\centering
\caption{منابع داده‌های مورد استفاده در این فصل}
\label{tab:data-sources}
\small
\begin{tabular}{|l|p{5cm}|p{5cm}|}
\hline
\textbf{داده} & \textbf{منبع} & \textbf{محدودیت‌ها} \\
\hline
تولید ناخالص داخلی & بانک جهانی، IMF، بانک مرکزی ایران & تفاوت در روش محاسبه \\
\hline
شاخص فساد & Transparency International & مبتنی بر ادراک، نه واقعیت \\
\hline
صادرات نفت & OPEC، EIA، منابع ردیابی نفتکش & عدم قطعیت در صادرات غیررسمی \\
\hline
نرخ ارز & بانک مرکزی، سایت‌های مالی & تفاوت نرخ‌ها \\
\hline
تحریم‌ها & OFAC، EU، UNSC & پیچیدگی و تعدد \\
\hline
فساد بانکی & گزارش‌های قضایی، رسانه‌ها & عدم دسترسی به اسناد کامل \\
\hline
اقتصاد غیررسمی & برآوردهای محققان & عدم قطعیت بالا \\
\hline
\end{tabular}
\end{table}

\begin{tcolorbox}[colback=gray!5!white,colframe=gray!75!black,
    title=یادداشت روش‌شناختی]
بسیاری از داده‌های مربوط به فساد و اقتصاد غیررسمی در ایران با عدم قطعیت 
بالا همراه است. ارقام ارائه‌شده در این فصل عمدتاً برآوردهایی هستند که بر 
اساس منابع متعدد و روش‌های مثلث‌بندی به دست آمده‌اند. خواننده باید این 
محدودیت‌ها را در نظر داشته باشد.
\end{tcolorbox}