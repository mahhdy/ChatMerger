% فصل: آناتومی فساد ساختاری و ناترازی‌های اقتصادی ایران
% نویسنده: تنظیم‌شده بر اساس تحقیقات اقتصادی
% تاریخ: بهمن ۱۴۰۴

\chapter{آناتومی فساد ساختاری و ناترازی‌های اقتصادی ایران}
\label{ch:corruption}

\begin{abstract}
اقتصاد ایران در دهه‌های اخیر با مجموعه‌ای از ناترازی‌های ساختاری مواجه بوده که هر یک به تنهایی می‌تواند بحران‌آفرین باشد، اما همزمانی آن‌ها «تقاطع بحران‌ها» را رقم زده است. این فصل به تشریح ریشه‌های اصلی فساد اقتصادی و رانت‌های ساختاری می‌پردازد و نشان می‌دهد چگونه نظام ارز ترجیحی، ناترازی بانکی، یارانه‌های پنهان انرژی، نظام مالیاتی ناکارآمد، و حضور نهادهای شبه‌دولتی غیرپاسخگو، چرخه معیوبی از فساد، تورم و فقر را تداوم بخشیده‌اند. فصل حاضر همچنین ماتریس اولویت‌بندی اصلاحات و پیش‌نیازهای گذار به اقتصاد شفاف و رقابتی را ترسیم می‌کند.
\end{abstract}

% ===========================
\section{مقدمه: تقاطع بحران‌ها}
\label{sec:intro_corruption}
% ===========================

اقتصاد ایران در آستانهٔ دهه ۱۴۰۰ شمسی وارد فازی شد که برخی اقتصاددانان آن را «تقاطع بحران‌ها» نامیده‌اند: بحران آب و فرونشست زمین، بحران انرژی (برق، گاز، بنزین)، بحران بانکی، فرسودگی زیرساخت‌ها، ناترازی صندوق‌های بازنشستگی، کسری مزمن بودجه، تورم مزمن بالای ۳۰ درصد، و کاهش مداوم ارزش پول ملی. هر یک از این بحران‌ها ریشه‌هایی مستقل دارد، اما همگی در یک نقطه مشترک به هم گره می‌خورند: \textbf{فساد ساختاری} ناشی از دخالت گسترده دولت در اقتصاد، غیاب شفافیت، و توزیع رانت به‌جای ایجاد ارزش افزوده.

فساد در ایران صرفاً معلول بی‌اخلاقی افراد نیست؛ بلکه محصول ساختارهایی است که فرصت‌های رانت‌جویی را به‌صورت سیستماتیک تولید می‌کنند. وقتی نرخ ارز چندگانه باشد، وقتی انرژی با یک‌دهم قیمت واقعی عرضه شود، وقتی بانک‌ها بتوانند بدون محدودیت از بانک مرکزی استقراض کنند، و وقتی نهادهای عظیم اقتصادی هیچ پاسخگویی به مردم و مجلس نداشته باشند، فساد نه استثنا بلکه قاعده می‌شود.

سازمان شفافیت بین‌الملل رتبه ایران را در شاخص درک فساد ۱۴۶ از ۱۸۰ کشور ارزیابی کرده است. اما این عدد تنها نوک کوه یخ است. فساد ساختاری ایران در لایه‌های عمیق‌تری ریشه دارد که در این فصل به تشریح آن‌ها می‌پردازیم.

% ===========================
\section{ستون اول: رانت ارز ترجیحی}
\label{sec:forex_rent}
% ===========================

\subsection{مکانیسم تولید رانت}

نظام چندنرخی ارز، بزرگ‌ترین و پرهزینه‌ترین منبع رانت در اقتصاد ایران بوده است. در این نظام، دولت ارز حاصل از فروش نفت را با نرخی بسیار پایین‌تر از نرخ بازار آزاد به برخی واردکنندگان اختصاص می‌دهد. تفاوت نرخ دولتی و نرخ بازار آزاد، رانت خالصی ایجاد می‌کند که بدون هیچ تولید ارزش افزوده‌ای به دریافت‌کنندگان ارز ترجیحی منتقل می‌شود.

به‌عنوان مثال، هنگامی که نرخ دولتی ارز ۴,۲۰۰ تومان و نرخ بازار آزاد بالای ۵۰,۰۰۰ تومان بود، هر دلار ارز ترجیحی معادل حدود ۴۶,۰۰۰ تومان رانت خالص تولید می‌کرد. اگر واردکننده‌ای ۱۰ میلیون دلار ارز ترجیحی دریافت می‌کرد، رانت خالص آن معادل ۴۶۰ میلیارد تومان بود --- بدون آنکه لزوماً کالایی تولید یا خدمتی ارائه کرده باشد.

\subsection{پیامدهای اقتصادی}

\begin{enumerate}
    \item \textbf{انتقال ثروت معکوس:} ارز ترجیحی ظاهراً برای «حمایت از اقشار کم‌درآمد» و ارزان نگه‌داشتن کالاهای اساسی تخصیص می‌یابد، اما در عمل بیشترین نفع آن نصیب واردکنندگان بزرگ و واسطه‌ها می‌شود. قیمت کالاها در بازار به نرخ آزاد محاسبه می‌شود، نه نرخ ترجیحی، و مابه‌التفاوت در جیب دلالان باقی می‌ماند.

    \item \textbf{تخریب تولید داخلی:} وقتی واردات با ارز ارزان انجام شود، تولیدکنندهٔ داخلی نمی‌تواند با کالای وارداتی رقابت کند. این مسئله صنایع داخلی را تضعیف و وابستگی به واردات را تشدید می‌کند.

    \item \textbf{از بین رفتن ذخایر ارزی:} تجربه دهه‌های گذشته نشان داده که هر بار ارز ترجیحی اعطا شده، نهایتاً ذخایر بانک مرکزی تحلیل رفته و شوک ارزی اجتناب‌ناپذیر شده است. همان‌طور که رئیس‌کل بانک مرکزی اعتراف کرده: «تاکنون چندین مورد ارز ترجیحی داشتیم که در نهایت دوباره مجبور به احیا و شوک قیمتی شدیم و مقدار زیادی ارز از دست دادیم.»

    \item \textbf{قاچاق و خروج سرمایه:} تفاوت چندبرابری نرخ ارز، انگیزه عظیمی برای قاچاق ارز ایجاد می‌کند. بخش قابل‌توجهی از ارز ترجیحی به‌جای واردات کالا، به بازار آزاد منتقل می‌شود.
\end{enumerate}

\subsection{هزینه مالی}

برآوردها نشان می‌دهد که طی دوره ۱۳۹۷ تا ۱۴۰۱، سالانه بین ۱۰ تا ۲۰ میلیارد دلار ارز ترجیحی تخصیص یافته که با احتساب شکاف نرخ، رانت سالانه معادل ده‌ها هزار میلیارد تومان تولید شده است. این رقم به‌تنهایی از کل بودجه عمرانی کشور بیشتر بوده است.

\subsection{راهکار: یکسان‌سازی نرخ ارز}

اقتصاددانان و حتی مسئولان بانک مرکزی بارها بر ضرورت تک‌نرخی کردن ارز تأکید کرده‌اند. این اصلاح شامل مراحل زیر است:
\begin{itemize}
    \item حذف تدریجی ارز ترجیحی و جایگزینی آن با یارانه نقدی مستقیم به خانوارهای نیازمند
    \item ایجاد شفافیت کامل در تخصیص ارز
    \item آزادسازی بازار ارز با نظارت بانک مرکزی برای جلوگیری از نوسانات شدید
    \item اتصال یکسان‌سازی ارز به اصلاح نظام یارانه‌ای (فصل قبل: توزیع مستقیم درآمد منابع طبیعی)
\end{itemize}

% ===========================
\section{ستون دوم: ناترازی نظام بانکی}
\label{sec:banking_imbalance}
% ===========================

\subsection{ابعاد ناترازی}

ناترازی بانکی به وضعیتی گفته می‌شود که مصارف بانک بیش از منابع آن باشد. این عدم تعادل در سه حوزه نمایان می‌شود:

\begin{enumerate}
    \item \textbf{ناترازی نقدینگی:} بانک توان پاسخگویی به تقاضای برداشت سپرده‌گذاران را ندارد.
    \item \textbf{ناترازی دارایی-بدهی:} بدهی‌های بانک (سپرده‌های مشتریان) از دارایی‌های واقعی آن (وام‌های قابل وصول، املاک نقدشونده) بیشتر است.
    \item \textbf{ناترازی درآمد-هزینه:} هزینه‌های بانک (سود پرداختی به سپرده‌ها) از درآمدش (سود وام‌ها) بیشتر است.
\end{enumerate}

\subsection{ریشه‌های ناترازی بانکی در ایران}

\subsubsection{سلطه مالی دولت}
کسری بودجه مزمن دولت، فشار سنگینی بر شبکه بانکی وارد می‌کند. دولت از طریق اوراق بدهی، تسهیلات تکلیفی، و فشار بر بانک‌ها برای تأمین مالی پروژه‌های دولتی، عملاً بخشی از کسری بودجه را به ترازنامه بانک‌ها منتقل می‌کند.

\subsubsection{تسهیلات تکلیفی و رانتی}
بانک‌ها ملزم به اعطای تسهیلات به بخش‌هایی می‌شوند که بازده اقتصادی ندارند. همچنین، بخش قابل‌توجهی از تسهیلات کلان به «شرکت‌های خودی» اختصاص می‌یابد. نمونه بارز، بانک آینده بود که ۹۰ درصد تسهیلات کلان خود را به شرکت‌های وابسته اختصاص داد و ۴۴ درصد کل تسهیلات را به یک پروژه واحد (ایران‌مال) پرداخت کرد که هرگز بازنگشت.

\subsubsection{نرخ سود دستوری}
تعیین دستوری نرخ سود بانکی (پایین‌تر از نرخ تورم) باعث می‌شود نرخ سود واقعی منفی شود. این امر انگیزه سپرده‌گذاری را کاهش داده و بانک‌ها را به رقابت ناسالم در جذب سپرده (با نرخ‌های غیررسمی بالاتر) سوق می‌دهد.

\subsubsection{بنگاه‌داری بانک‌ها}
بانک‌های ایرانی به‌جای تمرکز بر واسطه‌گری مالی، به بنگاه‌داری گسترده روی آورده‌اند. سرمایه‌گذاری در املاک، شرکت‌های زیرمجموعه، و فعالیت‌های غیربانکی باعث انجماد دارایی‌ها و کاهش نقدشوندگی شده است.

\subsection{پیامدهای کلان اقتصادی}

ناترازی بانکی از طریق مکانیسم «اضافه‌برداشت از بانک مرکزی» به تورم تبدیل می‌شود. بانک‌های ناتراز برای پوشش کسری خود از منابع بانک مرکزی برداشت می‌کنند و این برداشت مستقیماً به افزایش پایه پولی و در نتیجه رشد نقدینگی و تورم منجر می‌شود.

بررسی‌ها نشان می‌دهد که در دهه ۱۳۹۰، حدود ۸۷ درصد نقدینگی کشور توسط بانک‌ها خلق شده و نیمی از رشد پایه پولی نیز به‌دلیل اضافه‌برداشت بانک‌ها بوده است. به‌عبارت دیگر، بانک‌های ناتراز بیش از ۹۰ درصد از خلق پول جدید در اقتصاد را بر عهده داشته‌اند.

\subsection{ابعاد هزینه‌ای}

\begin{table}[htbp]
\centering
\caption{برآورد ابعاد مالی ناترازی‌های اقتصادی ایران (ارقام سالانه)}
\label{tab:imbalance_costs}
\begin{tabular}{|r|r|r|}
\hline
\textbf{نوع ناترازی} & \textbf{هزینه تخمینی (هزار میلیارد تومان)} & \textbf{درصد تقریبی از GDP} \\
\hline
بودجه شرکت‌های دولتی & ۳,۵۰۰ & --- \\
\hline
یارانه پنهان انرژی و قیمت‌گذاری دستوری & ۴۰۰ تا ۷۰۰ & --- \\
\hline
ناترازی صندوق‌های بازنشستگی & ۴۰۰ تا ۵۰۰ & --- \\
\hline
ناترازی بانکی (هزینه اضافه‌برداشت) & چندصد & --- \\
\hline
\end{tabular}
\end{table}

مجموع این ارقام بسیار فراتر از بودجه عمومی دولت است و نشان می‌دهد که بحران واقعی اقتصاد ایران نه در ارقام نمادین بودجه، بلکه در لایه‌های پنهان رانت و ناترازی نهفته است.

\subsection{راهکارها}

\begin{itemize}
    \item اعمال جریمه‌های بازدارنده روزانه برای اضافه‌برداشت بانک‌ها
    \item تشکیل صندوق مدیریت دارایی‌های سمی (\lr{AMC}) برای خارج‌سازی دارایی‌های منجمد از ترازنامه بانک‌ها
    \item الزام بانک‌ها به رعایت استانداردهای بازل III (حداقل نسبت کفایت سرمایه ۱۰.۵ درصد)
    \item خروج تدریجی بانک‌ها از بنگاه‌داری
    \item واگذاری سهام بانک‌های دولتی به بخش خصوصی واقعی
    \item شفافیت کامل صورت‌های مالی و انتشار عمومی جدول ناترازی بانک‌ها
\end{itemize}

% ===========================
\section{ستون سوم: یارانه‌های پنهان انرژی}
\label{sec:energy_subsidy}
% ===========================

\subsection{ابعاد یارانه پنهان}

ایران یکی از بزرگ‌ترین پرداخت‌کنندگان یارانه انرژی در جهان است. قیمت بنزین، گاز طبیعی، برق، و گازوئیل در ایران کسری از قیمت بین‌المللی آن‌هاست. صندوق بین‌المللی پول برآورد کرده که یارانه پنهان انرژی در ایران به حدود ۶۰ درصد تولید ناخالص داخلی رسیده است --- رقمی شگفت‌انگیز که ایران را در صدر فهرست کشورهای یارانه‌ده قرار می‌دهد.

\subsection{چرا یارانه انرژی فسادزاست؟}

\begin{enumerate}
    \item \textbf{قاچاق سوخت:} تفاوت قیمت سوخت ایران با کشورهای همسایه (افغانستان، پاکستان، عراق، ترکیه) انگیزه عظیمی برای قاچاق ایجاد می‌کند. سالانه میلیاردها لیتر سوخت از ایران قاچاق می‌شود.

    \item \textbf{رشد صنایع انرژی‌بر ناکارآمد:} انرژی ارزان باعث شکل‌گیری صنایعی شده که تنها با یارانه زنده‌اند. این صنایع سودآوری واقعی ندارند و اگر یارانه حذف شود، بسیاری از آن‌ها ورشکست خواهند شد. رئیس سازمان بازرسی کل کشور تصریح کرده که «یارانه‌های سنگین انرژی، سودآوری جعلی برای برخی معادن ایجاد کرده و ناکارآمدی‌ها را پنهان کرده است.»

    \item \textbf{اسراف و مصرف بی‌رویه:} ایران از نظر شدت انرژی (مصرف انرژی به ازای هر واحد تولید ناخالص داخلی) در بدترین وضعیت جهانی قرار دارد. انرژی ارزان، انگیزه‌ای برای صرفه‌جویی باقی نمی‌گذارد.

    \item \textbf{ناترازی بودجه:} یارانه انرژی بخش عظیمی از منابع مالی دولت را می‌بلعد و فضایی برای سرمایه‌گذاری عمرانی، بهداشت، و آموزش باقی نمی‌گذارد.

    \item \textbf{بحران محیط زیست:} مصرف بی‌رویه سوخت‌های فسیلی، آلودگی هوا، و تغییرات اقلیمی از پیامدهای مستقیم یارانه انرژی هستند.
\end{enumerate}

\subsection{توزیع ناعادلانه منافع}

برخلاف تصور عمومی، یارانه انرژی نه‌تنها به فقرا کمک نمی‌کند، بلکه عمدتاً نصیب دهک‌های بالای درآمدی می‌شود. خانوارهایی که خودروی شخصی، خانه بزرگ‌تر، و مصرف بالاتر دارند، سهم بسیار بیشتری از یارانه انرژی دریافت می‌کنند. پژوهش‌ها نشان می‌دهد دهک دهم درآمدی حدود ۱۰ برابر دهک اول از یارانه بنزین بهره‌مند می‌شود.

\subsection{راهکار: واقعی‌سازی قیمت انرژی}

\begin{itemize}
    \item افزایش تدریجی قیمت حامل‌های انرژی به‌سمت قیمت‌های منطقه‌ای (نه لزوماً بین‌المللی در گام اول)
    \item بازتوزیع صرفه‌جویی حاصل به‌صورت یارانه نقدی مستقیم به شهروندان (ارتباط مستقیم با فصل قبل)
    \item حمایت هدفمند از صنایع در دوره گذار
    \item سرمایه‌گذاری در انرژی‌های تجدیدپذیر و بهینه‌سازی مصرف
\end{itemize}

% ===========================
\section{ستون چهارم: نظام مالیاتی ناکارآمد}
\label{sec:tax_system}
% ===========================

\subsection{ویژگی‌های نظام مالیاتی ایران}

نظام مالیاتی ایران دارای چند ایراد ساختاری اساسی است:

\subsubsection{کارمندمحوری}
بار اصلی مالیات بر دوش حقوق‌بگیران (کارمندان و کارگران) است که مالیات آن‌ها از منبع کسر می‌شود. در مقابل، بخش‌های بزرگی از اقتصاد --- از جمله بازار، نهادهای شبه‌دولتی، و بخش غیررسمی --- یا مالیات نمی‌پردازند یا مالیات ناچیزی پرداخت می‌کنند.

\subsubsection{ممیزمحوری و فساد}
نظام مالیاتی ایران هنوز عمدتاً بر ممیزی انسانی متکی است، نه داده‌محور و الکترونیکی. این ساختار فضای گسترده‌ای برای مذاکره، رشوه، و فرار مالیاتی ایجاد می‌کند. وزیر اقتصاد تصریح کرده: «فساد الان در مالیات، گمرک و بانک به شدت وجود دارد.»

\subsubsection{نسبت مالیات به \lr{GDP} پایین}
نسبت درآمد مالیاتی به تولید ناخالص داخلی در ایران حدود ۷ تا ۹ درصد است، در حالی که میانگین جهانی حدود ۱۵ درصد و در کشورهای OECD بالای ۳۰ درصد است. این شکاف عظیم نشان‌دهنده گستردگی فرار مالیاتی و ناکارآمدی نظام وصول است.

\subsubsection{معافیت‌های گسترده}
بخش‌های وسیعی از اقتصاد از معافیت مالیاتی برخوردارند: نهادهای نظامی، بنیادهای شبه‌دولتی، بخش کشاورزی، مناطق آزاد تجاری، و بسیاری از صنایع وابسته به نهادهای قدرت. این معافیت‌ها هم ناعادلانه‌اند و هم پایه مالیاتی را به‌شدت محدود می‌کنند.

\subsubsection{تبعیض جغرافیایی}
مالیات وصول‌شده در استان‌ها لزوماً به همان استان بازنمی‌گردد. این ساختار هم ضد عدالت است و هم ضد انگیزه توسعه محلی.

\subsection{راهکارها}

\begin{itemize}
    \item هوشمندسازی و دیجیتالی‌سازی کامل نظام مالیاتی
    \item ایجاد پایگاه اطلاعات اقتصادی ایرانیان با استفاده از هوش مصنوعی و داده‌کاوی
    \item حذف ممیزمحوری و جایگزینی آن با اظهارنامه الکترونیکی و تطبیق خودکار
    \item لغو معافیت‌های ناموجه (به‌ویژه برای نهادهای شبه‌دولتی و بنیادها)
    \item اصلاح نظام توزیع مالیات بین استان‌ها (ارتباط مستقیم با فصل فدرالیسم مالی)
    \item کاهش مالیات بنگاه‌هایی که اشتغال حفظ یا افزایش می‌دهند
\end{itemize}

% ===========================
\section{ستون پنجم: نهادهای شبه‌دولتی غیرپاسخگو}
\label{sec:quasi_governmental}
% ===========================

\subsection{نقشه نهادهای اقتصادی خارج از نظارت}

یکی از منحصربه‌فردترین ویژگی‌های اقتصاد ایران، حضور گسترده نهادهای اقتصادی بزرگ است که نه دولتی‌اند و نه خصوصی، و عملاً خارج از نظارت دولت، مجلس، و دیوان محاسبات فعالیت می‌کنند.

\begin{table}[htbp]
\centering
\caption{مهم‌ترین نهادهای اقتصادی شبه‌دولتی}
\label{tab:quasi_gov}
\begin{tabular}{|r|r|r|}
\hline
\textbf{نهاد} & \textbf{حوزه فعالیت} & \textbf{پاسخگویی} \\
\hline
بنیاد مستضعفان (۱۲ هلدینگ، ۱۷۵ شرکت) & همه بخش‌ها & رهبری \\
\hline
ستاد اجرایی فرمان امام & سرمایه‌گذاری، بانکداری & رهبری \\
\hline
آستان قدس رضوی & کشاورزی تا صنعت & رهبری \\
\hline
بنیاد شهید و ایثارگران (۴ هلدینگ، ۱۹ شرکت) & خدمات اجتماعی و اقتصادی & رهبری \\
\hline
سازمان تأمین اجتماعی (۱۶ شرکت و هلدینگ) & بیمه، بهداشت، سرمایه‌گذاری & دولت (محدود) \\
\hline
قرارگاه خاتم‌الانبیاء (سپاه) & عمران، نفت، انرژی & نظامی \\
\hline
\end{tabular}
\end{table}

\subsection{مشکلات ساختاری}

\begin{enumerate}
    \item \textbf{غیاب شفافیت مالی:} نماینده مجلس اعلام کرده از مجموع ۳۲۰ شرکت غیردولتی وابسته به این نهادها، فقط ۴ درصد صورت‌های مالی خود را ارائه کرده‌اند.

    \item \textbf{رقابت ناعادلانه با بخش خصوصی:} این نهادها با برخورداری از معافیت مالیاتی، دسترسی ترجیحی به ارز و تسهیلات، و حمایت‌های سیاسی، رقابت را برای بخش خصوصی واقعی غیرممکن می‌کنند. حجم فعالیت آن‌ها از کل بخش خصوصی ایران بزرگ‌تر تخمین زده می‌شود.

    \item \textbf{ورشکستگی صنایع:} بنیادها با تصاحب کارخانه‌ها و مدیریت ناکارآمد، بسیاری از واحدهای صنعتی را به تعطیلی کشانده‌اند.

    \item \textbf{موازی‌کاری در خدمات رفاهی:} حدود ۳۰ صندوق دولتی و شبه‌دولتی برای کاهش فقر فعالیت می‌کنند اما چون در چارچوب برنامه جامع نیستند، فقرزدایی آن‌ها ناکارآمد است.

    \item \textbf{تضعیف حاکمیت دولت:} بخش قابل‌توجهی از اقتصاد خارج از کنترل دولت منتخب قرار دارد، که این مسئله اجرای هرگونه سیاست اقتصادی منسجم را ناممکن می‌سازد.
\end{enumerate}

\subsection{بنیاد مستضعفان: مطالعه موردی}

بنیاد مستضعفان پس از شرکت ملی نفت ایران، دومین مؤسسه اقتصادی بزرگ ایران محسوب می‌شود و یکی از بزرگ‌ترین هلدینگ‌های خاورمیانه است. این بنیاد در حوزه‌های انتشارات، تولید برق، جهانگردی، حمل‌ونقل، بیمه، بانکداری، پیمانکاری، و بازرگانی فعالیت می‌کند.

از نظر حقوقی، بنیاد نه دولتی است و نه خصوصی. مؤسسه‌ای ناسودبر است که دولت حق دخالت در آن را ندارد و تنها به رهبر پاسخگوست. این ساختار حقوقی بی‌نظیر، یک «جزیره اقتصادی» خارج از تمام سازوکارهای نظارتی ایجاد کرده است.

\subsection{راهکارها}

\begin{itemize}
    \item الزام تمام نهادهای اقتصادی (بدون استثنا) به ارائه صورت‌های مالی حسابرسی‌شده و شفاف
    \item لغو معافیت‌های مالیاتی نهادهای شبه‌دولتی
    \item ادغام فعالیت‌های رفاهی این نهادها در چارچوب برنامه جامع دولت
    \item واگذاری تدریجی شرکت‌های اقتصادی این نهادها به بخش خصوصی واقعی (نه خصوصی‌سازی صوری)
    \item ایجاد نظام نظارتی مستقل برای تمام فعالیت‌های اقتصادی نهادهای زیر نظر رهبری
\end{itemize}

% ===========================
\section{ستون ششم: قیمت‌گذاری دستوری و خلق ناترازی}
\label{sec:price_controls}
% ===========================

قیمت‌گذاری دستوری --- تعیین قیمت کالاها و خدمات پایین‌تر از بهای تمام‌شده --- یکی دیگر از ریشه‌های ناترازی اقتصادی ایران است. این سیاست در بلندمدت زنجیره‌ای از پیامدها ایجاد می‌کند:

\begin{enumerate}
    \item شرکت‌ها و دستگاه‌های اجرایی عرضه‌کننده زیان می‌کنند.
    \item دولت باید زیان را جبران کند، اما اغلب در همان سال انجام نمی‌شود.
    \item زیان‌ها انباشت شده و به‌صورت بدهی در شبکه بانکی ظاهر می‌شود.
    \item بانک‌ها برای جبران، از بانک مرکزی اضافه‌برداشت می‌کنند.
    \item پایه پولی افزایش یافته و تورم ایجاد می‌شود.
    \item تورم باعث می‌شود قیمت‌های دستوری بازهم از واقعیت فاصله بگیرند.
\end{enumerate}

این چرخه معیوب، یکی از عوامل پنهان اما اثرگذار در ایجاد ناترازی‌های گسترده در شبکه بانکی و مالی کشور است. راه‌حل، حرکت تدریجی به‌سمت آزادسازی قیمت‌ها همراه با حمایت مستقیم از خانوارهای آسیب‌پذیر است.

% ===========================
\section{چرخه معیوب: چگونه ناترازی‌ها یکدیگر را تقویت می‌کنند}
\label{sec:vicious_cycle}
% ===========================

ناترازی‌های اقتصادی ایران به‌صورت مجزا عمل نمی‌کنند بلکه یکدیگر را تغذیه و تشدید می‌کنند. نمودار زیر این چرخه معیوب را نشان می‌دهد:

\begin{center}
\fbox{\parbox{0.9\textwidth}{
\textbf{چرخه معیوب ناترازی‌های اقتصادی ایران:}

\medskip
\textbf{کسری بودجه دولت} $\rightarrow$ فشار بر بانک‌ها برای تأمین مالی $\rightarrow$ \textbf{ناترازی بانکی} $\rightarrow$ اضافه‌برداشت از بانک مرکزی $\rightarrow$ \textbf{رشد نقدینگی و تورم} $\rightarrow$ کاهش ارزش پول ملی $\rightarrow$ \textbf{افزایش شکاف ارز ترجیحی و آزاد} $\rightarrow$ رانت بیشتر $\rightarrow$ فساد بیشتر $\rightarrow$ \textbf{ناکارآمدی اقتصاد} $\rightarrow$ کاهش درآمد مالیاتی $\rightarrow$ \textbf{کسری بودجه بیشتر}

\medskip
همزمان: \textbf{یارانه انرژی} $\rightarrow$ قاچاق + اسراف + ناترازی انرژی $\rightarrow$ کسری بودجه بیشتر

\medskip
و: \textbf{نهادهای شبه‌دولتی} $\rightarrow$ رقابت ناعادلانه $\rightarrow$ تضعیف بخش خصوصی $\rightarrow$ کاهش رشد اقتصادی و اشتغال $\rightarrow$ فقر بیشتر $\rightarrow$ وابستگی بیشتر مردم به دولت
}}
\end{center}

شکستن این چرخه معیوب مستلزم اصلاحات هم‌زمان و هماهنگ در چندین حوزه است. اصلاح یک بخش بدون اصلاح بخش‌های مرتبط، نتیجه پایداری نخواهد داشت.

% ===========================
\section{ماتریس اولویت‌بندی اصلاحات}
\label{sec:reform_matrix}
% ===========================

با توجه به محدودیت منابع سیاسی و اقتصادی، اولویت‌بندی اصلاحات ضروری است. ماتریس زیر بر اساس دو معیار «تأثیر اقتصادی» و «امکان‌پذیری سیاسی» تنظیم شده است:

\begin{table}[htbp]
\centering
\caption{ماتریس اولویت‌بندی اصلاحات اقتصادی}
\label{tab:reform_priority}
\begin{tabular}{|r|r|r|r|r|}
\hline
\textbf{اصلاح} & \textbf{تأثیر} & \textbf{امکان‌پذیری} & \textbf{افق زمانی} & \textbf{اولویت} \\
\hline
یکسان‌سازی نرخ ارز & بسیار بالا & متوسط & کوتاه‌مدت & ۱ \\
\hline
شفافیت مالی نظام بانکی & بسیار بالا & متوسط & کوتاه‌مدت & ۲ \\
\hline
هوشمندسازی نظام مالیاتی & بالا & بالا & میان‌مدت & ۳ \\
\hline
واقعی‌سازی قیمت انرژی & بسیار بالا & پایین & میان‌مدت & ۴ \\
\hline
اصلاح ساختار بانکی (بازل III) & بسیار بالا & متوسط & میان‌مدت & ۵ \\
\hline
شفافیت نهادهای شبه‌دولتی & بالا & پایین & بلندمدت & ۶ \\
\hline
خصوصی‌سازی واقعی & بالا & پایین & بلندمدت & ۷ \\
\hline
اصلاح صندوق‌های بازنشستگی & بسیار بالا & پایین & بلندمدت & ۸ \\
\hline
\end{tabular}
\end{table}

% ===========================
\section{پیش‌شرط‌های اصلاحات: اراده سیاسی و شفافیت}
\label{sec:prerequisites}
% ===========================

\subsection{فساد یک امر سیاسی است}

فساد در ایران نهادینه شده و مبارزه با آن بیش از هر چیز به \textbf{اراده سیاسی} نیاز دارد. تصویب قوانین جدید به‌تنهایی کافی نیست؛ بلکه باید «قوانین حمایتی که فساد می‌زایند» برداشته شود. مقررات‌زدایی هوشمند، کاهش اصطکاک مردم با بروکراسی، و شفافیت داده‌محور، سه کلید اصلی مبارزه با فساد ساختاری هستند.

\subsection{شفافیت به‌عنوان زیربنا}

بدون شفافیت، هیچ اصلاحی پایدار نخواهد بود. اقدامات زیربنایی شفافیت عبارتند از:

\begin{itemize}
    \item ایجاد پایگاه داده یکپارچه اقتصادی شامل تمام تراکنش‌های مالی، مالیاتی، ارزی، و بانکی
    \item الزام تمام نهادهای اقتصادی (بدون استثنا) به انتشار عمومی صورت‌های مالی حسابرسی‌شده
    \item دیجیتالی‌سازی کامل فرآیندهای دولتی برای حذف فضای مذاکره و رشوه
    \item آزادی رسانه و فعالیت نهادهای نظارت مردمی
    \item عضویت در ابتکارهای بین‌المللی شفافیت (مانند \lr{EITI} برای شفافیت صنایع استخراجی)
\end{itemize}

\subsection{نقش فناوری}

وزارت اقتصاد اعلام کرده که پایگاه اطلاعات اقتصادی ایرانیان با کمک هوش مصنوعی و داده‌کاوی در حال ساخت است و قرار است تمام تسهیلات، ارز، بیمه، و یارانه بر اساس این پایگاه باشد. این حرکت اگر واقعاً اجرا شود، می‌تواند گام بزرگی در مسیر شفافیت باشد، اما تجربه نشان داده که اجرای چنین سیستم‌هایی در غیاب اراده سیاسی و مقاومت ذی‌نفعان رانت، با چالش‌های جدی مواجه خواهد شد.

% ===========================
\section{پیوند اصلاحات: رویکرد یکپارچه}
\label{sec:integrated_reform}
% ===========================

همان‌طور که در فصل قبل (توزیع مستقیم درآمد منابع طبیعی) تشریح شد، اصلاحات اقتصادی ایران نمی‌توانند به‌صورت جزیره‌ای انجام شوند. بسته اصلاحی یکپارچه شامل اجزای زیر است:

\begin{enumerate}
    \item \textbf{یکسان‌سازی نرخ ارز + توزیع مستقیم:} حذف ارز ترجیحی و جایگزینی آن با پرداخت نقدی مستقیم به شهروندان از محل درآمد نفت.

    \item \textbf{واقعی‌سازی قیمت انرژی + یارانه نقدی:} افزایش قیمت انرژی به سطح منطقه‌ای و بازتوزیع صرفه‌جویی حاصل.

    \item \textbf{اصلاح بانکی + شفافیت:} ساماندهی بانک‌های ناتراز، اعمال استانداردهای بازل، و شفافیت کامل صورت‌های مالی.

    \item \textbf{اصلاح مالیاتی + فدرالیسم مالی:} هوشمندسازی نظام مالیاتی، لغو معافیت‌ها، و بازتوزیع عادلانه مالیات بین استان‌ها (موضوع فصل بعد).

    \item \textbf{خصوصی‌سازی واقعی + مقررات‌زدایی:} واگذاری واقعی شرکت‌های دولتی و شبه‌دولتی به بخش خصوصی مستقل همراه با حذف مقررات زائد.
\end{enumerate}

% ===========================
\section{جمع‌بندی}
\label{sec:conclusion_corruption}
% ===========================

فساد ساختاری در ایران محصول عوامل تصادفی یا صرفاً بی‌اخلاقی افراد نیست. این فساد ریشه در ساختارهایی دارد که به‌صورت سیستماتیک رانت تولید می‌کنند: ارز چندنرخی، انرژی زیر قیمت، بانک‌داری بدون نظارت، مالیات‌ستانی ناعادلانه، و نهادهای اقتصادی غیرپاسخگو. مجموع هزینه این ناترازی‌ها از کل بودجه رسمی دولت فراتر می‌رود و فشار نهایی آن بر دوش مردم عادی --- از طریق تورم، کاهش ارزش پول ملی، و فقر --- است.

اصلاح این وضعیت نیازمند رویکردی یکپارچه، شجاعانه، و مبتنی بر شفافیت است. هر اصلاحی بدون توجه به ارتباطات آن با سایر حوزه‌ها، محکوم به شکست خواهد بود. کلید موفقیت در این است که اصلاحات نه از بالا به پایین، بلکه با مشارکت مردم و توزیع مستقیم منافع به شهروندان اجرا شود --- تا ائتلاف اجتماعی لازم برای غلبه بر مقاومت ذی‌نفعان رانت شکل بگیرد.

% ===========================
\section*{منابع}
% ===========================

\begin{enumerate}
    \item سازمان شفافیت بین‌المللی (\lr{Transparency International}), \textit{Corruption Perceptions Index}, 2019--2024.

    \item صندوق بین‌المللی پول (\lr{IMF}), \textit{Energy Subsidies in Iran: Reform Challenges and Fiscal Implications}, 2023.

    \item شقاقی شهری، وحید، «تقاطع بحران‌ها و ناترازی‌های اقتصادی ایران»، مصاحبه با فرارو، آذر ۱۴۰۴.

    \item همتی، عبدالناصر، سخنرانی رئیس‌کل بانک مرکزی درباره ارز ترجیحی و ناترازی بانکی، دی ۱۴۰۴.

    \item مدنی‌زاده، سید مهدی (وزیر اقتصاد)، سخنرانی در مراسم روز دانشجو، آذر ۱۴۰۴.

    \item خداییان، ذبیح‌الله (رئیس سازمان بازرسی)، «۲۵ ایراد حکمرانی معدن: یارانه پنهان و سودآوری جعلی»، آذر ۱۴۰۴.

    \item کولیوند، محمدجواد (نماینده مجلس)، «۳۲۰ شرکت غیردولتی و فقط ۴ درصد شفافیت مالی»، مرداد ۱۳۹۷.

    \item ویکی‌پدیا فارسی، \textit{بنیاد مستضعفان انقلاب اسلامی}؛ \textit{فساد در ایران}.

    \item \lr{Ross, Michael L.}, \textit{The Oil Curse: How Petroleum Wealth Shapes the Development of Nations}, Princeton University Press, 2012.

    \item \lr{Karl, Terry Lynn}, \textit{The Paradox of Plenty: Oil Booms and Petro-States}, University of California Press, 1997.

    \item \lr{Acemoglu, Daron \& Robinson, James A.}, \textit{Why Nations Fail: The Origins of Power, Prosperity, and Poverty}, Crown Business, 2012.

    \item بانکر (\lr{Banker.ir}), «چرا بانک‌ها همیشه کم می‌آورند؟ داستان تکراری ناترازی»، بهمن ۱۴۰۴.

    \item آفتاب‌نیوز، «چگونه بانک‌های خصوصی اقتصاد ایران را به بحران کشاندند؟»، ۱۴۰۴.

    \item اقتصاد آنلاین، «یارانه پنهان انرژی؛ تله شوم اقتصاد ایران»، ۱۴۰۳.

    \item تابناک، «چالش جدید دولت؛ کدام اعداد پنهان می‌شوند؟»، دی ۱۴۰۴.
\end{enumerate}

\end{document}
