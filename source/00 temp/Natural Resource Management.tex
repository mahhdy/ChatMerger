
\chapter{دولت، ملت و مالکیت منابع طبیعی: تحلیل تطبیقی نظام‌های جهانی}
\label{chap:natural-resources-ownership}

\begin{abstract}
این فصل به بررسی تطبیقی نظام‌های مالکیت منابع طبیعی در جهان می‌پردازد. 
با تحلیل تجربیات کشورهای مختلف از آمریکا تا نروژ و از ایران تا بوتسوانا، 
الگوهای متفاوت مالکیت، مدیریت و توزیع منافع حاصل از منابع طبیعی شناسایی 
و طبقه‌بندی می‌شوند. همچنین دیدگاه‌های متخصصان ایرانی و خاورمیانه‌ای 
در این زمینه مرور می‌گردد.
\end{abstract}

%═══════════════════════════════════════════════════════════════════════════════
\section{مقدمه: پرسش بنیادین مالکیت}
\label{sec:ownership-question}
%═══════════════════════════════════════════════════════════════════════════════

پرسش «منابع طبیعی متعلق به کیست؟» یکی از بنیادی‌ترین پرسش‌های فلسفه 
سیاسی و حقوق است. پاسخ به این پرسش، کل ساختار سیاسی-اقتصادی یک جامعه 
را شکل می‌دهد:

\begin{itemize}
    \item اگر منابع متعلق به \textbf{دولت} باشد: دولت بزرگ، وابستگی 
    شهروندان، امکان استبداد
    \item اگر منابع متعلق به \textbf{افراد خصوصی} باشد: نابرابری شدید، 
    انباشت ثروت، قدرت الیگارشی
    \item اگر منابع متعلق به \textbf{همه شهروندان} باشد: نیاز به 
    مکانیزم‌های توزیع عادلانه و شفاف
\end{itemize}

\begin{tcolorbox}[colback=blue!5!white,colframe=blue!75!black,
    title=پرسش‌های کلیدی این فصل]
\begin{enumerate}
    \item آیا منابع زیرزمینی باید تابع مالکیت زمین سطحی باشد؟
    \item نقش دولت در مدیریت منابع چیست: مالک، امانتدار، یا تنظیم‌گر؟
    \item عدالت بین‌نسلی در بهره‌برداری از منابع تجدیدناپذیر چگونه تأمین می‌شود؟
    \item چه مکانیزم‌هایی برای جلوگیری از «نفرین منابع» وجود دارد؟
\end{enumerate}
\end{tcolorbox}

%═══════════════════════════════════════════════════════════════════════════════
\section{نظریه‌پردازان و متخصصان کلیدی}
\label{sec:key-theorists}
%═══════════════════════════════════════════════════════════════════════════════

\subsection{متخصصان ایرانی}

\subsubsection{دکتر پویا ناظران}
اقتصاددان ایرانی مقیم آمریکا که دیدگاه‌های او در فصل قبل تشریح شد. 
ایده‌های کلیدی:
\begin{itemize}
    \item تفکیک «دارایی» از «درآمد» در منابع طبیعی
    \item تبدیل صندوق توسعه ملی به صندوق ثروت ملی واقعی
    \item توزیع مستقیم سود به شهروندان
    \item پرهیز از پیش‌خور کردن منابع نسل‌های آینده
\end{itemize}

\subsubsection{دکتر محسن رنانی}
اقتصاددان نهادگرای ایرانی، استاد دانشگاه اصفهان:

\begin{quote}
\textit{«دولت رانتیر ایران به جای اتکا به مالیات شهروندان، به درآمد نفت 
وابسته است. این وابستگی، رابطه دولت-ملت را معکوس می‌کند: به جای اینکه 
دولت پاسخگوی مردم باشد، مردم وابسته به دولت می‌شوند.»}
\end{quote}

ایده‌های کلیدی رنانی:
\begin{itemize}
    \item نقد ساختاری «دولت رانتیر»
    \item تأکید بر اهمیت سرمایه اجتماعی و اعتماد عمومی
    \item لزوم گذار از اقتصاد نفتی به اقتصاد مالیاتی
    \item توسعه نهادهای مدنی به عنوان پیش‌شرط توسعه اقتصادی
\end{itemize}

\subsubsection{دکتر حسین راغفر}
اقتصاددان رفاه، استاد دانشگاه الزهرا:
\begin{itemize}
    \item تأکید بر عدالت توزیعی در درآمدهای نفتی
    \item نقد یارانه‌های غیرهدفمند
    \item طرفداری از یارانه نقدی هدفمند برای دهک‌های پایین
    \item مطالعات گسترده درباره فقر و نابرابری در ایران
\end{itemize}

\subsubsection{دکتر جواد صالحی‌اصفهانی}
اقتصاددان ایرانی-آمریکایی، استاد دانشگاه ویرجینیا تک:
\begin{itemize}
    \item مطالعات تطبیقی ایران با سایر کشورهای نفتی
    \item تحلیل اثرات بلندمدت درآمدهای نفتی بر توسعه
    \item بررسی نقش نهادها در عملکرد اقتصادی
\end{itemize}

\subsubsection{دکتر موسی غنی‌نژاد}
اقتصاددان لیبرال ایرانی:
\begin{itemize}
    \item طرفداری از خصوصی‌سازی گسترده
    \item کاهش نقش دولت در اقتصاد
    \item انتقاد از دخالت‌های دولتی در قیمت‌گذاری
    \item تأکید بر حقوق مالکیت خصوصی
\end{itemize}

\subsection{متخصصان خاورمیانه و جهان}

\subsubsection{حازم الببلاوی (مصر) - نظریه دولت رانتیر}

الببلاوی در دهه ۱۹۸۰ نظریه «دولت رانتیر» 
\LTRfootnote{Rentier State Theory} را مطرح کرد:

\begin{tcolorbox}[colback=yellow!5!white,colframe=yellow!75!black,
    title=ویژگی‌های دولت رانتیر]
\begin{enumerate}
    \item \textbf{رانت خارجی:} درآمد اصلی از منابع خارجی (صادرات نفت) نه 
    از فعالیت اقتصادی داخلی
    \item \textbf{تمرکز دریافت:} رانت مستقیماً به دولت می‌رسد نه به 
    اقتصاد خصوصی
    \item \textbf{اقلیت تولیدکننده:} فقط بخش کوچکی از جامعه در تولید 
    رانت دخیل است
    \item \textbf{دولت توزیع‌کننده:} دولت نقش اصلی را در توزیع ثروت دارد 
    نه تولید آن
\end{enumerate}
\end{tcolorbox}

پیامدهای سیاسی دولت رانتیر:
\begin{itemize}
    \item کاهش پاسخگویی دموکراتیک («بدون مالیات، بدون نمایندگی»)
    \item تضعیف جامعه مدنی
    \item گسترش دستگاه بوروکراتیک و امنیتی
    \item فساد و رانت‌جویی سیستماتیک
\end{itemize}

\subsubsection{دارون عجم‌اوغلو (ترکیه) - نهادها و توسعه}

برنده جایزه نوبل اقتصاد ۲۰۲۴ (همراه با جیمز رابینسون)، در کتاب 
معروف «چرا ملت‌ها شکست می‌خورند» استدلال می‌کند:

\begin{quote}
\textit{«منابع طبیعی فراوان لزوماً نعمت نیست. در غیاب نهادهای فراگیر 
\LTRfootnote{Inclusive Institutions}، منابع طبیعی می‌توانند به تقویت 
نهادهای استخراجی \LTRfootnote{Extractive Institutions} و تداوم فقر 
و استبداد منجر شوند.»}
\end{quote}

تفاوت نهادهای فراگیر و استخراجی:

\begin{table}[htbp]
\centering
\caption{مقایسه نهادهای فراگیر و استخراجی}
\label{tab:institutions-comparison}
\begin{tabular}{|p{6cm}|p{6cm}|}
\hline
\textbf{نهادهای فراگیر} & \textbf{نهادهای استخراجی} \\
\hline
حقوق مالکیت امن برای همه & حقوق مالکیت ناامن یا محدود به نخبگان \\
\hline
فرصت برابر اقتصادی & انحصار و رانت برای خودی‌ها \\
\hline
دولت پاسخگو و محدود & دولت خودکامه و نامحدود \\
\hline
رقابت سیاسی باز & انحصار قدرت سیاسی \\
\hline
نوآوری و «تخریب خلاق» & حفظ وضع موجود و مقاومت در برابر تغییر \\
\hline
\end{tabular}
\end{table}

\subsubsection{مایکل راس (آمریکا) - نفرین نفت}

راس در کتاب «نفرین نفت» (۲۰۱۲) با داده‌های گسترده نشان می‌دهد:
\begin{itemize}
    \item کشورهای نفتی ۵۰٪ بیشتر احتمال دارد که اقتدارگرا باشند
    \item نفت جنگ داخلی را طولانی‌تر و خونین‌تر می‌کند
    \item درآمد نفت با نابرابری جنسیتی همبستگی مثبت دارد
    \item اثرات منفی نفت در کشورهای فقیرتر شدیدتر است
\end{itemize}

\subsubsection{تری لین کارل (آمریکا) - پارادوکس فراوانی}

کارل در کتاب «پارادوکس فراوانی» (۱۹۹۷) مکانیزم‌های «نفرین منابع» را 
تشریح می‌کند:
\begin{itemize}
    \item \textbf{اثر جابجایی:} درآمد آسان نفتی انگیزه توسعه بخش‌های 
    دیگر را از بین می‌برد
    \item \textbf{بیماری هلندی:} تقویت پول ملی و آسیب به صادرات غیرنفتی
    \item \textbf{نوسان درآمد:} بی‌ثباتی بودجه و برنامه‌ریزی
    \item \textbf{وابستگی مسیر:} تصمیمات اولیه ساختارهای بعدی را قفل می‌کند
\end{itemize}

%═══════════════════════════════════════════════════════════════════════════════
\section{نظام‌های حقوقی مالکیت منابع طبیعی}
\label{sec:legal-systems}
%═══════════════════════════════════════════════════════════════════════════════

\subsection{طبقه‌بندی کلی نظام‌های مالکیت}

\begin{figure}[htbp]
\centering
\begin{tikzpicture}[
    level 1/.style={sibling distance=6cm, level distance=2cm},
    level 2/.style={sibling distance=3cm, level distance=2cm},
    every node/.style={rectangle, draw, align=center, font=\small}
]
\node {مالکیت منابع طبیعی}
    child {node {مالکیت خصوصی}
        child {node {کامل\\(تگزاس)}}
        child {node {مشروط\\(استرالیا)}}
    }
    child {node {مالکیت عمومی}
        child {node {دولتی متمرکز\\(ایران)}}
        child {node {دولتی فدرال\\(کانادا)}}
        child {node {شهروندی\\(آلاسکا)}}
    }
    child {node {مالکیت مشترک}
        child {node {اشتراکی محلی\\(بومیان)}}
        child {node {بین‌المللی\\(اقیانوس‌ها)}}
    }
;
\end{tikzpicture}
\caption{طبقه‌بندی نظام‌های مالکیت منابع طبیعی}
\label{fig:ownership-taxonomy}
\end{figure}

\subsection{ایالات متحده آمریکا: استثنای جهانی}

آمریکا \textbf{تنها کشور بزرگ صنعتی} است که مالکیت خصوصی منابع 
زیرزمینی را به طور گسترده به رسمیت می‌شناسد. این ویژگی ریشه در 
تاریخ استعماری و انقلابی آمریکا دارد.

\subsubsection{تاریخچه حقوقی}

\begin{itemize}
    \item \textbf{قانون عمومی انگلیسی:} در انگلستان، معادن طلا و نقره 
    «معادن سلطنتی» بودند، اما سایر معادن متعلق به مالک زمین بود.
    
    \item \textbf{انقلاب آمریکا:} رد ادعاهای سلطنتی؛ تأکید بر حقوق 
    مالکیت فردی.
    
    \item \textbf{قانون معادن ۱۸۷۲:} اجازه ادعای مالکیت معادن در 
    زمین‌های عمومی به کاشفان.
    
    \item \textbf{قاعده تصرف (Rule of Capture):} در تگزاس و برخی ایالات، 
    هر کس نفت یا گاز را استخراج کند مالک آن است، حتی اگر از زیر زمین 
    همسایه بیاید!
\end{itemize}

\subsubsection{ساختار فعلی مالکیت در آمریکا}

\begin{table}[htbp]
\centering
\caption{مالکیت منابع طبیعی در ایالات متحده}
\label{tab:us-ownership}
\begin{tabular}{|l|l|p{7cm}|}
\hline
\textbf{نوع منبع} & \textbf{مالکیت} & \textbf{توضیحات} \\
\hline
نفت و گاز & خصوصی/ایالتی/فدرال & مالک زمین معمولاً مالک منابع 
زیرزمینی است (قابل جداسازی) \\
\hline
زغال‌سنگ & خصوصی/ایالتی/فدرال & مشابه نفت و گاز \\
\hline
فلزات گرانبها & خصوصی/فدرال & در زمین خصوصی: خصوصی؛ در زمین 
فدرال: نیاز به امتیاز \\
\hline
آب سطحی & ایالتی متفاوت & شرق: حقوق ساحلی؛ غرب: تخصیص 
تاریخی (هر که زودتر) \\
\hline
آب زیرزمینی & خصوصی/ایالتی & تگزاس: قاعده تصرف مطلق؛ سایر: 
«استفاده معقول» \\
\hline
جنگل‌ها & خصوصی/ایالتی/فدرال & ۵۸٪ خصوصی، ۳۳٪ فدرال \\
\hline
فلات قاره & فدرال & درآمد به خزانه فدرال \\
\hline
\end{tabular}
\end{table}

\subsubsection{حقوق معدنی جداشدنی (Severed Mineral Rights)}

یکی از ویژگی‌های منحصربه‌فرد نظام آمریکایی، امکان جداسازی حقوق معدنی 
از حقوق سطحی زمین است:

\begin{tcolorbox}[colback=orange!5!white,colframe=orange!75!black,
    title=مثال: جداسازی حقوق معدنی]
آقای اسمیت مالک یک مزرعه ۱۰۰ هکتاری است. او می‌تواند:
\begin{enumerate}
    \item زمین سطحی را به آقای جونز بفروشد
    \item حقوق معدنی (نفت، گاز، معادن) را برای خود نگه دارد
    \item حقوق معدنی را به شرکت نفتی اجاره دهد
\end{enumerate}
نتیجه: جونز مالک مزرعه است اما اسمیت می‌تواند اجازه حفاری در آن را 
به شرکت نفتی بدهد!
\end{tcolorbox}

این سیستم منجر به پیچیدگی‌های فراوان شده:
\begin{itemize}
    \item در تگزاس، میلیون‌ها قطعه زمین دارای مالکیت معدنی جداگانه هستند
    \item برخی حقوق معدنی دهه‌ها پیش جدا شده و ردیابی مالکان دشوار است
    \item «معدنکاوان سطحی» حق دارند زمین را برای دسترسی به معادن 
    حفاری کنند
\end{itemize}

\subsubsection{آلاسکا: استثنا در استثنا}

آلاسکا مدلی متفاوت از سایر ایالات دارد:

\begin{itemize}
    \item \textbf{مالکیت دولتی:} منابع نفت و گاز عمدتاً در زمین‌های 
    ایالتی یا فدرال است
    \item \textbf{صندوق دائمی آلاسکا (APF):} حداقل ۲۵٪ درآمدهای نفتی 
    به صندوق واریز می‌شود
    \item \textbf{سود سالانه (PFD):} هر ساکن آلاسکا سود سالانه دریافت 
    می‌کند (۱۰۰۰-۳۰۰۰ دلار)
\end{itemize}

\begin{figure}[htbp]
\centering
\begin{tikzpicture}
\begin{axis}[
    ybar,
    xlabel={سال},
    ylabel={سود سالانه (دلار)},
    symbolic x coords={2015,2016,2017,2018,2019,2020,2021,2022,2023},
    xtick=data,
    nodes near coords,
    nodes near coords align={vertical},
    width=14cm,
    height=7cm,
    bar width=0.6cm,
    ymin=0,
    ymax=4000,
]
\addplot coordinates {(2015,2072) (2016,1022) (2017,1100) (2018,1600) 
    (2019,1606) (2020,992) (2021,1114) (2022,3284) (2023,1312)};
\end{axis}
\end{tikzpicture}
\caption{سود سالانه صندوق دائمی آلاسکا به هر شهروند}
\label{fig:alaska-pfd}
\end{figure}

\subsection{نروژ: مدل طلایی دولت رفاه}

\subsubsection{اصول حقوقی}

\begin{itemize}
    \item \textbf{مالکیت دولتی کامل:} تمام منابع نفت و گاز فلات قاره 
    متعلق به دولت است
    \item \textbf{امتیازدهی شفاف:} شرکت‌ها برای اکتشاف و استخراج 
    امتیاز می‌گیرند
    \item \textbf{شرکت ملی قوی:} Equinor (سابقاً Statoil) با مالکیت 
    ۶۷٪ دولتی
    \item \textbf{مالیات بالا:} نرخ مالیات ۷۸٪ بر سود نفتی
\end{itemize}

\subsubsection{صندوق بازنشستگی دولتی (GPFG)}

\begin{tcolorbox}[colback=green!5!white,colframe=green!75!black,
    title=آمار صندوق نروژ (۲۰۲۴)]
\begin{itemize}
    \item ارزش کل: بیش از ۱.۵ تریلیون دلار
    \item سرانه: حدود ۲۸۰,۰۰۰ دلار به ازای هر شهروند نروژی
    \item سرمایه‌گذاری در: بیش از ۹۰۰۰ شرکت در ۷۰ کشور
    \item مالکیت متوسط: ۱.۵٪ از کل سهام جهان!
\end{itemize}
\end{tcolorbox}

قواعد کلیدی صندوق:
\begin{enumerate}
    \item \textbf{قاعده مالی:} دولت فقط می‌تواند «بازده واقعی مورد 
    انتظار» (حدود ۳٪) را برداشت کند
    \item \textbf{سرمایه‌گذاری خارجی:} برای جلوگیری از بیماری هلندی، 
    ۱۰۰٪ سرمایه‌گذاری خارج از نروژ
    \item \textbf{شفافیت کامل:} لیست تمام سرمایه‌گذاری‌ها آنلاین 
    در دسترس است
    \item \textbf{ملاحظات اخلاقی:} محرومیت شرکت‌های ناقض حقوق بشر، 
    محیط زیست، یا تولیدکننده سلاح
\end{enumerate}

\subsection{کشورهای خلیج فارس: مالکیت خانوادگی-دولتی}

\subsubsection{عربستان سعودی}

\begin{itemize}
    \item نفت متعلق به دولت (عملاً خانواده سلطنتی)
    \item آرامکو: بزرگ‌ترین شرکت نفتی جهان با ارزش بیش از ۲ تریلیون دلار
    \item عرضه اولیه ۱.۵٪ سهام در ۲۰۱۹ (۲۵ میلیارد دلار)
    \item صندوق سرمایه‌گذاری عمومی (PIF): ۷۰۰ میلیارد دلار
\end{itemize}

\subsubsection{کویت}

\begin{itemize}
    \item صندوق نسل‌های آینده: از ۱۹۷۶ (قدیمی‌ترین صندوق ثروت ملی)
    \item سازمان سرمایه‌گذاری کویت (KIA): ۷۵۰ میلیارد دلار
    \item قانون: حداقل ۱۰٪ درآمدهای دولت به صندوق واریز شود
\end{itemize}

\subsubsection{امارات (ابوظبی)}

\begin{itemize}
    \item ADIA: ۸۵۰ میلیارد دلار (یکی از بزرگ‌ترین‌ها)
    \item Mubadala: ۲۸۰ میلیارد دلار (تنوع‌بخشی اقتصادی)
    \item عدم شفافیت نسبی در مقایسه با نروژ
\end{itemize}

\subsubsection{قطر}

\begin{itemize}
    \item QIA: ۴۵۰ میلیارد دلار
    \item سرمایه‌گذاری‌های شاخص: هارودز، فولکس‌واگن، بارکلیز
    \item بالاترین سرانه تولید ناخالص داخلی جهان
\end{itemize}

\begin{table}[htbp]
\centering
\caption{مقایسه صندوق‌های ثروت ملی خاورمیانه}
\label{tab:me-swf-comparison}
\begin{tabular}{|l|r|r|l|c|}
\hline
\textbf{کشور} & \textbf{دارایی (B\$)} & \textbf{جمعیت (M)} & 
    \textbf{سرانه (\$)} & \textbf{شفافیت} \\
\hline
نروژ & ۱,۵۰۰ & ۵.۴ & ۲۷۸,۰۰۰ & بسیار بالا \\
ابوظبی & ۸۵۰ & ۳.۰ & ۲۸۳,۰۰۰ & پایین \\
کویت & ۷۵۰ & ۴.۳ & ۱۷۴,۰۰۰ & متوسط \\
قطر & ۴۵۰ & ۲.۹ & ۱۵۵,۰۰۰ & پایین \\
عربستان & ۷۰۰ & ۳۴ & ۲۱,۰۰۰ & پایین \\
ایران & $<$۵۰ & ۸۵ & $<$۶۰۰ & بسیار پایین \\
\hline
\end{tabular}
\end{table}

\subsection{روسیه: مالکیت دولتی-الیگارشی}

\begin{itemize}
    \item \textbf{قانون اساسی:} منابع طبیعی «متعلق به مردم روسیه»
    \item \textbf{واقعیت:} کنترل توسط شرکت‌های دولتی و الیگارش‌ها
    \item \textbf{گازپروم:} بزرگ‌ترین شرکت گاز جهان، ۵۰٪ دولتی
    \item \textbf{روس‌نفت:} ۴۰٪ دولتی
    \item \textbf{صندوق ثروت ملی:} حدود ۱۵۰ میلیارد دلار (کاهش‌یافته 
    پس از تحریم‌ها)
\end{itemize}

\subsection{چین: مالکیت سوسیالیستی}

\begin{itemize}
    \item تمام منابع طبیعی «متعلق به کل خلق» و تحت مدیریت دولت
    \item سه شرکت بزرگ دولتی: CNPC، Sinopec، CNOOC
    \item CIC (صندوق ثروت ملی): ۱.۲ تریلیون دلار
    \item کنترل شدید دولتی بر استخراج و توزیع
\end{itemize}

\subsection{آمریکای لاتین: تجربیات متنوع}

\subsubsection{شیلی: مدل نسبتاً موفق}

\begin{itemize}
    \item مس متعلق به دولت (Codelco بزرگ‌ترین تولیدکننده مس جهان)
    \item صندوق تثبیت اقتصادی و اجتماعی (ESSF)
    \item صندوق ذخیره بازنشستگی (PRF)
    \item قاعده مالی: مازاد بودجه ساختاری
\end{itemize}

\subsubsection{ونزوئلا: مدل شکست‌خورده}

\begin{itemize}
    \item ملی‌سازی کامل نفت (PDVSA)
    \item عدم وجود صندوق ثروت پایدار
    \item مصرف تمام درآمدها در هزینه‌های جاری
    \item نتیجه: فروپاشی اقتصادی علی‌رغم بزرگ‌ترین ذخایر نفت جهان
\end{itemize}

\subsubsection{بولیوی: ملی‌سازی با توزیع مستقیم}

\begin{itemize}
    \item ملی‌سازی گاز در ۲۰۰۶
    \item برنامه‌های انتقال نقدی مشروط
    \item کاهش فقر از ۶۰٪ به ۳۵٪
\end{itemize}

\subsection{آفریقا: چالش‌ها و فرصت‌ها}

\subsubsection{بوتسوانا: معجزه الماس}

\begin{itemize}
    \item \textbf{استثنای آفریقا:} موفق‌ترین اقتصاد قاره
    \item مالکیت مشترک دولت-De Beers در Debswana
    \item صندوق Pula با مدیریت محتاطانه
    \item سرمایه‌گذاری در آموزش و بهداشت
    \item رشد متوسط ۷٪ در ۵۰ سال
\end{itemize}

\subsubsection{نیجریه: نفرین نفت}

\begin{itemize}
    \item فساد گسترده در صنعت نفت
    \item بی‌ثباتی سیاسی در مناطق نفت‌خیز (دلتای نیجر)
    \item نابرابری شدید منطقه‌ای
    \item صندوق ثروت ملی: عملاً ناکارآمد
\end{itemize}

%═══════════════════════════════════════════════════════════════════════════════
\section{عوامل تعیین‌کننده نظام مالکیت}
\label{sec:determinants}
%═══════════════════════════════════════════════════════════════════════════════

\subsection{طبقه‌بندی بر اساس متغیرهای کلیدی}

\begin{figure}[htbp]
\centering
\begin{tikzpicture}[
    factor/.style={rectangle, draw, fill=blue!20, minimum width=3cm, 
        minimum height=0.8cm, text centered},
    outcome/.style={rectangle, draw, fill=green!20, minimum width=2.5cm, 
        minimum height=0.6cm, text centered, font=\small},
]
    % عوامل
    \node[factor] (history) at (0,4) {سنت حقوقی-تاریخی};
    \node[factor] (ideology) at (0,2.5) {ایدئولوژی اقتصادی};
    \node[factor] (regime) at (0,1) {نوع رژیم سیاسی};
    \node[factor] (timing) at (0,-0.5) {زمان کشف منابع};
    \node[factor] (colonial) at (0,-2) {میراث استعماری};
    
    % نتایج
    \node[outcome] (private) at (6,4) {مالکیت خصوصی};
    \node[outcome] (state) at (6,2.5) {مالکیت دولتی};
    \node[outcome] (hybrid) at (6,1) {مالکیت ترکیبی};
    \node[outcome] (citizen) at (6,-0.5) {مالکیت شهروندی};
    \node[outcome] (communal) at (6,-2) {مالکیت اشتراکی};
    
    % فلش‌ها و ارتباطات
    \draw[arrow, bend left=15] (history) to (private);
    \draw[arrow, bend left=10] (history) to (state);
    \draw[arrow] (ideology) to (state);
    \draw[arrow] (ideology) to (hybrid);
    \draw[arrow, bend right=10] (regime) to (state);
    \draw[arrow] (regime) to (citizen);
    \draw[arrow, bend right=15] (timing) to (hybrid);
    \draw[arrow] (timing) to (state);
    \draw[arrow, bend right=20] (colonial) to (state);
    \draw[arrow] (colonial) to (communal);

    \end{tikzpicture}
    \caption{عوامل تعیین‌کننده نظام مالکیت منابع طبیعی}
    \label{fig:ownership-determinants}
\end{figure}

\subsection{تحلیل عوامل تعیین‌کننده}

\subsubsection{سنت حقوقی-تاریخی}

\begin{table}[htbp]
\centering
\caption{تأثیر سنت حقوقی بر نظام مالکیت منابع}
\label{tab:legal-traditions}
\begin{tabular}{|l|p{5cm}|p{5cm}|}
\hline
\textbf{سنت حقوقی} & \textbf{ویژگی‌ها} & \textbf{مثال‌ها} \\
\hline
کامن‌لا (انگلیسی) & تأکید بر مالکیت خصوصی، حقوق معدنی قابل جداسازی & آمریکا، استرالیا، کانادا \\
\hline
حقوق مدنی (رومی-ژرمنی) & مالکیت دولتی منابع زیرزمینی، تفکیک سطح و عمق & فرانسه، آلمان، آمریکای لاتین \\
\hline
حقوق اسلامی & منابع طبیعی به عنوان «انفال» متعلق به حکومت/امت & ایران، عربستان \\
\hline
حقوق سوسیالیستی & مالکیت عمومی/دولتی همه منابع & چین، کوبا، ویتنام \\
\hline
حقوق عرفی & مالکیت جمعی قبیله‌ای/اشتراکی & بومیان آمریکا، آفریقا، استرالیا \\
\hline
\end{tabular}
\end{table}

\subsubsection{ایدئولوژی اقتصادی}

\begin{figure}[htbp]
\centering
\begin{tikzpicture}
    % طیف ایدئولوژیک
    \draw[thick, ->] (0,0) -- (14,0);
    \node[below] at (0,0) {لیبرالیسم};
    \node[below] at (7,0) {سوسیال‌دموکراسی};
    \node[below] at (14,0) {سوسیالیسم};
    
    % موضع کشورها
    \filldraw[blue] (2,0.5) circle (4pt) node[above, font=\tiny] {آمریکا};
    \filldraw[green] (5,0.5) circle (4pt) node[above, font=\tiny] {استرالیا};
    \filldraw[orange] (8,0.5) circle (4pt) node[above, font=\tiny] {نروژ};
    \filldraw[red] (10,0.5) circle (4pt) node[above, font=\tiny] {ایران};
    \filldraw[purple] (12,0.5) circle (4pt) node[above, font=\tiny] {چین};
    
    % نوع مالکیت غالب
    \draw[decorate, decoration={brace, amplitude=5pt}] (0,-0.8) -- (4,-0.8) 
        node[midway, below=8pt, font=\small] {مالکیت خصوصی};
    \draw[decorate, decoration={brace, amplitude=5pt}] (4,-0.8) -- (10,-0.8) 
        node[midway, below=8pt, font=\small] {مالکیت دولتی + بازتوزیع};
    \draw[decorate, decoration={brace, amplitude=5pt}] (10,-0.8) -- (14,-0.8) 
        node[midway, below=8pt, font=\small] {مالکیت دولتی کامل};
    
\end{tikzpicture}
\caption{طیف ایدئولوژیک و نظام مالکیت منابع طبیعی}
\label{fig:ideology-spectrum}
\end{figure}

\subsubsection{نوع رژیم سیاسی}

\begin{tcolorbox}[colback=blue!5!white,colframe=blue!75!black,
    title=رابطه نوع رژیم و مالکیت منابع]

\textbf{دموکراسی‌های پایدار:}
\begin{itemize}
    \item تمایل به شفافیت در مدیریت منابع
    \item فشار عمومی برای توزیع عادلانه
    \item نهادهای نظارتی قوی‌تر
    \item مثال: نروژ، کانادا، استرالیا
\end{itemize}

\textbf{رژیم‌های اقتدارگرا:}
\begin{itemize}
    \item تمرکز درآمدها در دست نخبگان حاکم
    \item عدم شفافیت و پاسخگویی
    \item استفاده از درآمد منابع برای حفظ قدرت
    \item مثال: عربستان، روسیه، ونزوئلا
\end{itemize}

\textbf{یافته کلیدی:} شکل حقوقی مالکیت کمتر از کیفیت نهادها اهمیت دارد. 
نروژ و عربستان هر دو مالکیت دولتی دارند، اما نتایج کاملاً متفاوت.
\end{tcolorbox}

\subsubsection{زمان کشف منابع}

\begin{table}[htbp]
\centering
\caption{تأثیر زمان کشف منابع بر نظام مالکیت}
\label{tab:timing-effect}
\begin{tabular}{|l|p{5cm}|p{5cm}|}
\hline
\textbf{زمان کشف} & \textbf{ویژگی} & \textbf{مثال} \\
\hline
قبل از دولت مدرن & مالکیت خصوصی/محلی محتمل‌تر & نفت تگزاس (۱۹۰۱)، زغال انگلستان \\
\hline
پس از استقلال & مالکیت دولتی به نام ملت & نفت ایران (ملی‌سازی ۱۹۵۱)، عربستان \\
\hline
پس از موج ملی‌گرایی & ملی‌سازی منابع موجود & مکزیک (۱۹۳۸)، ونزوئلا (۱۹۷۶) \\
\hline
دوره نئولیبرال & خصوصی‌سازی مجدد & بریتانیا، آرژانتین (دهه ۱۹۹۰) \\
\hline
\end{tabular}
\end{table}

\subsubsection{میراث استعماری}

\begin{itemize}
    \item \textbf{مستعمرات بریتانیا:} تمایل به حفظ عناصر کامن‌لا، اما با 
    اصلاحات پس از استقلال (نیجریه، مالزی)
    
    \item \textbf{مستعمرات فرانسه:} پذیرش مدل حقوق مدنی، مالکیت دولتی 
    زیرزمین (الجزایر، گابن)
    
    \item \textbf{مستعمرات اسپانیا:} سنت «Regalian» - منابع متعلق به 
    تاج/دولت (آمریکای لاتین)
    
    \item \textbf{کشورهای غیرمستعمره:} تنوع بیشتر بر اساس عوامل داخلی 
    (ایران، ترکیه، تایلند)
\end{itemize}

%═══════════════════════════════════════════════════════════════════════════════
\section{تحلیل تطبیقی عمیق: چهار مدل اصلی}
\label{sec:four-models}
%═══════════════════════════════════════════════════════════════════════════════

\subsection{مدل اول: مالکیت خصوصی (آمریکا)}

\begin{table}[htbp]
\centering
\caption{جزئیات نظام مالکیت منابع در آمریکا}
\label{tab:usa-details}
\small
\begin{tabular}{|l|p{5cm}|p{5cm}|}
\hline
\textbf{جنبه} & \textbf{ویژگی} & \textbf{پیامد} \\
\hline
\multicolumn{3}{|c|}{\cellcolor{blue!15}\textbf{چارچوب حقوقی}} \\
\hline
اصل مالکیت & «Ad coelum» - مالک زمین، مالک زیرزمین & انگیزه اکتشاف خصوصی \\
\hline
قابلیت جداسازی & حقوق سطحی و معدنی جداشدنی & پیچیدگی حقوقی \\
\hline
قاعده تصرف & «Rule of Capture» در برخی ایالات & رقابت برای استخراج سریع \\
\hline
\multicolumn{3}{|c|}{\cellcolor{green!15}\textbf{توزیع مالکیت زمین}} \\
\hline
زمین خصوصی & حدود ۶۰٪ & مالکیت خصوصی منابع \\
\hline
زمین ایالتی & حدود ۱۲٪ & مالکیت ایالت \\
\hline
زمین فدرال & حدود ۲۸٪ & امتیازدهی و اجاره \\
\hline
\multicolumn{3}{|c|}{\cellcolor{orange!15}\textbf{نظام مالی}} \\
\hline
حق امتیاز (Royalty) & ۱۲.۵-۲۵٪ از تولید & درآمد برای مالک \\
\hline
مالیات بر تولید & Severance Tax ایالتی & درآمد ایالت \\
\hline
مالیات بر درآمد & مالیات فدرال و ایالتی & درآمد عمومی \\
\hline
\end{tabular}
\end{table}

\begin{tcolorbox}[colback=orange!5!white,colframe=orange!75!black,
    title=مطالعه موردی: انقلاب شیل آمریکا]

\textbf{زمینه:} انقلاب نفت و گاز شیل (۲۰۰۸-کنون) تا حد زیادی مدیون 
نظام مالکیت خصوصی آمریکاست.

\textbf{چرا در آمریکا ممکن شد:}
\begin{enumerate}
    \item مالکان خصوصی انگیزه اجازه حفاری داشتند (Royalty)
    \item شرکت‌های کوچک می‌توانستند با مالکان قرارداد ببندند
    \item رقابت و نوآوری در تکنولوژی
    \item عدم نیاز به مجوزهای پیچیده دولتی
\end{enumerate}

\textbf{چرا در اروپا تکرار نشد:}
\begin{enumerate}
    \item مالکیت دولتی زیرزمین: مالکان سطحی سودی نمی‌برند
    \item مخالفت محلی بدون منفعت مالی
    \item بوروکراسی مجوزدهی دولتی
\end{enumerate}

\textbf{درس:} نظام مالکیت می‌تواند بر نوآوری و توسعه منابع تأثیر 
چشمگیر داشته باشد.
\end{tcolorbox}

\subsection{مدل دوم: مالکیت دولتی + شفافیت (نروژ)}

\begin{table}[htbp]
\centering
\caption{ارکان موفقیت مدل نروژ}
\label{tab:norway-pillars}
\begin{tabular}{|c|p{5cm}|p{5cm}|}
\hline
\textbf{رکن} & \textbf{شرح} & \textbf{سازوکار} \\
\hline
۱ & \textbf{مالکیت دولتی کامل} & همه منابع فلات قاره متعلق به دولت \\
\hline
۲ & \textbf{جداسازی نقش‌ها} & سیاست‌گذاری (وزارت) ≠ عملیات (Equinor) ≠ سرمایه‌گذاری (GPFG) \\
\hline
۳ & \textbf{مالیات بالا} & ۷۸٪ مالیات بر سود نفتی \\
\hline
۴ & \textbf{صندوق ثروت ملی} & ذخیره درآمدها برای آینده \\
\hline
۵ & \textbf{قاعده مالی} & برداشت حداکثر ۳٪ سود صندوق \\
\hline
۶ & \textbf{شفافیت کامل} & گزارش‌دهی عمومی همه جزئیات \\
\hline
۷ & \textbf{سرمایه‌گذاری خارجی} & جلوگیری از بیماری هلندی \\
\hline
۸ & \textbf{نظارت دموکراتیک} & کنترل پارلمانی قوی \\
\hline
\end{tabular}
\end{table}

\begin{figure}[htbp]
\centering
\begin{tikzpicture}[
    box/.style={rectangle, draw, fill=#1, minimum width=3.5cm, 
        minimum height=1cm, text centered, font=\small, rounded corners},
    arrow/.style={->, thick, >=stealth}
]

% جریان درآمد نروژ
\node[box=blue!30] (oil) at (0,4) {درآمد نفت و گاز\\(حدود ۵۰ B\$/سال)};
\node[box=orange!30] (tax) at (-4,2) {مالیات ۷۸٪\\(به خزانه)};
\node[box=green!30] (sdfi) at (0,2) {SDFI\\(سهم مستقیم دولت)};
\node[box=purple!30] (dividend) at (4,2) {سود Equinor\\(۶۷٪ دولتی)};
\node[box=yellow!40] (gpfg) at (0,0) {صندوق GPFG\\(۱.۵ تریلیون دلار)};
\node[box=red!30] (budget) at (0,-2) {بودجه دولت\\(حداکثر ۳٪ سود صندوق)};

% فلش‌ها
\draw[arrow] (oil) -- (tax);
\draw[arrow] (oil) -- (sdfi);
\draw[arrow] (oil) -- (dividend);
\draw[arrow] (tax) -- (gpfg);
\draw[arrow] (sdfi) -- (gpfg);
\draw[arrow] (dividend) -- (gpfg);
\draw[arrow] (gpfg) -- (budget);

\end{tikzpicture}
\caption{جریان درآمد نفتی در مدل نروژ}
\label{fig:norway-flow}
\end{figure}

\subsection{مدل سوم: مالکیت دولتی + عدم شفافیت (خاورمیانه)}

\begin{table}[htbp]
\centering
\caption{مقایسه کشورهای نفتی خاورمیانه}
\label{tab:middle-east-comparison}
\small
\begin{tabular}{|l|c|c|c|c|c|}
\hline
\textbf{کشور} & \textbf{مالکیت} & \textbf{شرکت ملی} & \textbf{صندوق ثروت} & \textbf{شفافیت} & \textbf{توزیع مستقیم} \\
\hline
عربستان & سلطنتی/دولتی & Aramco & PIF & پایین & خیر \\
\hline
امارات & امیری/دولتی & ADNOC & ADIA, Mubadala & پایین & خیر (شهروندان ویژه) \\
\hline
کویت & دولتی & KPC & KIA & متوسط & خیر \\
\hline
قطر & امیری/دولتی & QP & QIA & پایین & خیر (رفاه بالا) \\
\hline
عراق & دولتی & --- & --- & بسیار پایین & خیر \\
\hline
ایران & دولتی & NIOC & NDF (ضعیف) & بسیار پایین & یارانه غیرمستقیم \\
\hline
\end{tabular}
\end{table}

\begin{tcolorbox}[colback=red!5!white,colframe=red!75!black,
    title=چرا نتایج متفاوت است؟]

\textbf{همه این کشورها مالکیت دولتی دارند، اما نتایج بسیار متفاوت:}

\textbf{کویت و امارات (نسبتاً موفق):}
\begin{itemize}
    \item جمعیت کم، درآمد سرانه بالا
    \item صندوق‌های ثروت قدیمی و بزرگ
    \item تنوع‌بخشی اقتصادی (امارات)
\end{itemize}

\textbf{عربستان (در حال تغییر):}
\begin{itemize}
    \item جمعیت بزرگ‌تر، فشار بودجه‌ای
    \item Vision 2030 برای تنوع‌بخشی
    \item IPO آرامکو برای جذب سرمایه
\end{itemize}

\textbf{ایران و عراق (ناموفق):}
\begin{itemize}
    \item عدم شفافیت شدید
    \item فساد سیستماتیک
    \item مصرف درآمدها به جای سرمایه‌گذاری
    \item بی‌ثباتی سیاسی
\end{itemize}

\textbf{نتیجه‌گیری:} مالکیت دولتی شرط لازم موفقیت نیست، نهادهای حاکم 
بر آن تعیین‌کننده است.
\end{tcolorbox}

\subsection{مدل چهارم: مالکیت شهروندی + توزیع مستقیم (آلاسکا)}

\begin{table}[htbp]
\centering
\caption{ساختار صندوق دائمی آلاسکا}
\label{tab:alaska-structure}
\begin{tabular}{|l|p{8cm}|}
\hline
\textbf{جنبه} & \textbf{شرح} \\
\hline
تأسیس & ۱۹۷۶ (اصلاحیه قانون اساسی ایالتی) \\
\hline
منبع & حداقل ۲۵٪ درآمدهای نفتی ایالت \\
\hline
ارزش فعلی & حدود ۷۵ میلیارد دلار \\
\hline
سرانه & حدود ۱۰۰,۰۰۰ دلار به ازای هر ساکن \\
\hline
سرمایه‌گذاری & متنوع جهانی (سهام، اوراق، املاک) \\
\hline
توزیع سود & سالانه به همه ساکنان واجد شرایط \\
\hline
شرط دریافت & اقامت حداقل یک سال در آلاسکا \\
\hline
مبلغ سود & ۱,۰۰۰ تا ۳,۲۸۴ دلار (بسته به عملکرد) \\
\hline
\end{tabular}
\end{table}

\begin{figure}[htbp]
\centering
\begin{tikzpicture}
\begin{axis}[
    xlabel={سال},
    ylabel={دلار},
    xmin=1982, xmax=2024,
    ymin=0, ymax=3500,
    xtick={1985,1990,1995,2000,2005,2010,2015,2020,2024},
    legend pos=north west,
    ymajorgrids=true,
    grid style=dashed,
    width=14cm,
    height=7cm,
]

\addplot[thick, color=blue, mark=*, mark size=1.5pt] coordinates {
    (1982,1000) (1983,386) (1984,331) (1985,404) (1986,556)
    (1987,708) (1988,827) (1989,873) (1990,952) (1991,931)
    (1992,915) (1993,949) (1994,984) (1995,990) (1996,1131)
    (1997,1297) (1998,1541) (1999,1770) (2000,1964) (2001,1850)
    (2002,1541) (2003,1107) (2004,920) (2005,846) (2006,1107)
    (2007,1654) (2008,2069) (2009,1305) (2010,1281) (2011,1174)
    (2012,878) (2013,900) (2014,1884) (2015,2072) (2016,1022)
    (2017,1100) (2018,1600) (2019,1606) (2020,992) (2021,1114)
    (2022,3284) (2023,1312)
};
\addlegendentry{سود سالانه (PFD)}

\end{axis}
\end{tikzpicture}
\caption{روند سود سالانه صندوق دائمی آلاسکا (۱۹۸۲-۲۰۲۳)}
\label{fig:alaska-pfd-history}
\end{figure}

\begin{tcolorbox}[colback=green!5!white,colframe=green!75!black,
    title=مزایا و معایب مدل آلاسکا]

\textbf{مزایا:}
\begin{enumerate}
    \item \textbf{مالکیت واقعی شهروندی:} هر ساکن مستقیماً از منابع بهره می‌برد
    \item \textbf{کاهش نابرابری:} پرداخت یکسان به همه
    \item \textbf{محبوبیت سیاسی:} حمایت گسترده مردمی از صندوق
    \item \textbf{شفافیت:} گزارش‌دهی عمومی کامل
    \item \textbf{ثبات:} حفاظت قانون اساسی از صندوق
\end{enumerate}

\textbf{معایب:}
\begin{enumerate}
    \item \textbf{مقیاس کوچک:} جمعیت ۷۳۰,۰۰۰ نفری
    \item \textbf{وابستگی:} برخی ساکنان به PFD وابسته شده‌اند
    \item \textbf{فشار سیاسی:} وسوسه برداشت بیشتر
    \item \textbf{نوسان:} تغییرات سالانه در مبلغ
\end{enumerate}
\end{tcolorbox}

%═══════════════════════════════════════════════════════════════════════════════
\section{منابع طبیعی خاص: آب، جنگل، معادن}
\label{sec:specific-resources}
%═══════════════════════════════════════════════════════════════════════════════

\subsection{آب: طیف مالکیت}

\begin{table}[htbp]
\centering
\caption{نظام‌های مالکیت آب در جهان}
\label{tab:water-ownership}
\small
\begin{tabular}{|l|p{4cm}|p{3cm}|p{4cm}|}
\hline
\textbf{نظام} & \textbf{اصل} & \textbf{کشورها} & \textbf{ویژگی} \\
\hline
\multicolumn{4}{|c|}{\cellcolor{blue!15}\textbf{آب سطحی}} \\
\hline
حقوق ساحلی (Riparian) & مالکان کنار رودخانه حق برداشت دارند & شرق آمریکا، انگلستان & مناطق پرآب \\
\hline
تخصیص تاریخی (Prior Appropriation) & «هر که زودتر، حق بیشتر» & غرب آمریکا & مناطق کم‌آب \\
\hline
مالکیت دولتی & دولت تخصیص می‌دهد & اکثر کشورها & کنترل مرکزی \\
\hline
\multicolumn{4}{|c|}{\cellcolor{green!15}\textbf{آب زیرزمینی}} \\
\hline
قاعده تصرف مطلق & مالک زمین = مالک آب زیرزمین & تگزاس & استخراج بی‌رویه \\
\hline
استفاده معقول & محدودیت به «استفاده معقول» & اکثر ایالات آمریکا & تعادل نسبی \\
\hline
مجوز دولتی & نیاز به مجوز برای حفر چاه & ایران، اروپا & کنترل (اغلب ناموفق) \\
\hline
\end{tabular}
\end{table}

\subsection{جنگل‌ها}

\begin{table}[htbp]
\centering
\caption{مالکیت جنگل در کشورهای منتخب}
\label{tab:forest-ownership}
\begin{tabular}{|l|c|c|c|c|}
\hline
\textbf{کشور} & \textbf{دولتی} & \textbf{خصوصی} & \textbf{اشتراکی} & \textbf{مساحت (M ha)} \\
\hline
روسیه & ۱۰۰٪ & ۰٪ & ۰٪ & ۸۱۵ \\
\hline
کانادا & ۹۲٪ & ۶٪ & ۲٪ & ۳۴۷ \\
\hline
برزیل & ۸۰٪ & ۲۰٪ & --- & ۴۹۷ \\
\hline
آمریکا & ۳۳٪ & ۵۸٪ & ۹٪ & ۳۱۰ \\
\hline
سوئد & ۱۷٪ & ۸۰٪ & ۳٪ & ۲۸ \\
\hline
ایران & ۱۰۰٪ & ۰٪ & ۰٪ & ۱۱ \\
\hline
\end{tabular}
\end{table}

\subsection{معادن غیرنفتی}

\begin{table}[htbp]
\centering
\caption{نظام مالکیت معادن در کشورهای منتخب}
\label{tab:mining-ownership}
\begin{tabular}{|l|l|p{6cm}|}
\hline
\textbf{کشور} & \textbf{مالکیت} & \textbf{توضیح} \\
\hline
آمریکا & خصوصی/دولتی & در زمین خصوصی: مالک؛ در زمین دولتی: امتیاز \\
\hline
کانادا & ایالتی & مالکیت ایالت‌ها، امتیازدهی \\
\hline
استرالیا & ایالتی & Crown ownership، حق امتیاز \\
\hline
شیلی & دولتی & مس متعلق به دولت (Codelco) \\
\hline
کنگو & دولتی & اما امتیازات گسترده به شرکت‌های خارجی \\
\hline
آفریقای جنوبی & دولتی (از ۲۰۰۲) & قبلاً خصوصی، ملی‌سازی تدریجی \\
\hline
ایران & دولتی & قانون معادن: همه معادن متعلق به دولت \\
\hline
\end{tabular}
\end{table}

%═══════════════════════════════════════════════════════════════════════════════
\section{مدل‌های توزیع درآمد منابع طبیعی}
\label{sec:distribution-models}
%═══════════════════════════════════════════════════════════════════════════════

\begin{figure}[htbp]
\centering
\begin{tikzpicture}[
    model/.style={rectangle, draw, fill=#1, minimum width=4.5cm, 
        minimum height=2cm, text centered, font=\small, rounded corners}
]

% چهار مدل
\node[model=blue!30] (m1) at (-5,3) {\textbf{مدل ۱: مصرف دولتی}\\درآمد → بودجه جاری\\مثال: ونزوئلا، ایران};

\node[model=green!30] (m2) at (5,3) {\textbf{مدل ۲: سرمایه‌گذاری دولتی}\\درآمد → صندوق ثروت\\مثال: نروژ، کویت};

\node[model=orange!30] (m3) at (-5,-2) {\textbf{مدل ۳: توزیع مستقیم}\\درآمد → شهروندان\\مثال: آلاسکا};

\node[model=purple!30] (m4) at (5,-2) {\textbf{مدل ۴: ترکیبی}\\درآمد → صندوق + بودجه + توزیع\\مثال: پیشنهادی};

% رتبه‌بندی
\node[font=\tiny, red] at (-5,1.5) {نتیجه: فقر نسل آینده};
\node[font=\tiny, green!60!black] at (5,1.5) {نتیجه: ثروت بین‌نسلی};
\node[font=\tiny, blue] at (-5,-3.5) {نتیجه: عدالت کوتاه‌مدت};
\node[font=\tiny, purple] at (5,-3.5) {نتیجه: توازن و پایداری};

\end{tikzpicture}
\caption{چهار مدل اصلی توزیع درآمد منابع طبیعی}
\label{fig:distribution-models}
\end{figure}

\begin{table}[htbp]
\centering
\caption{مقایسه مدل‌های توزیع درآمد منابع}
\label{tab:distribution-models-comparison}
\small
\begin{tabular}{|l|c|c|c|c|c|}
\hline
\textbf{معیار} & \textbf{مصرف دولتی} & \textbf{صندوق ثروت} & \textbf{توزیع مستقیم} & \textbf{ترکیبی} \\
\hline
عدالت بین‌نسلی & ● & ●●●●● & ●●● & ●●●● \\
\hline
عدالت درون‌نسلی & ●● & ●● & ●●●●● & ●●●● \\
\hline
کارآمدی اقتصادی & ●● & ●●●● & ●●● & ●●●● \\
\hline
ریسک فساد & ●●●●● & ●● & ● & ●● \\
\hline
پایداری سیاسی & ●● & ●●● & ●●●●● & ●●●● \\
\hline
سادگی اجرا & ●●●●● & ●●● & ●●●● & ●● \\
\hline
\end{tabular}
\end{table}

%═══════════════════════════════════════════════════════════════════════════════
\section{درس‌ها برای ایران}
\label{sec:lessons-iran}
%═══════════════════════════════════════════════════════════════════════════════

\subsection{وضعیت فعلی ایران}

\begin{table}[htbp]
\centering
\caption{وضعیت فعلی مالکیت و مدیریت منابع در ایران}
\label{tab:iran-current}
\begin{tabular}{|l|p{4cm}|p{6cm}|}
\hline
\textbf{منبع} & \textbf{نظام مالکیت} & \textbf{مشکلات} \\
\hline
نفت و گاز & دولتی (انفال) & عدم شفافیت، مصرف در بودجه جاری، فساد \\
\hline
معادن & دولتی & امتیازدهی مبهم، رانت‌جویی \\
\hline
آب & دولتی (نظری) & برداشت بی‌رویه، عدم قیمت‌گذاری واقعی \\
\hline
جنگل & دولتی & تخریب گسترده، ضعف نظارت \\
\hline
زمین شهری & دولتی/خصوصی & سوداگری، عدم شفافیت معاملات \\
\hline
فضای فرکانس & دولتی & انحصار، ناکارآمدی \\
\hline
\end{tabular}
\end{table}

\subsection{اصول پیشنهادی برای اصلاح}

\begin{tcolorbox}[colback=green!5!white,colframe=green!75!black,
    title=اصول پیشنهادی برای نظام مالکیت منابع در ایران]

\textbf{اصل ۱: شفافیت کامل}
\begin{itemize}
    \item انتشار عمومی همه قراردادهای منابع طبیعی
    \item گزارش‌دهی منظم درآمدها و مصارف
    \item عضویت در EITI (ابتکار شفافیت صنایع استخراجی)
\end{itemize}

\textbf{اصل ۲: جداسازی نقش‌ها}
\begin{itemize}
    \item وزارتخانه: سیاست‌گذاری و نظارت
    \item شرکت ملی: عملیات (با حاکمیت شرکتی مستقل)
    \item صندوق ثروت: سرمایه‌گذاری (مستقل از دولت)
\end{itemize}

\textbf{اصل ۳: عدالت بین‌نسلی}
\begin{itemize}
    \item حداقل ۵۰٪ درآمدها به صندوق ثروت ملی
    \item محدودیت برداشت از صندوق (قاعده مالی)
    \item سرمایه‌گذاری بلندمدت
\end{itemize}

\textbf{اصل ۴: مشارکت شهروندان}
\begin{itemize}
    \item توزیع بخشی از سود صندوق به صورت مستقیم
    \item ایجاد احساس مالکیت عمومی
    \item کاهش وابستگی به دولت
\end{itemize}

\textbf{اصل ۵: حفاظت از محیط زیست}
\begin{itemize}
    \item استانداردهای زیست‌محیطی الزام‌آور
    \item صندوق جبران خسارات زیست‌محیطی
    \item حق اعتراض جوامع محلی
\end{itemize}
\end{tcolorbox}

\subsection{مقایسه وضعیت فعلی و مطلوب}

\begin{table}[htbp]
\centering
\caption{مقایسه وضعیت فعلی و مطلوب مدیریت منابع در ایران}
\label{tab:iran-current-vs-proposed}
\begin{tabular}{|l|p{5cm}|p{5cm}|}
\hline
\textbf{معیار} & \textbf{وضعیت فعلی} & \textbf{وضعیت مطلوب} \\
\hline
شفافیت & بسیار پایین & شفافیت کامل (EITI+) \\
\hline
صندوق ثروت & ضعیف و غیرمستقل & قوی و مستقل \\
\hline
سهم صندوق & ۲۰٪ (نظری) & ۵۰٪+ \\
\hline
قاعده برداشت & بدون محدودیت مؤثر & حداکثر ۴٪ سود \\
\hline
توزیع مستقیم & ندارد & سود سالانه شهروندی \\
\hline
نظارت & ضعیف و غیرمستقل & پارلمانی + عمومی \\
\hline
استقلال شرکت ملی & پایین & حاکمیت شرکتی مستقل \\
\hline
سرمایه‌گذاری صندوق & عمدتاً داخلی و تسهیلات & متنوع جهانی \\
\hline
\end{tabular}
\end{table}

%═══════════════════════════════════════════════════════════════════════════════
\section{جمع‌بندی: نه مالکیت، بلکه حکمرانی}
\label{sec:ownership-conclusion}
%═══════════════════════════════════════════════════════════════════════════════

\begin{tcolorbox}[colback=blue!5!white,colframe=blue!75!black,
    title=یافته‌های کلیدی فصل]

\textbf{۱. تنوع گسترده جهانی:}
\begin{itemize}
    \item هیچ مدل جهان‌شمول «بهترین» وجود ندارد
    \item از مالکیت خصوصی کامل (تگزاس) تا دولتی کامل (نروژ)
    \item آمریکا استثنای جهانی در مالکیت خصوصی زیرزمین است
\end{itemize}

\textbf{۲. مهم‌تر از مالکیت: کیفیت نهادها}
\begin{itemize}
    \item نروژ و ونزوئلا هر دو مالکیت دولتی دارند: نتایج متضاد
    \item شفافیت، پاسخگویی، حاکمیت قانون تعیین‌کننده‌اند
    \item دموکراسی عامل کلیدی در مدیریت موفق منابع است
\end{itemize}

\textbf{۳. توزیع مستقیم: ایده‌ای در حال گسترش}
\begin{itemize}
    \item آلاسکا الگوی موفق ۴۰ ساله
    \item منافع: کاهش فساد، عدالت، پاسخگویی
    \item قابل ترکیب با صندوق ثروت ملی
\end{itemize}

\textbf{۴. ایران: نیاز به تحول بنیادین}
\begin{itemize}
    \item مالکیت دولتی بدون شفافیت و پاسخگویی = فساد
    \item نیاز به صندوق ثروت مستقل و قوی
    \item توزیع مستقیم می‌تواند رابطه دولت-ملت را دگرگون کند
\end{itemize}
\end{tcolorbox}

\begin{figure}[htbp]
\centering
\begin{tikzpicture}
    % محورها
    \draw[thick, ->] (0,0) -- (12,0) node[right] {کیفیت نهادها و حکمرانی};
    \draw[thick, ->] (0,0) -- (0,8) node[above] {رفاه از منابع طبیعی};
    
    % کشورها
    \filldraw[green!60!black] (10,7) circle (5pt) node[above right, font=\small] {نروژ};
    \filldraw[blue] (8,6) circle (5pt) node[above, font=\small] {کانادا};
    \filldraw[orange] (6,4.5) circle (5pt) node[above, font=\small] {شیلی};
    \filldraw[purple] (5,5.5) circle (5pt) node[above, font=\small] {امارات};
    \filldraw[cyan] (4,4) circle (5pt) node[below, font=\small] {مالزی};
    \filldraw[yellow!80!black] (3,2.5) circle (5pt) node[below, font=\small] {عربستان};
    \filldraw[red] (2,1.5) circle (5pt) node[below, font=\small] {ایران};
    \filldraw[red!60!black] (1.5,0.8) circle (5pt) node[below, font=\small] {ونزوئلا};
    \filldraw[gray] (1,1) circle (5pt) node[below left, font=\small] {نیجریه};
    
    % خط روند
    \draw[dashed, thick, gray] (0.5,0.5) -- (11,7.5);
    
    % توضیح
    \node[font=\small, text width=4cm, align=center] at (9,2) {همبستگی قوی بین\\کیفیت نهادها و\\رفاه از منابع};
    
\end{tikzpicture}
\caption{رابطه کیفیت نهادها و رفاه حاصل از منابع طبیعی}
\label{fig:institutions-welfare}
\end{figure}

\begin{tcolorbox}[colback=green!5!white,colframe=green!75!black,
    title=پیام نهایی]

\begin{center}
\textbf{«مهم نیست منابع طبیعی متعلق به کیست،\\
مهم این است که چگونه مدیریت و توزیع می‌شوند.»}
\end{center}

\vspace{0.5cm}

منابع طبیعی می‌توانند برکت باشند یا نفرین. تفاوت در:
\begin{itemize}
    \item \textbf{شفافیت:} مردم باید بدانند درآمدها کجا می‌رود
    \item \textbf{پاسخگویی:} مسئولان باید جوابگو باشند
    \item \textbf{عدالت:} همه نسل‌ها باید بهره‌مند شوند
    \item \textbf{پایداری:} منابع نباید پیش‌خور شوند
\end{itemize}

ایران با داشتن منابع عظیم نفت، گاز و معادن، می‌تواند به یکی از 
ثروتمندترین کشورهای جهان تبدیل شود - اگر نهادهای درست بسازد.
\end{tcolorbox}

%═══════════════════════════════════════════════════════════════════════════════
% منابع
%═══════════════════════════════════════════════════════════════════════════════
\section*{منابع فصل}
\addcontentsline{toc}{section}{منابع فصل}

\begin{thebibliography}{99}

\bibitem{ross2012}
Ross, M. L. (2012).
\textit{The Oil Curse: How Petroleum Wealth Shapes the Development of Nations}.
Princeton University Press.

\bibitem{wenar2016}
Wenar, L. (2016).
\textit{Blood Oil: Tyrants, Violence, and the Rules that Run the World}.
Oxford University Press.

\bibitem{humphreys2007}
Humphreys, M., Sachs, J. D., \& Stiglitz, J. E. (Eds.). (2007).
\textit{Escaping the Resource Curse}.
Columbia University Press.

\bibitem{karl1997}
Karl, T. L. (1997).
\textit{The Paradox of Plenty: Oil Booms and Petro-States}.
University of California Press.

\bibitem{collier2010}
Collier, P. (2010).
\textit{The Plundered Planet: Why We Must—and How We Can—Manage Nature for Global Prosperity}.
Oxford University Press.

\bibitem{anderson2018}
Anderson, O. L. (2018).
\textit{Oil and Gas Law and Taxation}.
LexisNexis.

\bibitem{smith2018}
Smith, E. E., et al. (2018).
\textit{International Petroleum Transactions}.
Rocky Mountain Mineral Law Foundation.

\bibitem{gpfg2023}
Norges Bank Investment Management. (2023).
\textit{Government Pension Fund Global: Annual Report 2023}.
Oslo: NBIM.

\bibitem{apfc2023}
Alaska Permanent Fund Corporation. (2023).
\textit{Annual Report 2023}.
Juneau: APFC.

\bibitem{eiti2023}
Extractive Industries Transparency Initiative. (2023).
\textit{EITI Standard 2023}.
Oslo: EITI International Secretariat.

\bibitem{nrgi2021}
Natural Resource Governance Institute. (2021).
\textit{Resource Governance Index 2021}.
New York: NRGI.

\bibitem{weo2023}
International Energy Agency. (2023).
\textit{World Energy Outlook 2023}.
Paris: IEA.

\end{thebibliography}

%═══════════════════════════════════════════════════════════════════════════════
% پیوست
%═══════════════════════════════════════════════════════════════════════════════
\appendix
\section{پیوست: جدول جامع نظام‌های مالکیت منابع در ۵۰ کشور}
\label{app:ownership-table}

\begin{longtable}{|l|l|l|l|c|}
\caption{نظام مالکیت منابع طبیعی در کشورهای منتخب} \\
\hline
\textbf{کشور} & \textbf{نفت و گاز} & \textbf{معادن} & \textbf{آب} & \textbf{شاخص RGI} \\
\hline
\endfirsthead
\hline
\textbf{کشور} & \textbf{نفت و گاز} & \textbf{معادن} & \textbf{آب} & \textbf{شاخص RGI} \\
\hline
\endhead
\hline
\endfoot

\multicolumn{5}{|c|}{\cellcolor{green!15}\textbf{کشورهای با حکمرانی خوب}} \\
\hline
نروژ & دولتی & دولتی & دولتی & ۸۶ \\
\hline
کانادا & ایالتی & ایالتی/خصوصی & ایالتی & ۷۵ \\
\hline
استرالیا & ایالتی & ایالتی & ایالتی & ۷۱ \\
\hline
آمریکا & خصوصی/فدرال & خصوصی/فدرال & ایالتی & ۷۳ \\
\hline
بریتانیا & دولتی & خصوصی & دولتی & ۷۸ \\
\hline
شیلی & دولتی & دولتی (Codelco) & دولتی & ۶۹ \\
\hline
بوتسوانا & دولتی & مشترک & دولتی & ۶۵ \\
\hline

\multicolumn{5}{|c|}{\cellcolor{yellow!15}\textbf{کشورهای با حکمرانی متوسط}} \\
\hline
مکزیک & دولتی & دولتی/امتیاز & دولتی & ۵۵ \\
\hline
برزیل & دولتی & مختلط & دولتی & ۵۸ \\
\hline
هند & دولتی & دولتی & ایالتی & ۵۲ \\
\hline
اندونزی & دولتی & دولتی & دولتی & ۵۳ \\
\hline
مالزی & دولتی & ایالتی & ایالتی & ۵۶ \\
\hline
کلمبیا & دولتی & دولتی & دولتی & ۵۴ \\
\hline
غنا & دولتی & دولتی & دولتی & ۵۱ \\
\hline

\multicolumn{5}{|c|}{\cellcolor{orange!15}\textbf{کشورهای خاورمیانه و شمال آفریقا}} \\
\hline
عربستان & سلطنتی & دولتی & دولتی & ۴۱ \\
\hline
امارات & امیری & دولتی & دولتی & ۳۹ \\
\hline
کویت & دولتی & دولتی & دولتی & ۴۵ \\
\hline
قطر & امیری & دولتی & دولتی & ۳۸ \\
\hline
عراق & دولتی & دولتی & دولتی & ۳۱ \\
\hline
ایران & دولتی (انفال) & دولتی & دولتی & ۲۹ \\
\hline
الجزایر & دولتی & دولتی & دولتی & ۳۵ \\
\hline
لیبی & دولتی & دولتی & دولتی & ۲۲ \\
\hline

\multicolumn{5}{|c|}{\cellcolor{red!15}\textbf{کشورهای با حکمرانی ضعیف}} \\
\hline
روسیه & دولتی/الیگارشی & دولتی & دولتی & ۳۵ \\
\hline
ونزوئلا & دولتی & دولتی & دولتی & ۲۸ \\
\hline
نیجریه & دولتی & دولتی & دولتی & ۳۰ \\
\hline
آنگولا & دولتی & دولتی & دولتی & ۲۵ \\
\hline
کنگو (DRC) & دولتی & دولتی (امتیاز) & دولتی & ۲۰ \\
\hline
ترکمنستان & دولتی & دولتی & دولتی & ۱۵ \\
\hline

\end{longtable}