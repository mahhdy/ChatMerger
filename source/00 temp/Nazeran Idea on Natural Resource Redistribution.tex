\chapter{بازاندیشی در مدیریت درآمدهای نفتی: از صندوق توسعه ملی تا صندوق ثروت ملی}
\label{chap:sovereign-wealth-fund}

\begin{abstract}
این فصل به بررسی انتقادی ساختار فعلی صندوق توسعه ملی ایران و ارائه نقشه راه 
برای تبدیل آن به یک صندوق ثروت ملی واقعی می‌پردازد. با بهره‌گیری از 
دیدگاه‌های دکتر پویا ناظران و تجربیات موفق جهانی، چارچوبی برای توزیع عادلانه 
و پایدار درآمدهای حاصل از منابع طبیعی ارائه می‌شود.
\end{abstract}

%═══════════════════════════════════════════════════════════════════════════════
\section{مقدمه: نفرین منابع یا برکت منابع؟}
\label{sec:resource-curse}
%═══════════════════════════════════════════════════════════════════════════════

کشورهای دارای منابع طبیعی فراوان اغلب با پدیده‌ای موسوم به «نفرین منابع» 
\LTRfootnote{Resource Curse} مواجه می‌شوند. این پدیده شامل:

\begin{itemize}
    \item بیماری هلندی و تضعیف بخش‌های مولد اقتصاد
    \item نوسانات شدید درآمدی و بی‌ثباتی اقتصاد کلان
    \item رانت‌جویی و فساد ناشی از تمرکز درآمدها در دست دولت
    \item ناپایداری مالی بین‌نسلی
\end{itemize}

تجربه ایران در چهار دهه گذشته نشان می‌دهد که علی‌رغم درآمدهای عظیم نفتی، 
ثروت ملی نه‌تنها افزایش نیافته، بلکه با پیش‌خور کردن منابع نسل‌های آینده، 
کاهش یافته است.

\begin{tcolorbox}[colback=red!5!white,colframe=red!75!black,title=آمار تکان‌دهنده]
بر اساس برآوردها، ایران در دوره ۱۳۸۴-۱۴۰۰ بیش از ۱۰۰۰ میلیارد دلار درآمد 
نفتی داشته است، اما صندوق توسعه ملی در بهترین حالت موجودی‌ای کمتر از ۱۰۰ 
میلیارد دلار داشت که بخش عمده آن نیز به صورت تسهیلات غیرقابل بازیافت 
خارج شده است.
\end{tcolorbox}

%═══════════════════════════════════════════════════════════════════════════════
\section{نقد ساختار فعلی صندوق توسعه ملی}
\label{sec:ndf-critique}
%═══════════════════════════════════════════════════════════════════════════════

\subsection{مشکلات بنیادین}

دکتر پویا ناظران، اقتصاددان ایرانی، در تحلیل‌های متعدد به نقد ساختاری 
صندوق توسعه ملی پرداخته است. مشکلات اصلی عبارتند از:

\begin{enumerate}
    \item \textbf{فقدان استقلال واقعی:} صندوق عملاً به منبع تأمین مالی 
    کسری بودجه دولت تبدیل شده است.
    
    \item \textbf{عدم شفافیت:} گزارش‌دهی منظم و قابل اتکا از عملکرد 
    سرمایه‌گذاری‌ها وجود ندارد.
    
    \item \textbf{تسهیلات‌دهی به جای سرمایه‌گذاری:} صندوق به جای 
    سرمایه‌گذاری در دارایی‌های مالی جهانی، تسهیلات ارزان‌قیمت به 
    پروژه‌های داخلی (اغلب ناکارآمد) پرداخت می‌کند.
    
    \item \textbf{نبود سازوکار توزیع عواید:} هیچ مکانیزم مشخصی برای 
    بهره‌مندی مستقیم شهروندان از عواید صندوق وجود ندارد.
    
    \item \textbf{پیش‌خور کردن منابع:} استفاده از درآمدهای نفتی برای 
    هزینه‌های جاری به جای تبدیل به دارایی‌های پایدار.
\end{enumerate}

\begin{figure}[htbp]
\centering
\begin{tikzpicture}[node distance=2cm, auto]
    % Nodes
    \node[draw, rectangle, fill=blue!20, minimum width=3cm, minimum height=1cm] 
        (oil) {درآمد نفتی};
    \node[draw, rectangle, fill=orange!20, minimum width=3cm, minimum height=1cm, 
        below=of oil] (ndf) {صندوق توسعه ملی};
    \node[draw, rectangle, fill=red!20, minimum width=2.5cm, minimum height=0.8cm, 
        below left=1.5cm and 0.5cm of ndf] (gov) {بودجه دولت};
    \node[draw, rectangle, fill=red!20, minimum width=2.5cm, minimum height=0.8cm, 
        below=of ndf] (loans) {تسهیلات};
    \node[draw, rectangle, fill=gray!20, minimum width=2.5cm, minimum height=0.8cm, 
        below right=1.5cm and 0.5cm of ndf] (invest) {سرمایه‌گذاری};
    
    % Arrows with percentages
    \draw[->, thick] (oil) -- (ndf) node[midway, right] {۲۰-۴۰\%};
    \draw[->, thick, red] (ndf) -- (gov) node[midway, left] {برداشت};
    \draw[->, thick, orange] (ndf) -- (loans) node[midway, right] {عمده};
    \draw[->, thick, green!50!black, dashed] (ndf) -- (invest) node[midway, right] {ناچیز};
\end{tikzpicture}
\caption{جریان فعلی منابع صندوق توسعه ملی}
\label{fig:current-ndf-flow}
\end{figure}

\subsection{تفاوت ماهوی «دارایی» و «درآمد»}

نکته محوری در دیدگاه دکتر ناظران، تفکیک مفهومی بین «دارایی» و «درآمد» است:

\begin{quote}
\textit{«نفت و گاز دارایی‌های طبیعی هستند، نه درآمد. وقتی نفت می‌فروشیم، 
در واقع یک دارایی را به دارایی دیگر (ارز) تبدیل می‌کنیم. اگر این ارز را 
خرج کنیم، دارایی را مصرف کرده‌ایم. برای حفظ ثروت ملی، باید ارز حاصل از 
فروش نفت به دارایی‌های مالی پایدار تبدیل شود و فقط سود این دارایی‌ها 
قابل مصرف باشد.»}

\hfill --- دکتر پویا ناظران، گفت‌وگو با تجارت‌نیوز، خرداد ۱۴۰۱
\end{quote}

\begin{table}[htbp]
\centering
\caption{مقایسه رویکرد فعلی و رویکرد مطلوب به درآمدهای نفتی}
\label{tab:approach-comparison}
\begin{tabular}{|p{4cm}|p{5cm}|p{5cm}|}
\hline
\textbf{معیار} & \textbf{رویکرد فعلی ایران} & \textbf{رویکرد مطلوب} \\
\hline
ماهیت نفت & درآمد دولت & دارایی ملی \\
\hline
مالکیت & دولت & همه شهروندان (حقیقی) \\
\hline
نحوه استفاده & هزینه جاری بودجه & سرمایه‌گذاری و حفظ اصل \\
\hline
بهره‌مندی مردم & غیرمستقیم (یارانه کالایی) & مستقیم (سود نقدی) \\
\hline
افق زمانی & کوتاه‌مدت & بین‌نسلی \\
\hline
شفافیت & پایین & بالا \\
\hline
\end{tabular}
\end{table}

%═══════════════════════════════════════════════════════════════════════════════
\section{مدل‌های موفق جهانی}
\label{sec:global-models}
%═══════════════════════════════════════════════════════════════════════════════

\subsection{صندوق بازنشستگی دولتی نروژ (GPFG)}

صندوق ثروت ملی نروژ با بیش از ۱.۴ تریلیون دلار دارایی، بزرگ‌ترین صندوق 
ثروت ملی جهان است.

\begin{tcolorbox}[colback=green!5!white,colframe=green!75!black,title=ویژگی‌های کلیدی صندوق نروژ]
\begin{itemize}
    \item \textbf{قاعده مالی:} دولت فقط مجاز به برداشت سود واقعی مورد 
    انتظار (حدود ۳٪ سالانه) است.
    \item \textbf{سرمایه‌گذاری جهانی:} ۷۰٪ سهام، ۲۵٪ اوراق قرضه، ۵٪ 
    املاک در بیش از ۷۰ کشور.
    \item \textbf{شفافیت کامل:} گزارش‌های روزانه ارزش دارایی‌ها، 
    گزارش‌های سه‌ماهه و سالانه.
    \item \textbf{حاکمیت مستقل:} مدیریت توسط بانک مرکزی، نظارت پارلمان.
    \item \textbf{ملاحظات اخلاقی:} عدم سرمایه‌گذاری در شرکت‌های ناقض 
    حقوق بشر یا محیط زیست.
\end{itemize}
\end{tcolorbox}

\subsection{صندوق دائمی آلاسکا (APF)}

این مدل به دلیل پرداخت مستقیم سالانه به شهروندان، برای ایران الگوی 
جذاب‌تری است.

\begin{itemize}
    \item تأسیس: ۱۹۷۶
    \item منبع: ۲۵٪ درآمدهای نفتی ایالت
    \item ارزش فعلی: حدود ۷۵ میلیارد دلار
    \item \textbf{سود سالانه به هر شهروند:} بین ۱۰۰۰ تا ۳۰۰۰ دلار 
    (بسته به عملکرد صندوق)
\end{itemize}

\begin{figure}[htbp]
\centering
\begin{tikzpicture}
\begin{axis}[
    xlabel={سال},
    ylabel={سود سالانه (دلار)},
    xmin=1982, xmax=2022,
    ymin=0, ymax=3500,
    xtick={1985,1995,2005,2015,2022},
    ytick={0,500,1000,1500,2000,2500,3000,3500},
    legend pos=north west,
    ymajorgrids=true,
    grid style=dashed,
    width=12cm,
    height=6cm,
]
\addplot[
    color=blue,
    mark=*,
    mark size=1pt,
    ]
    coordinates {
    (1982,1000)(1985,400)(1990,950)(1995,990)(2000,1960)
    (2005,850)(2008,2100)(2010,1280)(2015,2070)(2019,1600)
    (2022,3280)
    };
\legend{سود هر شهروند آلاسکا}
\end{axis}
\end{tikzpicture}
\caption{روند سود سالانه صندوق دائمی آلاسکا به هر شهروند}
\label{fig:alaska-dividend}
\end{figure}

\subsection{سایر مدل‌ها}

\begin{table}[htbp]
\centering
\caption{مقایسه صندوق‌های ثروت ملی منتخب}
\label{tab:swf-comparison}
\begin{tabular}{|l|r|l|l|}
\hline
\textbf{صندوق} & \textbf{دارایی (میلیارد \$)} & \textbf{منبع} & \textbf{توزیع مستقیم} \\
\hline
نروژ (GPFG) & ۱,۴۰۰ & نفت و گاز & خیر \\
ابوظبی (ADIA) & ۸۵۰ & نفت & خیر \\
کویت (KIA) & ۷۵۰ & نفت & خیر \\
سنگاپور (GIC) & ۶۹۰ & ذخایر ارزی & خیر \\
آلاسکا (APF) & ۷۵ & نفت & \textbf{بله} \\
\hline
\end{tabular}
\end{table}

%═══════════════════════════════════════════════════════════════════════════════
\section{طرح پیشنهادی: صندوق ثروت ملی ایران}
\label{sec:iran-swf-proposal}
%═══════════════════════════════════════════════════════════════════════════════

بر اساس دیدگاه‌های دکتر ناظران و با الهام از تجربیات موفق جهانی، 
طرح زیر پیشنهاد می‌شود:

\subsection{اصول بنیادین}

\begin{enumerate}
    \item \textbf{تفکیک دارایی از درآمد:} تمام درآمدهای حاصل از فروش 
    منابع طبیعی تجدیدناپذیر (نفت، گاز، معادن) باید به صندوق واریز شود.
    
    \item \textbf{مالکیت شهروندی:} هر شهروند ایرانی مالک سهمی برابر از 
    صندوق است که این مالکیت قابل انتقال یا فروش نیست.
    
    \item \textbf{حفظ اصل سرمایه:} اصل دارایی‌های صندوق غیرقابل برداشت 
    است و فقط بخشی از سود قابل توزیع است.
    
    \item \textbf{استقلال مدیریتی:} صندوق باید از دولت مستقل بوده و 
    توسط هیئت امنای حرفه‌ای اداره شود.
    
    \item \textbf{شفافیت کامل:} گزارش‌های مالی، استراتژی سرمایه‌گذاری 
    و پرتفوی دارایی‌ها باید به صورت عمومی منتشر شود.
\end{enumerate}

\subsection{ساختار پیشنهادی}

\begin{figure}[htbp]
\centering
\begin{tikzpicture}[
    box/.style={rectangle, draw, fill=#1, minimum width=3cm, minimum height=1cm, 
        text centered, rounded corners},
    arrow/.style={->, thick, >=stealth}
]
    % منابع
    \node[box=blue!20] (oil) at (0,6) {درآمد نفت و گاز};
    \node[box=blue!20] (mining) at (5,6) {درآمد معادن};
    \node[box=blue!20] (other) at (10,6) {سایر منابع طبیعی};
    
    % صندوق اصلی
    \node[box=green!30, minimum width=10cm, minimum height=1.5cm] (fund) at (5,4) 
        {\textbf{صندوق ثروت ملی ایران}};
    
    % زیرصندوق‌ها
    \node[box=orange!20, minimum width=3cm] (stab) at (0,2) {حساب تثبیت\\(۲۰٪)};
    \node[box=orange!20, minimum width=3cm] (invest) at (5,2) {حساب سرمایه‌گذاری\\(۶۰٪)};
    \node[box=orange!20, minimum width=3cm] (future) at (10,2) {حساب نسل آینده\\(۲۰٪)};
    
    % خروجی‌ها
    \node[box=yellow!30] (budget) at (0,0) {تثبیت بودجه};
    \node[box=yellow!30] (dividend) at (5,0) {سود شهروندی};
    \node[box=yellow!30] (reserve) at (10,0) {ذخیره بین‌نسلی};
    
    % فلش‌ها
    \draw[arrow] (oil) -- (fund);
    \draw[arrow] (mining) -- (fund);
    \draw[arrow] (other) -- (fund);
    \draw[arrow] (fund) -- (stab);
    \draw[arrow] (fund) -- (invest);
    \draw[arrow] (fund) -- (future);
    \draw[arrow] (stab) -- (budget);
    \draw[arrow] (invest) -- (dividend);
    \draw[arrow] (future) -- (reserve);
\end{tikzpicture}
\caption{ساختار پیشنهادی صندوق ثروت ملی ایران}
\label{fig:proposed-swf-structure}
\end{figure}

\subsection{سه حساب اصلی}

\subsubsection{حساب تثبیت (۲۰٪)}
\begin{itemize}
    \item هدف: جذب نوسانات قیمت نفت و جلوگیری از بی‌ثباتی بودجه
    \item سقف: معادل یک سال هزینه‌های جاری دولت
    \item برداشت: فقط در صورت کاهش شدید درآمدهای نفتی
    \item سرمایه‌گذاری: دارایی‌های نقدشونده و کم‌ریسک
\end{itemize}

\subsubsection{حساب سرمایه‌گذاری (۶۰٪)}
\begin{itemize}
    \item هدف: رشد ثروت ملی و تولید درآمد پایدار
    \item استراتژی: سرمایه‌گذاری متنوع جهانی (سهام، اوراق، املاک)
    \item توزیع سود: سالانه به صورت مستقیم به شهروندان
    \item قاعده برداشت: حداکثر ۴٪ ارزش میانگین پنج‌ساله
\end{itemize}

\subsubsection{حساب نسل آینده (۲۰٪)}
\begin{itemize}
    \item هدف: حفظ ثروت برای نسل‌های آینده پس از پایان منابع
    \item قاعده: غیرقابل برداشت تا ۵۰ سال
    \item سرمایه‌گذاری: بلندمدت با پذیرش ریسک بالاتر
\end{itemize}

\subsection{مکانیزم توزیع مستقیم سود (سود شهروندی)}

\begin{algorithm}[H]
\caption{محاسبه و توزیع سود شهروندی سالانه}
\label{alg:dividend-distribution}
\begin{algorithmic}[1]
\REQUIRE ارزش صندوق سرمایه‌گذاری: $V$، جمعیت واجد شرایط: $P$
\STATE محاسبه میانگین ارزش پنج‌ساله: $\bar{V}_5 = \frac{1}{5}\sum_{i=0}^{4} V_{t-i}$
\STATE تعیین نرخ برداشت پایدار: $r = 0.04$ (۴ درصد)
\STATE محاسبه کل سود قابل توزیع: $D = r \times \bar{V}_5$
\STATE محاسبه سهم هر شهروند: $d = \frac{D}{P}$
\STATE واریز به حساب بانکی متصل به کد ملی
\ENSURE هر شهروند مبلغ $d$ را دریافت می‌کند
\end{algorithmic}
\end{algorithm}

\begin{example}[محاسبه نمونه]
فرض کنید:
\begin{itemize}
    \item ارزش حساب سرمایه‌گذاری: ۳۰۰ میلیارد دلار
    \item نرخ برداشت: ۴٪
    \item جمعیت: ۸۵ میلیون نفر
\end{itemize}

محاسبه:
\begin{align}
D &= 300 \times 0.04 = 12 \text{ میلیارد دلار} \\
d &= \frac{12,000,000,000}{85,000,000} \approx 141 \text{ دلار به ازای هر نفر}
\end{align}

با نرخ ارز ۵۰,۰۰۰ تومان، این معادل حدود ۷ میلیون تومان سالانه برای 
هر شهروند است.
\end{example}

%═══════════════════════════════════════════════════════════════════════════════
\section{نقشه راه پیاده‌سازی}
\label{sec:implementation-roadmap}
%═══════════════════════════════════════════════════════════════════════════════

\subsection{فاز یکم: آماده‌سازی (سال اول)}

\begin{enumerate}
    \item تصویب قانون اساسنامه صندوق ثروت ملی
    \item تعیین هیئت امنای مستقل و حرفه‌ای
    \item ایجاد زیرساخت فناوری اطلاعات
    \item راه‌اندازی سامانه ثبت شهروندان واجد شرایط
    \item تدوین سیاست سرمایه‌گذاری
\end{enumerate}

\subsection{فاز دوم: انتقال (سال دوم و سوم)}

\begin{enumerate}
    \item انتقال دارایی‌های صندوق توسعه ملی
    \item بازیافت تسهیلات پرداختی قابل وصول
    \item شروع سرمایه‌گذاری متنوع جهانی
    \item واریز ۱۰۰٪ درآمدهای نفتی جدید به صندوق
    \item کاهش تدریجی وابستگی بودجه به نفت
\end{enumerate}

\subsection{فاز سوم: بلوغ (سال چهارم به بعد)}

\begin{enumerate}
    \item شروع پرداخت سود شهروندی
    \item گزارش‌دهی شفاف سه‌ماهه
    \item ارزیابی و بهبود مستمر عملکرد
    \item گسترش تدریجی پرتفوی سرمایه‌گذاری
\end{enumerate}

\begin{figure}[htbp]
\centering
\begin{ganttchart}[
    hgrid,
    vgrid,
    x unit=0.8cm,
    y unit chart=0.7cm,
    bar/.style={fill=blue!40},
    bar height=0.5,
    title height=1,
    title label font=\footnotesize,
    bar label font=\footnotesize,
]{1}{12}
\gantttitle{نقشه راه پیاده‌سازی (سال)}{12} \\
\gantttitlelist{1,...,12}{1} \\
\ganttbar{تصویب قانون}{1}{2} \\
\ganttbar{انتخاب هیئت امنا}{2}{3} \\
\ganttbar{زیرساخت فناوری}{2}{4} \\
\ganttbar{انتقال دارایی‌ها}{4}{6} \\
\ganttbar{شروع سرمایه‌گذاری}{5}{12} \\
\ganttbar{کاهش وابستگی بودجه}{4}{10} \\
\ganttbar{اولین پرداخت سود}{8}{8} \\
\ganttbar{گزارش‌دهی منظم}{6}{12}
\end{ganttchart}
\caption{جدول زمانی پیاده‌سازی صندوق ثروت ملی}
\label{fig:implementation-gantt}
\end{figure}

%═══════════════════════════════════════════════════════════════════════════════
\section{حاکمیت و نظارت}
\label{sec:governance}
%═══════════════════════════════════════════════════════════════════════════════

\subsection{ساختار حاکمیتی}

\begin{figure}[htbp]
\centering
\begin{tikzpicture}[
    org/.style={rectangle, draw, fill=#1, minimum width=3.5cm, 
        minimum height=0.8cm, text centered, font=\small},
    level 1/.style={sibling distance=5cm},
    level 2/.style={sibling distance=3cm},
    edge from parent/.style={draw, -latex}
]
\node[org=red!20] {مجلس (نظارت عالی)}
    child {node[org=orange!20] {هیئت امنا (۷ نفر)}
        child {node[org=green!20] {مدیرعامل}}
        child {node[org=blue!20] {کمیته سرمایه‌گذاری}}
        child {node[org=yellow!20] {کمیته ریسک}}
    };
\end{tikzpicture}
\caption{ساختار حاکمیتی صندوق}
\label{fig:governance-structure}
\end{figure}

\subsection{تضمین‌های استقلال}

\begin{itemize}
    \item \textbf{قانون اساسی:} درج اصول کلی در قانون اساسی برای جلوگیری 
    از تغییرات مکرر
    \item \textbf{دوره ثابت:} دوره پنج‌ساله برای اعضای هیئت امنا، غیرقابل 
    عزل جز با رأی دوسوم مجلس
    \item \textbf{ممنوعیت برداشت:} برداشت از اصل صندوق فقط با رأی 
    سه‌چهارم مجلس و تأیید شورای نگهبان
    \item \textbf{حسابرسی مستقل:} حسابرسی سالانه توسط شرکت‌های معتبر 
    بین‌المللی
\end{itemize}

%═══════════════════════════════════════════════════════════════════════════════
\section{چالش‌ها و راهکارها}
\label{sec:challenges}
%═══════════════════════════════════════════════════════════════════════════════

\begin{table}[htbp]
\centering
\caption{چالش‌های پیش‌رو و راهکارهای پیشنهادی}
\label{tab:challenges}
\begin{tabular}{|p{4cm}|p{5cm}|p{5cm}|}
\hline
\textbf{چالش} & \textbf{توضیح} & \textbf{راهکار} \\
\hline
تحریم‌های مالی & محدودیت در سرمایه‌گذاری جهانی & 
تمرکز اولیه بر بازارهای غیرغربی، سرمایه‌گذاری داخلی، 
انتظار برای رفع تحریم‌ها \\
\hline
مقاومت سیاسی & گروه‌های ذی‌نفع از وضع موجود & 
شفاف‌سازی منافع عمومی، ائتلاف‌سازی اجتماعی \\
\hline
کسری بودجه & وابستگی فعلی بودجه به نفت & 
دوره انتقال تدریجی، اصلاحات مالیاتی همزمان \\
\hline
ظرفیت مدیریتی & کمبود نیروی متخصص & 
استخدام مدیران بین‌المللی، آموزش نیروی داخلی \\
\hline
تورم انتظاری & نگرانی از اثر تورمی توزیع نقدی & 
توزیع تدریجی، امکان سرمایه‌گذاری مجدد \\
\hline
\end{tabular}
\end{table}

%═══════════════════════════════════════════════════════════════════════════════
\section{مزایای توزیع مستقیم}
\label{sec:direct-distribution-benefits}
%═══════════════════════════════════════════════════════════════════════════════

\subsection{مزایای اقتصادی}

\begin{enumerate}
    \item \textbf{کاهش رانت‌جویی:} وقتی درآمد نفت مستقیماً به مردم 
    برسد، انگیزه رانت‌جویی از دولت کاهش می‌یابد.
    
    \item \textbf{افزایش پاسخگویی:} شهروندانی که مستقیماً از نفت درآمد 
    دارند، نظارت بیشتری بر مدیریت آن خواهند داشت.
    
    \item \textbf{تحریک تقاضا:} توزیع عادلانه قدرت خرید، تقاضای مؤثر 
    را افزایش می‌دهد.
    
    \item \textbf{کاهش نابرابری:} توزیع مساوی به همه شهروندان، ضریب 
    جینی را کاهش می‌دهد.
\end{enumerate}

\subsection{مزایای سیاسی}

\begin{enumerate}
    \item \textbf{تقویت قرارداد اجتماعی:} ارتباط مستقیم شهروند-ثروت 
    ملی، حس مالکیت و مسئولیت ایجاد می‌کند.
    
    \item \textbf{مشروعیت‌بخشی:} دولتی که ثروت ملی را عادلانه توزیع 
    می‌کند، مشروعیت بیشتری دارد.
    
    \item \textbf{کاهش پوپولیسم:} وقتی همه از ثروت ملی بهره‌مند شوند، 
    وعده‌های پوپولیستی کمتر جذاب می‌شود.
\end{enumerate}

%═══════════════════════════════════════════════════════════════════════════════
\section{جمع‌بندی و پیشنهادها}
\label{sec:conclusion}
%═══════════════════════════════════════════════════════════════════════════════

تبدیل صندوق توسعه ملی به یک صندوق ثروت ملی واقعی با قابلیت توزیع مستقیم 
سود به شهروندان، یکی از مهم‌ترین اصلاحات ساختاری اقتصاد ایران است. این 
تحول می‌تواند:

\begin{itemize}
    \item نفرین منابع را به برکت منابع تبدیل کند
    \item عدالت بین‌نسلی را تضمین نماید
    \item رانت‌جویی و فساد را کاهش دهد
    \item پاسخگویی دولت را افزایش دهد
    \item نابرابری را کاهش دهد
\end{itemize}

\begin{tcolorbox}[colback=blue!5!white,colframe=blue!75!black,
    title=پیشنهادهای کلیدی]
\begin{enumerate}
    \item تصویب قانون تبدیل صندوق توسعه ملی به صندوق ثروت ملی
    \item تضمین استقلال صندوق از دولت در قانون اساسی
    \item شروع پرداخت سود شهروندی حداکثر تا پایان سال سوم
    \item شفافیت کامل در گزارش‌دهی و عملکرد
    \item اصلاحات مالیاتی همزمان برای جایگزینی درآمدهای نفتی بودجه
\end{enumerate}
\end{tcolorbox}

%═══════════════════════════════════════════════════════════════════════════════
% منابع این فصل
%═══════════════════════════════════════════════════════════════════════════════
\section*{منابع}
\addcontentsline{toc}{section}{منابع}

\begin{itemize}
    \item ناظران، پویا. «دولت قیمت‌گذار: بررسی ماهیت و چیستی اصلاحات 
    اقتصادی»، گفت‌وگو با تجارت‌نیوز، خرداد ۱۴۰۱.
    
    \item صندوق بازنشستگی دولتی نروژ، گزارش سالانه ۲۰۲۲.
    \lr{https://www.nbim.no}
    
    \item صندوق دائمی آلاسکا، گزارش‌های سالانه.
    \lr{https://apfc.org}
    
    \item Ross, Michael L. \textit{The Oil Curse: How Petroleum Wealth 
    Shapes the Development of Nations}. Princeton University Press, 2012.
    
    \item Wenar, Leif. \textit{Blood Oil: Tyrants, Violence, and the Rules 
    that Run the World}. Oxford University Press, 2016.
\end{itemize}