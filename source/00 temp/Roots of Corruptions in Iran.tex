\chapter{آناتومی فساد و ناترازی‌های ساختاری در اقتصاد ایران}
\label{chap:corruption-imbalances}

\begin{abstract}
این فصل به بررسی جامع زمینه‌ها، ساختارها و مکانیزم‌های فساد و ناترازی‌های 
اقتصادی در ایران معاصر می‌پردازد. با تحلیل نظام‌مند حوزه‌های کلیدی شامل 
نظام ارزی، مالیاتی، بانکی، بودجه‌ای و قیمت‌گذاری، ریشه‌های ساختاری فساد 
شناسایی و اولویت‌بندی اصلاحات پیشنهاد می‌شود. استدلال اصلی این است که 
فساد در ایران نه یک انحراف، بلکه محصول منطقی ساختارهای نهادی موجود است 
و مبارزه با آن مستلزم اصلاحات بنیادین است، نه صرفاً برخورد قضایی.
\end{abstract}

%═══════════════════════════════════════════════════════════════════════════════
\section{مقدمه: فساد به مثابه نظام}
\label{sec:corruption-system}
%═══════════════════════════════════════════════════════════════════════════════

\subsection{تعریف و گونه‌شناسی فساد}

فساد در ساده‌ترین تعریف، «سوءاستفاده از قدرت عمومی برای منفعت خصوصی» 
است. اما این تعریف ساده، پیچیدگی پدیده فساد را پنهان می‌کند. برای درک 
فساد در ایران، باید میان انواع مختلف آن تمایز قائل شد:

\begin{table}[htbp]
\centering
\caption{گونه‌شناسی فساد}
\label{tab:corruption-typology-intro}
\begin{tabular}{|l|p{4cm}|p{4cm}|p{3cm}|}
\hline
\textbf{نوع} & \textbf{تعریف} & \textbf{مثال در ایران} & \textbf{مقیاس} \\
\hline
\multicolumn{4}{|c|}{\cellcolor{red!15}\textbf{فساد بزرگ (Grand Corruption)}} \\
\hline
سیاسی & تصمیمات کلان برای منافع گروهی & تخصیص ارز ترجیحی & بسیار بزرگ \\
\hline
اقتصادی & معاملات بزرگ دولتی-خصوصی & قراردادهای نفتی، مناقصات & بسیار بزرگ \\
\hline
\multicolumn{4}{|c|}{\cellcolor{orange!15}\textbf{فساد نهادی (Institutional)}} \\
\hline
ساختاری & قوانین فسادزا & معافیت‌های مالیاتی نهادها & سیستمی \\
\hline
نظارتی & ضعف یا فقدان نظارت & عدم شفافیت صندوق توسعه & سیستمی \\
\hline
\multicolumn{4}{|c|}{\cellcolor{yellow!15}\textbf{فساد خُرد (Petty Corruption)}} \\
\hline
اداری & رشوه در امور روزمره & گمرک، شهرداری، پلیس & گسترده \\
\hline
خدماتی & پرداخت برای خدمات عمومی & بهداشت، آموزش & متوسط \\
\hline
\end{tabular}
\end{table}

\subsection{چارچوب تحلیلی: معادله فساد}

برای تحلیل فساد از چارچوب کلیتگارد (Klitgaard) استفاده می‌کنیم:

\begin{equation}
\boxed{\text{فساد} = \text{انحصار} + \text{صلاحدید} - \text{پاسخگویی}}
\label{eq:corruption-formula}
\end{equation}

\begin{tcolorbox}[colback=blue!5!white,colframe=blue!75!black,
    title=تفسیر معادله فساد در بستر ایران]
\begin{itemize}
    \item \textbf{انحصار (Monopoly):} تمرکز قدرت تصمیم‌گیری در دست 
    گروه محدود؛ در ایران: انحصارات دولتی و شبه‌دولتی، محدودیت رقابت
    
    \item \textbf{صلاحدید (Discretion):} آزادی عمل بدون قاعده مشخص؛ 
    در ایران: قوانین مبهم، استثناها و معافیت‌های فراوان
    
    \item \textbf{پاسخگویی (Accountability):} نظارت و مجازات؛ 
    در ایران: ضعف رسانه‌های مستقل، محدودیت جامعه مدنی، عدم استقلال قضایی
\end{itemize}
\end{tcolorbox}

\subsection{ویژگی‌های فساد در ایران}

فساد در ایران ویژگی‌های خاصی دارد که آن را از بسیاری کشورها متمایز می‌کند:

\begin{enumerate}
    \item \textbf{فساد سیستمی نه استثنایی:} فساد در بافت نهادها تنیده شده، 
    نه یک انحراف از هنجار.
    
    \item \textbf{ارتباط با رانت نفتی:} بخش عمده فساد به توزیع درآمدهای 
    نفتی مربوط است.
    
    \item \textbf{چندلایگی نهادی:} وجود نهادهای موازی (دولت، سپاه، بنیادها) 
    که هر کدام حوزه فساد خاص دارند.
    
    \item \textbf{قانونی‌سازی فساد:} بسیاری از رانت‌ها قانونی هستند 
    (معافیت‌ها، انحصارات رسمی).
    
    \item \textbf{تحریم به مثابه توجیه:} تحریم‌ها بهانه‌ای برای کاهش 
    شفافیت و افزایش صلاحدید شده‌اند.
\end{enumerate}

%═══════════════════════════════════════════════════════════════════════════════
\section{نقشه جامع ناترازی‌ها و رانت‌ها}
\label{sec:imbalances-map}
%═══════════════════════════════════════════════════════════════════════════════

\begin{figure}[htbp]
\centering
\begin{tikzpicture}[
    main/.style={rectangle, draw, fill=red!30, minimum width=4cm, 
        minimum height=1.2cm, text centered, font=\bfseries, rounded corners},
    sub/.style={rectangle, draw, fill=#1, minimum width=3.5cm, 
        minimum height=0.9cm, text centered, font=\small, rounded corners},
    arrow/.style={->, thick, >=stealth}
]

% مرکز
\node[main] (center) at (0,0) {ناترازی‌های ساختاری\\اقتصاد ایران};

% شاخه‌ها
\node[sub=orange!30] (currency) at (-6,3) {ناترازی ارزی};
\node[sub=blue!30] (tax) at (-3,4.5) {ناترازی مالیاتی};
\node[sub=green!30] (budget) at (3,4.5) {ناترازی بودجه‌ای};
\node[sub=purple!30] (banking) at (6,3) {ناترازی بانکی};
\node[sub=yellow!30] (price) at (-6,-3) {ناترازی قیمتی};
\node[sub=cyan!30] (labor) at (-3,-4.5) {ناترازی بازار کار};
\node[sub=pink!30] (pension) at (3,-4.5) {ناترازی بازنشستگی};
\node[sub=lime!30] (regional) at (6,-3) {ناترازی منطقه‌ای};

% فلش‌ها
\draw[arrow] (center) -- (currency);
\draw[arrow] (center) -- (tax);
\draw[arrow] (center) -- (budget);
\draw[arrow] (center) -- (banking);
\draw[arrow] (center) -- (price);
\draw[arrow] (center) -- (labor);
\draw[arrow] (center) -- (pension);
\draw[arrow] (center) -- (regional);

% برآورد رانت
\node[font=\tiny, red] at (-6,2.3) {۱۵-۲۵ B\$};
\node[font=\tiny, red] at (-3,3.8) {۳۰-۵۰ B\$};
\node[font=\tiny, red] at (3,3.8) {۱۵-۲۰ B\$};
\node[font=\tiny, red] at (6,2.3) {۲۰-۴۰ B\$};
\node[font=\tiny, red] at (-6,-3.7) {۲۵-۳۵ B\$};
\node[font=\tiny, red] at (-3,-5.2) {۱۰-۱۵ B\$};
\node[font=\tiny, red] at (3,-5.2) {۵-۱۰ B\$};
\node[font=\tiny, red] at (6,-3.7) {۱۰-۲۰ B\$};

\end{tikzpicture}
\caption{نقشه کلی ناترازی‌های ساختاری و برآورد رانت سالانه (میلیارد دلار)}
\label{fig:imbalances-map}
\end{figure}

\begin{table}[htbp]
\centering
\caption{خلاصه ناترازی‌های اصلی و مکانیزم فسادزایی}
\label{tab:imbalances-summary}
\small
\begin{tabular}{|l|p{3.5cm}|p{3.5cm}|r|c|}
\hline
\textbf{ناترازی} & \textbf{شرح} & \textbf{مکانیزم فساد} & \textbf{رانت (B\$)} & \textbf{اولویت} \\
\hline
ارزی & نرخ‌های چندگانه ارز & آربیتراژ، رانت‌جویی & ۱۵-۲۵ & ●●●●● \\
\hline
مالیاتی & معافیت‌های گسترده & فرار، رقابت ناعادلانه & ۳۰-۵۰ & ●●●●● \\
\hline
بودجه‌ای & کسری مزمن، وابستگی نفتی & چانه‌زنی، عدم شفافیت & ۱۵-۲۰ & ●●●●○ \\
\hline
بانکی & مطالبات معوق، بانک‌های ناتراز & تسهیلات رانتی & ۲۰-۴۰ & ●●●●● \\
\hline
قیمتی & یارانه پنهان انرژی & قاچاق، مصرف بیش از حد & ۲۵-۳۵ & ●●●●○ \\
\hline
بازار کار & دوگانگی رسمی/غیررسمی & فرار از قانون کار & ۱۰-۱۵ & ●●●○○ \\
\hline
بازنشستگی & کسری صندوق‌ها & فشار بر بودجه آینده & ۵-۱۰ & ●●●○○ \\
\hline
منطقه‌ای & تمرکز تهران & توزیع ناعادلانه منابع & ۱۰-۲۰ & ●●●●○ \\
\hline
\multicolumn{3}{|r|}{\textbf{مجموع برآورد سالانه}} & \textbf{۱۳۰-۲۱۵} & --- \\
\hline
\end{tabular}
\end{table}

%═══════════════════════════════════════════════════════════════════════════════
\section{ناترازی ارزی: مادر همه رانت‌ها}
\label{sec:currency-imbalance}
%═══════════════════════════════════════════════════════════════════════════════

\subsection{تاریخچه نظام چندنرخی ارز}

ایران تقریباً در تمام دوران پس از انقلاب با نظام چندنرخی ارز مواجه بوده است:

\begin{table}[htbp]
\centering
\caption{تاریخچه نظام‌های ارزی ایران (۱۳۵۸-۱۴۰۳)}
\label{tab:fx-history}
\small
\begin{tabular}{|c|l|p{5cm}|c|}
\hline
\textbf{دوره} & \textbf{نظام} & \textbf{ویژگی} & \textbf{شکاف ارزی} \\
\hline
۱۳۵۸-۱۳۶۸ & چندنرخی شدید & جنگ، کمبود ارز، سهمیه‌بندی & تا ۱۰۰۰٪ \\
\hline
۱۳۶۸-۱۳۷۲ & یکسان‌سازی تدریجی & دولت سازندگی، اصلاحات & کاهش‌یافته \\
\hline
۱۳۷۲-۱۳۸۱ & شناور مدیریت‌شده & بازار ثانویه، نرخ توافقی & ۱۰-۵۰٪ \\
\hline
۱۳۸۱-۱۳۹۰ & یکسان‌سازی & دولت خاتمی و احمدی‌نژاد (اوایل) & کمتر از ۱۰٪ \\
\hline
۱۳۹۰-۱۳۹۵ & چندنرخی مجدد & تحریم‌ها، ارز دولتی/آزاد & تا ۳۰۰٪ \\
\hline
۱۳۹۵-۱۳۹۷ & شبه‌یکسان & برجام، نرخ نیمایی & ۵-۱۵٪ \\
\hline
۱۳۹۷-۱۴۰۱ & چندنرخی شدید & ارز ۴۲۰۰، نیمایی، آزاد & تا ۱۲۰۰٪ \\
\hline
۱۴۰۱-کنون & دونرخی & نیمایی و آزاد & ۳۰-۸۰٪ \\
\hline
\end{tabular}
\end{table}

\subsection{ساختار فعلی بازار ارز}

\begin{figure}[htbp]
\centering
\begin{tikzpicture}[
    box/.style={rectangle, draw, fill=#1, minimum width=3cm, 
        minimum height=1.5cm, text centered, font=\small, rounded corners},
    arrow/.style={->, thick, >=stealth}
]

% نرخ‌های مختلف
\node[box=green!30] (nima) at (0,3) {نرخ نیمایی\\۵۰,۰۰۰ تومان};
\node[box=red!30] (free) at (0,0) {نرخ آزاد\\۶۵,۰۰۰ تومان};
\node[box=blue!30] (sana) at (0,-3) {نرخ سنا (توافقی)\\۵۵,۰۰۰ تومان};

% کاربردها
\node[font=\small, right] at (2.5,3) {صادرات غیرنفتی، واردات};
\node[font=\small, right] at (2.5,0) {صرافی‌ها، بازار موازی};
\node[font=\small, right] at (2.5,-3) {مصارف خدماتی};

% شکاف
\draw[<->, red, thick] (1.8,3) -- (1.8,0) node[midway, right, font=\small] {۳۰٪ شکاف};

\end{tikzpicture}
\caption{ساختار فعلی نرخ‌های ارز در ایران (تیر ۱۴۰۳)}
\label{fig:current-fx-structure}
\end{figure}

\subsection{مکانیزم‌های رانت‌جویی ارزی}

\begin{tcolorbox}[colback=red!5!white,colframe=red!75!black,
    title=انواع رانت‌جویی ارزی]

\textbf{۱. رانت واردات با ارز ترجیحی}
\begin{itemize}
    \item دریافت ارز با نرخ پایین‌تر (قبلاً ۴۲۰۰، اکنون نیمایی)
    \item واردات کالا و فروش به قیمت ارز آزاد
    \item سود: تفاوت نرخ ارز × مقدار
\end{itemize}

\textbf{۲. رانت صادرات}
\begin{itemize}
    \item کم‌اظهاری در ارزش صادرات
    \item بازنگرداندن ارز صادراتی
    \item فروش ارز در بازار آزاد
\end{itemize}

\textbf{۳. آربیتراژ ارزی}
\begin{itemize}
    \item خرید ارز از کانال رسمی با روابط
    \item فروش در بازار آزاد
    \item تکرار چرخه
\end{itemize}

\textbf{۴. کم‌اظهاری و اضافه‌اظهاری}
\begin{itemize}
    \item اظهار قیمت بالاتر برای واردات (دریافت ارز بیشتر)
    \item اظهار قیمت پایین‌تر برای صادرات (بازگشت ارز کمتر)
\end{itemize}
\end{tcolorbox}

\subsection{برآورد حجم رانت ارزی}

\begin{figure}[htbp]
\centering
\begin{tikzpicture}
\begin{axis}[
    ybar stacked,
    xlabel={سال},
    ylabel={میلیارد دلار},
    symbolic x coords={۱۳۹۷,۱۳۹۸,۱۳۹۹,۱۴۰۰,۱۴۰۱,۱۴۰۲},
    xtick=data,
    legend pos=north west,
    ymajorgrids=true,
    grid style=dashed,
    width=14cm,
    height=8cm,
    bar width=0.7cm,
    ymin=0,
    ymax=35,
]

\addplot[fill=red!60] coordinates {
    (۱۳۹۷,15) (۱۳۹۸,12) (۱۳۹۹,18) (۱۴۰۰,16) (۱۴۰۱,14) (۱۴۰۲,8)
};
\addlegendentry{رانت ارز ۴۲۰۰/ترجیحی}

\addplot[fill=orange!60] coordinates {
    (۱۳۹۷,5) (۱۳۹۸,6) (۱۳۹۹,8) (۱۴۰۰,7) (۱۴۰۱,6) (۱۴۰۲,5)
};
\addlegendentry{رانت آربیتراژ نیمایی-آزاد}

\addplot[fill=yellow!60] coordinates {
    (۱۳۹۷,3) (۱۳۹۸,4) (۱۳۹۹,5) (۱۴۰۰,5) (۱۴۰۱,4) (۱۴۰۲,4)
};
\addlegendentry{رانت عدم بازگشت ارز صادراتی}

\end{axis}
\end{tikzpicture}
\caption{برآورد رانت ارزی سالانه به تفکیک منبع (۱۳۹۷-۱۴۰۲)}
\label{fig:fx-rent-breakdown}
\end{figure}

\subsection{ذی‌نفعان رانت ارزی}

\begin{table}[htbp]
\centering
\caption{گروه‌های ذی‌نفع از رانت ارزی}
\label{tab:fx-rent-beneficiaries}
\begin{tabular}{|l|p{4cm}|p{3cm}|c|}
\hline
\textbf{گروه} & \textbf{نحوه بهره‌برداری} & \textbf{ارتباط قدرت} & \textbf{سهم تخمینی} \\
\hline
واردکنندگان بزرگ & واردات با ارز ترجیحی & اتاق بازرگانی، وزارتخانه‌ها & ۴۰٪ \\
\hline
شبکه‌های قاچاق & واردات غیررسمی، قاچاق معکوس & مرزها، برخی نهادها & ۲۵٪ \\
\hline
صرافان و دلالان & آربیتراژ و سفته‌بازی & شبکه‌های مالی & ۱۵٪ \\
\hline
صادرکنندگان متخلف & عدم بازگشت ارز & --- & ۱۲٪ \\
\hline
مقامات و واسطه‌ها & کمیسیون و رشوه تخصیص & دولت، بانک‌ها & ۸٪ \\
\hline
\end{tabular}
\end{table}

\subsection{راهکار: یکسان‌سازی نرخ ارز}

\begin{tcolorbox}[colback=green!5!white,colframe=green!75!black,
    title=نقشه راه یکسان‌سازی ارز]

\textbf{فاز ۱: آماده‌سازی (۶ ماه)}
\begin{itemize}
    \item اعلام عمومی تصمیم و زمان‌بندی
    \item شناسایی گروه‌های آسیب‌پذیر
    \item طراحی سیستم یارانه نقدی جایگزین
\end{itemize}

\textbf{فاز ۲: اجرا (۳ ماه)}
\begin{itemize}
    \item حذف تمام نرخ‌های ترجیحی
    \item تعیین نرخ شناور مدیریت‌شده واحد
    \item پرداخت یارانه نقدی به دهک‌های پایین
\end{itemize}

\textbf{فاز ۳: تثبیت (۱۲ ماه)}
\begin{itemize}
    \item نظارت بر بازار و جلوگیری از سفته‌بازی
    \item تعدیل تدریجی نرخ بر اساس عرضه و تقاضا
    \item ارزیابی اثرات و اصلاحات لازم
\end{itemize}

\textbf{پیش‌نیازها:}
\begin{itemize}
    \item ذخایر ارزی کافی (حداقل ۲۰ میلیارد دلار نقد)
    \item کنترل انتظارات تورمی
    \item ثبات نسبی سیاسی
\end{itemize}
\end{tcolorbox}

%═══════════════════════════════════════════════════════════════════════════════
\section{ناترازی مالیاتی: نظام دوگانه و ناعادلانه}
\label{sec:tax-imbalance}
%═══════════════════════════════════════════════════════════════════════════════

\subsection{ویژگی‌های نظام مالیاتی ایران}

نظام مالیاتی ایران یکی از ناکارآمدترین نظام‌های مالیاتی جهان است:

\begin{table}[htbp]
\centering
\caption{مقایسه شاخص‌های مالیاتی ایران با سایر کشورها}
\label{tab:tax-comparison}
\begin{tabular}{|l|c|c|c|c|}
\hline
\textbf{شاخص} & \textbf{ایران} & \textbf{ترکیه} & \textbf{OECD} & \textbf{نروژ} \\
\hline
نسبت مالیات به GDP & ۷٪ & ۲۴٪ & ۳۴٪ & ۳۹٪ \\
\hline
سهم مالیات از درآمد دولت & ۳۵٪ & ۸۵٪ & ۹۰٪ & ۶۵٪ \\
\hline
مالیات بر درآمد شرکت‌ها/GDP & ۱.۲٪ & ۲.۱٪ & ۳.۱٪ & ۵.۸٪ \\
\hline
مالیات بر دارایی/GDP & ۰.۳٪ & ۱.۳٪ & ۱.۹٪ & ۱.۱٪ \\
\hline
ضریب جینی مالیاتی & ۰.۸۵ & ۰.۶۵ & ۰.۵۵ & ۰.۴۵ \\
\hline
\end{tabular}
\end{table}

\subsection{معافیت‌های مالیاتی: سیاه‌چاله بودجه}

\begin{figure}[htbp]
\centering
\begin{tikzpicture}
\pie[
    text=legend,
    radius=4,
    color={red!60, orange!60, yellow!60, green!60, blue!60, purple!60, gray!60}
]{
    28/نهادهای معاف (بنیادها، آستان‌ها),
    22/مناطق آزاد و ویژه,
    18/معافیت‌های صنعتی و کشاورزی,
    12/معافیت‌های صادراتی,
    10/بخش غیررسمی (فرار مالیاتی),
    6/معافیت‌های فرهنگی و ورزشی,
    4/سایر معافیت‌ها
}
\end{tikzpicture}
\caption{ترکیب معافیت‌ها و فرار مالیاتی (درصد از کل)}
\label{fig:tax-exemptions-pie}
\end{figure}

\begin{table}[htbp]
\centering
\caption{برآورد معافیت‌های مالیاتی اصلی}
\label{tab:tax-exemptions-detail}
\small
\begin{tabular}{|l|p{4cm}|r|p{4cm}|}
\hline
\textbf{نهاد/بخش} & \textbf{نوع معافیت} & \textbf{هزینه مالیاتی (T\$)} & \textbf{مستند قانونی} \\
\hline
\multicolumn{4}{|c|}{\cellcolor{red!15}\textbf{نهادهای شبه‌دولتی}} \\
\hline
ستاد اجرایی فرمان امام & معافیت کامل & ۱۵۰,۰۰۰+ & حکم حکومتی \\
\hline
بنیاد مستضعفان & معافیت کامل & ۱۰۰,۰۰۰+ & قانون خاص \\
\hline
آستان قدس رضوی & معافیت کامل (موقوفات) & ۸۰,۰۰۰+ & قوانین موقوفات \\
\hline
کمیته امداد & معافیت کامل & ۳۰,۰۰۰+ & --- \\
\hline
بنیاد شهید & معافیت کامل & ۲۰,۰۰۰+ & --- \\
\hline
\multicolumn{4}{|c|}{\cellcolor{orange!15}\textbf{مناطق آزاد و ویژه}} \\
\hline
مناطق آزاد (۷ منطقه) & معافیت ۲۰ ساله & ۲۰۰,۰۰۰+ & قانون مناطق آزاد \\
\hline
مناطق ویژه (۲۰+ منطقه) & معافیت‌های متفاوت & ۱۰۰,۰۰۰+ & قوانین خاص \\
\hline
\multicolumn{4}{|c|}{\cellcolor{yellow!15}\textbf{بخش‌های خاص}} \\
\hline
کشاورزی & معافیت گسترده & ۱۵۰,۰۰۰+ & ماده ۸۱ ق.م.م \\
\hline
صنایع معاف & معافیت‌های تشویقی & ۱۰۰,۰۰۰+ & مواد متعدد \\
\hline
\multicolumn{4}{|c|}{\cellcolor{gray!15}\textbf{فرار مالیاتی}} \\
\hline
اقتصاد غیررسمی & عدم ثبت & ۳۰۰,۰۰۰+ & --- \\
\hline
اصناف و مشاغل & کم‌اظهاری & ۱۵۰,۰۰۰+ & --- \\
\hline
\multicolumn{2}{|r|}{\textbf{مجموع هزینه مالیاتی سالانه}} & \textbf{۱,۳۸۰,۰۰۰+} & معادل ۳۰+ میلیارد دلار \\
\hline
\end{tabular}
\end{table}

\subsection{نظام مالیاتی کارمندمحور}

\begin{figure}[htbp]
\centering
\begin{tikzpicture}
\begin{axis}[
    ybar,
    xlabel={منبع مالیات},
    ylabel={درصد از کل وصولی},
    symbolic x coords={حقوق‌بگیران,شرکت‌ها,مشاغل,ارزش افزوده,ثروت و دارایی,سایر},
    xtick=data,
    x tick label style={rotate=45, anchor=east, font=\small},
    nodes near coords,
    nodes near coords align={vertical},
    width=14cm,
    height=8cm,
    bar width=0.8cm,
    ymin=0,
    ymax=45,
    legend pos=north east,
]

\addplot[fill=red!60] coordinates {
    (حقوق‌بگیران,38) (شرکت‌ها,22) (مشاغل,8) (ارزش افزوده,25) (ثروت و دارایی,3) (سایر,4)
};

\end{axis}
\end{tikzpicture}
\caption{ترکیب منابع مالیاتی ایران}
\label{fig:tax-sources}
\end{figure}

\begin{tcolorbox}[colback=red!5!white,colframe=red!75!black,
    title=پارادوکس مالیاتی ایران]
\textbf{مشکل:} در ایران، کارمندان و حقوق‌بگیران که:
\begin{itemize}
    \item کمترین امکان فرار مالیاتی را دارند
    \item کمترین قدرت چانه‌زنی سیاسی را دارند
    \item کمترین سهم از درآمد ملی را دارند
\end{itemize}
\textbf{بیشترین سهم مالیات بر درآمد را می‌پردازند!}

در حالی که:
\begin{itemize}
    \item بنیادها و نهادهای بزرگ: معاف
    \item صاحبان مشاغل: کم‌اظهاری گسترده
    \item ثروتمندان: عدم وجود مالیات بر ثروت و ارث مؤثر
\end{itemize}
\end{tcolorbox}

\subsection{مقایسه بار مالیاتی گروه‌های مختلف}

\begin{table}[htbp]
\centering
\caption{برآورد نرخ مؤثر مالیات گروه‌های مختلف}
\label{tab:effective-tax-rates}
\begin{tabular}{|l|c|c|p{4cm}|}
\hline
\textbf{گروه} & \textbf{نرخ قانونی} & \textbf{نرخ مؤثر} & \textbf{توضیح} \\
\hline
کارمندان دولت & ۱۵-۳۵٪ & ۱۵-۲۵٪ & کسر از حقوق، امکان فرار نیست \\
\hline
کارمندان خصوصی & ۱۵-۳۵٪ & ۱۰-۲۰٪ & برخی توافقات زیرمیزی \\
\hline
پزشکان و وکلا & ۱۵-۳۵٪ & ۳-۸٪ & کم‌اظهاری گسترده \\
\hline
اصناف و کسبه & ۱۵-۲۵٪ & ۲-۵٪ & مالیات مقطوع، کم‌اظهاری \\
\hline
شرکت‌های بزرگ خصوصی & ۲۵٪ & ۱۰-۱۵٪ & معافیت‌ها، حسابداری خلاق \\
\hline
نهادهای شبه‌دولتی & ۲۵٪ & ۰-۲٪ & معافیت قانونی و عملی \\
\hline
بنیادها و موقوفات & ۰٪ & ۰٪ & معافیت کامل \\
\hline
مناطق آزاد & ۰٪ & ۰٪ & معافیت ۲۰ ساله \\
\hline
\end{tabular}
\end{table}

\subsection{پیشنهادات اصلاح نظام مالیاتی}

\begin{enumerate}
    \item \textbf{لغو معافیت‌های غیرضروری:}
    \begin{itemize}
        \item الزام تمام نهادها به پرداخت مالیات بر درآمد
        \item محدود کردن معافیت‌های مناطق آزاد
        \item حذف تدریجی معافیت‌های صنعتی غیرهدفمند
    \end{itemize}
    
    \item \textbf{گسترش پایه مالیاتی:}
    \begin{itemize}
        \item استقرار نظام جامع مالیات بر ثروت و دارایی
        \item اجرای مالیات بر عایدی سرمایه (CGT)
        \item مالیات بر ارث تصاعدی و مؤثر
    \end{itemize}
    
    \item \textbf{بهبود وصول:}
    \begin{itemize}
        \item الکترونیکی کردن کامل نظام مالیاتی
        \item اتصال به سامانه‌های بانکی و ثبتی
        \item افزایش ضمانت اجرا و مجازات فرار مالیاتی
    \end{itemize}
    
    \item \textbf{عدالت مالیاتی:}
    \begin{itemize}
        \item کاهش فشار بر حقوق‌بگیران
        \item افزایش سهم مالیات بر ثروت و درآمدهای سرمایه‌ای
        \item شفافیت در معافیت‌ها و هزینه‌های مالیاتی
    \end{itemize}
\end{enumerate}

%═══════════════════════════════════════════════════════════════════════════════
\section{ناترازی بانکی: بمب ساعتی اقتصاد}
\label{sec:banking-imbalance}
%═══════════════════════════════════════════════════════════════════════════════

\subsection{وضعیت نظام بانکی}

\begin{table}[htbp]
\centering
\caption{شاخص‌های کلیدی نظام بانکی ایران}
\label{tab:banking-indicators}
\begin{tabular}{|l|r|r|l|}
\hline
\textbf{شاخص} & \textbf{ایران} & \textbf{استاندارد} & \textbf{وضعیت} \\
\hline
نسبت کفایت سرمایه & ۴-۶٪ & $>$۸٪ & \textcolor{red}{بحرانی} \\
\hline
نسبت مطالبات غیرجاری (NPL) & ۱۵-۲۵٪ & $<$۵٪ & \textcolor{red}{بحرانی} \\
\hline
نسبت دارایی‌های منجمد & ۳۰-۴۰٪ & $<$۱۰٪ & \textcolor{red}{بحرانی} \\
\hline
نرخ سود سپرده واقعی & منفی & مثبت & \textcolor{red}{بحرانی} \\
\hline
نسبت تسهیلات به سپرده & ۹۰-۱۱۰٪ & ۷۰-۸۰٪ & \textcolor{orange}{نگران‌کننده} \\
\hline
\end{tabular}
\end{table}

\subsection{منابع ناترازی بانکی}

\begin{figure}[htbp]
\centering
\begin{tikzpicture}[
    node distance=1.5cm,
    box/.style={rectangle, draw, fill=#1, minimum width=3.5cm, 
        minimum height=1cm, text centered, font=\small, rounded corners},
    arrow/.style={->, thick, >=stealth}
]

% علل
\node[box=red!30] (directed) at (-5,3) {تسهیلات دستوری};
\node[box=red!30] (related) at (-5,1) {وام به اشخاص مرتبط};
\node[box=red!30] (frozen) at (-5,-1) {دارایی‌های منجمد};
\node[box=red!30] (rate) at (-5,-3) {سرکوب مالی};

% ناترازی
\node[box=purple!40, minimum width=4cm, minimum height=2cm] (imbalance) at (0,0) 
    {\textbf{ناترازی بانکی}\\NPL: ۲۰-۲۵٪\\کسری سرمایه: ۵۰۰+ تریلیون};

% پیامدها
\node[box=orange!30] (credit) at (5,3) {اختلال در تخصیص اعتبار};
\node[box=orange!30] (inflation) at (5,1) {فشار تورمی (خلق پول)};
\node[box=orange!30] (budget) at (5,-1) {فشار بر بودجه دولت};
\node[box=orange!30] (depositors) at (5,-3) {ریسک سپرده‌گذاران};

% فلش‌ها
\draw[arrow] (directed) -- (imbalance);
\draw[arrow] (related) -- (imbalance);
\draw[arrow] (frozen) -- (imbalance);
\draw[arrow] (rate) -- (imbalance);
\draw[arrow] (imbalance) -- (credit);
\draw[arrow] (imbalance) -- (inflation);
\draw[arrow] (imbalance) -- (budget);
\draw[arrow] (imbalance) -- (depositors);

\end{tikzpicture}
\caption{علل و پیامدهای ناترازی بانکی}
\label{fig:banking-imbalance-causes}
\end{figure}

\subsection{تسهیلات رانتی: مکانیزم فساد بانکی}

\begin{table}[htbp]
\centering
\caption{گونه‌شناسی تسهیلات رانتی در نظام بانکی}
\label{tab:rent-seeking-loans}
\small
\begin{tabular}{|l|p{4cm}|p{3cm}|r|}
\hline
\textbf{نوع} & \textbf{شرح} & \textbf{ذی‌نفعان} & \textbf{حجم (T\$)} \\
\hline
\multicolumn{4}{|c|}{\cellcolor{red!15}\textbf{تسهیلات دستوری}} \\
\hline
تکلیفی دولت & وام‌های ارزان اجباری & بخش‌های خاص & ۵۰۰,۰۰۰+ \\
\hline
طرح‌های عمرانی & تأمین مالی پروژه‌های دولتی & پیمانکاران خاص & ۳۰۰,۰۰۰+ \\
\hline
\multicolumn{4}{|c|}{\cellcolor{orange!15}\textbf{تسهیلات مرتبط}} \\
\hline
سهامداران بانک & وام به شرکت‌های وابسته & مالکان بانک & ۲۰۰,۰۰۰+ \\
\hline
اعضای هیئت‌مدیره & وام با شرایط ویژه & مدیران و آشنایان & ۱۰۰,۰۰۰+ \\
\hline
\multicolumn{4}{|c|}{\cellcolor{yellow!15}\textbf{تسهیلات سیاسی}} \\
\hline
اشخاص بانفوذ & وام با فشار سیاسی & نخبگان سیاسی & ۱۵۰,۰۰۰+ \\
\hline
نهادهای خاص & وام بدون وثیقه کافی & نهادهای شبه‌دولتی & ۲۵۰,۰۰۰+ \\
\hline
\end{tabular}
\end{table}

\subsection{پرونده‌های فساد بانکی}

\begin{table}[htbp]
\centering
\caption{بزرگ‌ترین پرونده‌های فساد بانکی ایران}
\label{tab:banking-scandals}
\small
\begin{tabular}{|l|c|p{5cm}|l|}
\hline
\textbf{پرونده} & \textbf{سال} & \textbf{شرح} & \textbf{مبلغ} \\
\hline
اختلاس ۳ هزار میلیاردی & ۱۳۹۰ & تسهیلات کلان بدون وثیقه، سوءاستفاده & ۳ هزار میلیارد تومان \\
\hline
بابک زنجانی & ۱۳۹۲ & عدم بازگشت ارز نفتی & ۲.۷ میلیارد دلار \\
\hline
موسسات غیرمجاز & ۱۳۹۵-۹۷ & کاسپین، ثامن‌الحجج، آرمان & ۱۰۰+ هزار میلیارد \\
\hline
پرونده بانک سرمایه & ۱۳۹۸ & تسهیلات بدون بازگشت & ۴ هزار میلیارد تومان \\
\hline
سکه ثامن & ۱۳۹۷ & کلاهبرداری پیش‌فروش طلا & ۲ هزار میلیارد تومان \\
\hline
\end{tabular}
\end{table}

%═══════════════════════════════════════════════════════════════════════════════
\section{ناترازی قیمتی: یارانه‌های پنهان}
\label{sec:price-imbalance}
%═══════════════════════════════════════════════════════════════════════════════

\subsection{ساختار یارانه‌های انرژی}

\begin{table}[htbp]
\centering
\caption{مقایسه قیمت حامل‌های انرژی در ایران و جهان (۱۴۰۲)}
\label{tab:energy-prices}
\begin{tabular}{|l|r|r|r|c|}
\hline
\textbf{حامل} & \textbf{قیمت ایران} & \textbf{قیمت جهانی} & \textbf{یارانه (٪)} & \textbf{انگیزه قاچاق} \\
\hline
بنزین (لیتر) & ۳,۰۰۰ تومان & ۶۰,۰۰۰ تومان & ۹۵٪ & بسیار بالا \\
\hline
گازوئیل (لیتر) & ۸۰۰ تومان & ۵۰,۰۰۰ تومان & ۹۸٪ & بسیار بالا \\
\hline
گاز طبیعی (m³) & ۱,۵۰۰ تومان & ۲۵,۰۰۰ تومان & ۹۴٪ & بالا (صنایع) \\
\hline
برق (kWh) & ۱,۰۰۰ تومان & ۸,۰۰۰ تومان & ۸۸٪ & متوسط \\
\hline
نان (قرص) & ۵,۰۰۰ تومان & ۱۵,۰۰۰ تومان & ۶۷٪ & پایین \\
\hline
\end{tabular}
\end{table}

\subsection{برآورد کل یارانه‌های پنهان}

\begin{figure}[htbp]
\centering
\begin{tikzpicture}
\begin{axis}[
    ybar stacked,
    xlabel={سال},
    ylabel={میلیارد دلار},
    symbolic x coords={۱۳۹۸,۱۳۹۹,۱۴۰۰,۱۴۰۱,۱۴۰۲},
    xtick=data,
    legend pos=outer north east,
    ymajorgrids=true,
    grid style=dashed,
    width=12cm,
    height=8cm,
    bar width=0.8cm,
    ymin=0,
    ymax=120,
]

\addplot[fill=red!60] coordinates {
    (۱۳۹۸,35) (۱۳۹۹,40) (۱۴۰۰,45) (۱۴۰۱,50) (۱۴۰۲,55)
};
\addlegendentry{بنزین و گازوئیل}

\addplot[fill=orange!60] coordinates {
    (۱۳۹۸,20) (۱۳۹۹,22) (۱۴۰۰,25) (۱۴۰۱,28) (۱۴۰۲,30)
};
\addlegendentry{گاز طبیعی}

\addplot[fill=yellow!60] coordinates {
    (۱۳۹۸,10) (۱۳۹۹,12) (۱۴۰۰,14) (۱۴۰۱,16) (۱۴۰۲,18)
};
\addlegendentry{برق}

\addplot[fill=green!60] coordinates {
    (۱۳۹۸,5) (۱۳۹۹,6) (۱۴۰۰,7) (۱۴۰۱,8) (۱۴۰۲,9)
};
\addlegendentry{آب و سایر}

\end{axis}
\end{tikzpicture}
\caption{برآورد یارانه‌های پنهان انرژی و کالاهای اساسی}
\label{fig:hidden-subsidies}
\end{figure}

\subsection{توزیع ناعادلانه یارانه‌ها}

یکی از مهم‌ترین مشکلات یارانه‌های پنهان، توزیع ناعادلانه آن‌هاست:

\begin{figure}[htbp]
\centering
\begin{tikzpicture}
\begin{axis}[
    ybar,
    xlabel={دهک درآمدی},
    ylabel={سهم از یارانه پنهان (٪)},
    symbolic x coords={دهک ۱,دهک ۲,دهک ۳,دهک ۴,دهک ۵,دهک ۶,دهک ۷,دهک ۸,دهک ۹,دهک ۱۰},
    xtick=data,
    x tick label style={rotate=45, anchor=east, font=\tiny},
    nodes near coords,
    nodes near coords align={vertical},
    nodes near coords style={font=\tiny},
    width=14cm,
    height=7cm,
    bar width=0.6cm,
    ymin=0,
    ymax=35,
]

\addplot[fill=red!60] coordinates {
    (دهک ۱,3) (دهک ۲,4) (دهک ۳,5) (دهک ۴,6) (دهک ۵,7)
    (دهک ۶,9) (دهک ۷,11) (دهک ۸,14) (دهک ۹,18) (دهک ۱۰,30)
};

\end{axis}
\end{tikzpicture}
\caption{توزیع یارانه‌های پنهان انرژی بین دهک‌های درآمدی}
\label{fig:subsidy-distribution}
\end{figure}

\begin{tcolorbox}[colback=red!5!white,colframe=red!75!black,
    title=پارادوکس یارانه‌های پنهان]
\textbf{واقعیت تلخ:}
\begin{itemize}
    \item دهک دهم (ثروتمندترین) حدود \textbf{۳۰٪} یارانه انرژی را دریافت می‌کند
    \item دهک اول (فقیرترین) تنها \textbf{۳٪} دریافت می‌کند
    \item نسبت: ثروتمندان \textbf{۱۰ برابر} فقرا یارانه می‌گیرند!
\end{itemize}

\textbf{دلایل:}
\begin{itemize}
    \item مصرف بیشتر انرژی (خودرو، خانه بزرگ‌تر، کولر و...)
    \item چند خودرو در خانواده‌های ثروتمند
    \item صنایع و کسب‌وکارهای وابسته
\end{itemize}

\textbf{نتیجه:} یارانه‌های پنهان عملاً «بازتوزیع معکوس» هستند!
\end{tcolorbox}

\subsection{پیامدهای ناترازی قیمتی}

\begin{table}[htbp]
\centering
\caption{پیامدهای چندگانه یارانه‌های پنهان انرژی}
\label{tab:subsidy-consequences}
\begin{tabular}{|l|p{5cm}|p{5cm}|}
\hline
\textbf{حوزه} & \textbf{پیامد منفی} & \textbf{ذی‌نفعان} \\
\hline
\multicolumn{3}{|c|}{\cellcolor{red!15}\textbf{پیامدهای اقتصادی}} \\
\hline
قاچاق & صادرات غیرقانونی ۱۰-۲۰ میلیون لیتر سوخت روزانه & شبکه‌های قاچاق، برخی مرزنشینان \\
\hline
ناکارآمدی صنایع & صنایع انرژی‌بر غیررقابتی، عدم انگیزه صرفه‌جویی & صنایع ناکارآمد \\
\hline
فشار بر بودجه & ۸۰-۱۰۰ میلیارد دلار سالانه هزینه فرصت & --- \\
\hline
\multicolumn{3}{|c|}{\cellcolor{orange!15}\textbf{پیامدهای زیست‌محیطی}} \\
\hline
آلودگی هوا & مصرف بالای سوخت، خودروهای فرسوده & --- \\
\hline
انتشار گازهای گلخانه‌ای & ایران جزو ۱۰ کشور اول انتشار CO2 & --- \\
\hline
کمبود آب & مصرف بی‌رویه آب کشاورزی (برق و سوخت ارزان) & --- \\
\hline
\multicolumn{3}{|c|}{\cellcolor{yellow!15}\textbf{پیامدهای اجتماعی}} \\
\hline
نابرابری & توزیع معکوس به نفع ثروتمندان & دهک‌های بالا \\
\hline
فرهنگ مصرف & عدم صرفه‌جویی، اتلاف منابع & --- \\
\hline
\end{tabular}
\end{table}

%═══════════════════════════════════════════════════════════════════════════════
\section{ناترازی بودجه‌ای و شفافیت مالی}
\label{sec:budget-imbalance}
%═══════════════════════════════════════════════════════════════════════════════

\subsection{ساختار بودجه و عدم شفافیت}

\begin{figure}[htbp]
\centering
\begin{tikzpicture}[
    box/.style={rectangle, draw, fill=#1, minimum width=3.5cm, 
        minimum height=1cm, text centered, font=\small, rounded corners},
    arrow/.style={->, thick, >=stealth}
]

% منابع
\node[font=\bfseries] at (-5,5) {منابع بودجه};
\node[box=green!30] (oil) at (-5,3.5) {درآمد نفت\\۳۰-۴۰٪};
\node[box=blue!30] (tax) at (-5,2) {مالیات‌ها\\۳۰-۳۵٪};
\node[box=orange!30] (other) at (-5,0.5) {سایر درآمدها\\۱۵-۲۰٪};
\node[box=red!30] (deficit) at (-5,-1) {کسری/استقراض\\۱۵-۲۵٪};

% بودجه
\node[box=purple!30, minimum height=3cm, minimum width=3cm] (budget) at (0,1.5) 
    {\textbf{بودجه عمومی}\\حدود ۱,۵۰۰\\هزار میلیارد تومان};

% مصارف
\node[font=\bfseries] at (5,5) {مصارف بودجه};
\node[box=red!30] (salary) at (5,3.5) {حقوق و دستمزد\\۶۰-۷۰٪};
\node[box=orange!30] (current) at (5,2) {سایر جاری\\۱۵-۲۰٪};
\node[box=yellow!30] (capital) at (5,0.5) {عمرانی\\۱۰-۱۵٪};
\node[box=gray!30] (hidden) at (5,-1) {ردیف‌های مبهم\\؟؟٪};

% فلش‌ها
\draw[arrow] (oil) -- (budget);
\draw[arrow] (tax) -- (budget);
\draw[arrow] (other) -- (budget);
\draw[arrow] (deficit) -- (budget);
\draw[arrow] (budget) -- (salary);
\draw[arrow] (budget) -- (current);
\draw[arrow] (budget) -- (capital);
\draw[arrow, dashed, red] (budget) -- (hidden);

\end{tikzpicture}
\caption{ساختار کلی بودجه عمومی ایران}
\label{fig:budget-structure}
\end{figure}

\subsection{ابعاد عدم شفافیت بودجه‌ای}

\begin{table}[htbp]
\centering
\caption{شاخص‌های شفافیت بودجه ایران در مقایسه با جهان}
\label{tab:budget-transparency}
\begin{tabular}{|l|c|c|c|l|}
\hline
\textbf{شاخص} & \textbf{ایران} & \textbf{میانگین منطقه} & \textbf{میانگین جهان} & \textbf{منبع} \\
\hline
شاخص بودجه باز (OBI) & ۳۱/۱۰۰ & ۳۸ & ۴۵ & IBP \\
\hline
مشارکت عمومی & ۱۵/۱۰۰ & ۲۰ & ۱۴ & IBP \\
\hline
نظارت بودجه‌ای & ۲۸/۱۰۰ & ۳۵ & ۳۳ & IBP \\
\hline
دسترسی به اسناد & محدود & --- & --- & --- \\
\hline
\end{tabular}
\end{table}

\begin{tcolorbox}[colback=orange!5!white,colframe=orange!75!black,
    title=ردیف‌های مبهم و سرّی بودجه]

\textbf{مصادیق عدم شفافیت:}
\begin{enumerate}
    \item \textbf{بودجه نهادهای خاص:} بودجه نهاد رهبری، شورای نگهبان، مجمع 
    تشخیص به صورت کلی و بدون جزئیات
    
    \item \textbf{بودجه نظامی-امنیتی:} جزئیات بودجه سپاه، ارتش، وزارت 
    اطلاعات غیرشفاف
    
    \item \textbf{شرکت‌های دولتی:} بودجه شرکت‌های دولتی عمدتاً خارج از 
    نظارت مجلس
    
    \item \textbf{صندوق توسعه ملی:} گزارش‌دهی ناقص از دارایی‌ها و مصارف
    
    \item \textbf{ردیف‌های متفرقه:} ردیف‌هایی با عناوین مبهم مثل «سایر»، 
    «هزینه‌های پیش‌بینی نشده»
\end{enumerate}

\textbf{حجم تخمینی بودجه غیرشفاف:} ۲۰-۳۰٪ کل بودجه عمومی
\end{tcolorbox}

\subsection{کسری بودجه و تأمین مالی آن}

\begin{figure}[htbp]
\centering
\begin{tikzpicture}
\begin{axis}[
    xlabel={سال},
    ylabel={کسری بودجه (هزار میلیارد تومان)},
    xmin=1395, xmax=1403,
    ymin=0, ymax=700,
    xtick={1395,1396,1397,1398,1399,1400,1401,1402,1403},
    legend pos=north west,
    ymajorgrids=true,
    grid style=dashed,
    width=14cm,
    height=7cm,
]

\addplot[thick, color=red, mark=*, mark size=2pt] coordinates {
    (1395,50) (1396,80) (1397,120) (1398,180) (1399,300)
    (1400,400) (1401,500) (1402,600) (1403,650)
};
\addlegendentry{کسری بودجه}

\end{axis}
\end{tikzpicture}
\caption{روند کسری بودجه عمومی (۱۳۹۵-۱۴۰۳)}
\label{fig:budget-deficit}
\end{figure}

\begin{table}[htbp]
\centering
\caption{روش‌های تأمین کسری بودجه و پیامدهای آن}
\label{tab:deficit-financing}
\begin{tabular}{|l|p{4cm}|p{4cm}|c|}
\hline
\textbf{روش} & \textbf{شرح} & \textbf{پیامد} & \textbf{سهم تخمینی} \\
\hline
استقراض از بانک مرکزی & چاپ پول (تنخواه‌گردان) & \textcolor{red}{تورم} & ۳۰-۴۰٪ \\
\hline
فروش اوراق & اوراق مشارکت، صکوک & افزایش بدهی دولت & ۲۵-۳۰٪ \\
\hline
برداشت از صندوق توسعه & استفاده از ذخایر ارزی & تضعیف صندوق & ۱۵-۲۰٪ \\
\hline
فروش دارایی & واگذاری سهام، اموال & یک‌باره، ناپایدار & ۱۰-۱۵٪ \\
\hline
معوقات & بدهی به پیمانکاران، بانک‌ها & انتقال بحران به آینده & ۱۰-۱۵٪ \\
\hline
\end{tabular}
\end{table}

%═══════════════════════════════════════════════════════════════════════════════
\section{ناترازی نهادی: قدرت بدون پاسخگویی}
\label{sec:institutional-imbalance}
%═══════════════════════════════════════════════════════════════════════════════

\subsection{ساختار نهادی اقتصاد ایران}

\begin{figure}[htbp]
\centering
\begin{tikzpicture}[
    level 1/.style={sibling distance=5cm, level distance=2.5cm},
    level 2/.style={sibling distance=2.5cm, level distance=2cm},
    every node/.style={rectangle, draw, align=center, font=\small}
]

\node[fill=red!20] {قدرت اقتصادی در ایران}
    child {node[fill=blue!20] {دولت رسمی\\(۲۵-۳۰٪ GDP)}
        child {node[fill=blue!10] {شرکت‌های\\دولتی}}
        child {node[fill=blue!10] {وزارتخانه‌ها}}
    }
    child {node[fill=orange!30] {نهادهای موازی\\(۲۵-۴۰٪ GDP)}
        child {node[fill=orange!20] {سپاه و\\قرارگاه‌ها}}
        child {node[fill=orange!20] {بنیادها}}
        child {node[fill=orange!20] {آستان‌ها}}
    }
    child {node[fill=green!20] {بخش خصوصی\\(۳۰-۴۰٪ GDP)}
        child {node[fill=green!10] {خصوصی\\واقعی}}
        child {node[fill=yellow!20] {شبه‌خصوصی}}
    }
;
\end{tikzpicture}
\caption{ساختار قدرت اقتصادی در ایران}
\label{fig:economic-power-structure}
\end{figure}

\subsection{ماتریس قدرت و پاسخگویی}

\begin{table}[htbp]
\centering
\caption{ماتریس قدرت اقتصادی و پاسخگویی نهادها}
\label{tab:power-accountability-matrix}
\begin{tabular}{|l|c|c|c|c|c|}
\hline
\textbf{نهاد} & \textbf{قدرت اقتصادی} & \textbf{پاسخگویی به مجلس} & \textbf{پاسخگویی به قوه قضائیه} & \textbf{شفافیت مالی} & \textbf{نظارت عمومی} \\
\hline
دولت (قوه مجریه) & بالا & متوسط & پایین & متوسط & متوسط \\
\hline
سپاه پاسداران & بسیار بالا & \textcolor{red}{ندارد} & \textcolor{red}{ندارد} & \textcolor{red}{ندارد} & \textcolor{red}{ندارد} \\
\hline
بنیاد مستضعفان & بالا & \textcolor{red}{ندارد} & پایین & \textcolor{red}{ندارد} & \textcolor{red}{ندارد} \\
\hline
آستان قدس & بالا & \textcolor{red}{ندارد} & \textcolor{red}{ندارد} & \textcolor{red}{ندارد} & \textcolor{red}{ندارد} \\
\hline
ستاد اجرایی & بالا & \textcolor{red}{ندارد} & \textcolor{red}{ندارد} & \textcolor{red}{ندارد} & \textcolor{red}{ندارد} \\
\hline
بخش خصوصی & متوسط & غیرمستقیم & بالا & متوسط & بالا \\
\hline
\end{tabular}
\end{table}

\begin{tcolorbox}[colback=red!5!white,colframe=red!75!black,
    title=یافته کلیدی: قدرت معکوس پاسخگویی]
\textbf{الگوی واضح:}
\begin{itemize}
    \item نهادهایی که بیشترین قدرت اقتصادی را دارند، کمترین پاسخگویی را دارند
    \item بخش خصوصی که بیشترین نظارت را تحمل می‌کند، کمترین حمایت را دارد
    \item این الگو دقیقاً برعکس یک نظام سالم است
\end{itemize}

\textbf{نتیجه:} امکان فساد بدون ریسک کشف و مجازات فراهم است
\end{tcolorbox}

%═══════════════════════════════════════════════════════════════════════════════
\section{اولویت‌بندی اصلاحات}
\label{sec:reform-priorities}
%═══════════════════════════════════════════════════════════════════════════════

\subsection{ماتریس اولویت‌بندی}

برای اولویت‌بندی اصلاحات از دو معیار استفاده می‌کنیم:
\begin{itemize}
    \item \textbf{اثربخشی:} میزان کاهش فساد/رانت در صورت اجرا
    \item \textbf{امکان‌پذیری:} سهولت نسبی پیاده‌سازی (فنی و سیاسی)
\end{itemize}

\begin{figure}[htbp]
\centering
\begin{tikzpicture}
    % محورها
    \draw[thick, ->] (0,0) -- (10,0) node[right] {امکان‌پذیری};
    \draw[thick, ->] (0,0) -- (0,8) node[above] {اثربخشی};
    
    % خطوط مرجع
    \draw[dashed, gray] (5,0) -- (5,8);
    \draw[dashed, gray] (0,4) -- (10,4);
    
    % برچسب ربع‌ها
    \node[font=\small, gray] at (2.5,6) {\textbf{اولویت دوم}};
    \node[font=\small, gray] at (7.5,6) {\textbf{اولویت اول}};
    \node[font=\small, gray] at (2.5,2) {\textbf{اولویت چهارم}};
    \node[font=\small, gray] at (7.5,2) {\textbf{اولویت سوم}};
    
    % اصلاحات
    \node[circle, fill=red!60, minimum size=0.8cm, font=\tiny] at (7,7) {یکسان‌سازی ارز};
    \node[circle, fill=orange!60, minimum size=0.8cm, font=\tiny] at (4,7.5) {اصلاح مالیاتی};
    \node[circle, fill=yellow!60, minimum size=0.8cm, font=\tiny] at (8,5) {شفافیت بودجه};
    \node[circle, fill=green!60, minimum size=0.8cm, font=\tiny] at (3,6) {اصلاح بانکی};
    \node[circle, fill=blue!60, minimum size=0.8cm, font=\tiny] at (6,3) {اصلاح یارانه};
    \node[circle, fill=purple!60, minimum size=0.8cm, font=\tiny] at (2,5) {اصلاح نهادی};
    
\end{tikzpicture}
\caption{ماتریس اولویت‌بندی اصلاحات}
\label{fig:reform-priority-matrix}
\end{figure}

\subsection{نقشه راه اصلاحات}

\begin{table}[htbp]
\centering
\caption{نقشه راه اصلاحات ضدفساد و ضدرانت}
\label{tab:reform-roadmap}
\small
\begin{tabular}{|c|l|p{5cm}|c|c|}
\hline
\textbf{اولویت} & \textbf{اصلاح} & \textbf{اقدامات کلیدی} & \textbf{زمان} & \textbf{رانت حذف‌شده} \\
\hline
\multicolumn{5}{|c|}{\cellcolor{green!20}\textbf{اولویت اول: فوری و امکان‌پذیر}} \\
\hline
۱ & یکسان‌سازی ارز & حذف نرخ ترجیحی، نرخ شناور واحد & ۶-۱۲ ماه & ۱۵-۲۵ B\$ \\
\hline
۲ & شفافیت بودجه & انتشار جزئیات، بودجه باز & ۱۲ ماه & غیرقابل محاسبه \\
\hline
\multicolumn{5}{|c|}{\cellcolor{yellow!20}\textbf{اولویت دوم: پراثر اما دشوار}} \\
\hline
۳ & اصلاح مالیاتی & حذف معافیت‌ها، مالیات بر ثروت & ۲-۳ سال & ۳۰-۵۰ B\$ \\
\hline
۴ & بازسازی بانکی & تعیین تکلیف NPL، افزایش سرمایه & ۳-۵ سال & ۲۰-۴۰ B\$ \\
\hline
\multicolumn{5}{|c|}{\cellcolor{orange!20}\textbf{اولویت سوم: امکان‌پذیر اما کم‌اثرتر}} \\
\hline
۵ & اصلاح یارانه انرژی & افزایش تدریجی قیمت + یارانه نقدی & ۳-۵ سال & ۲۵-۳۵ B\$ \\
\hline
۶ & مبارزه با قاچاق & کنترل مرزی، مجازات سنگین & ۲-۳ سال & ۱۰-۱۵ B\$ \\
\hline
\multicolumn{5}{|c|}{\cellcolor{red!20}\textbf{اولویت چهارم: بنیادین اما نیازمند تغییرات بزرگ}} \\
\hline
۷ & اصلاح نهادی & پاسخگوسازی نهادها، استقلال قضایی & ۵-۱۰ سال & کل سیستم \\
\hline
۸ & فدرالیسم مالی & توزیع قدرت و منابع & ۵-۱۰ سال & ۱۰-۲۰ B\$ \\
\hline
\end{tabular}
\end{table}

\subsection{پیش‌شرط‌های اصلاحات}

\begin{tcolorbox}[colback=blue!5!white,colframe=blue!75!black,
    title=پیش‌شرط‌های موفقیت اصلاحات]

\textbf{۱. اراده سیاسی:}
\begin{itemize}
    \item اجماع نخبگان بر ضرورت اصلاحات
    \item پذیرش هزینه‌های کوتاه‌مدت برای منافع بلندمدت
    \item مقاومت در برابر گروه‌های ذی‌نفع
\end{itemize}

\textbf{۲. ظرفیت فنی:}
\begin{itemize}
    \item زیرساخت اطلاعاتی (کد ملی، سامانه‌های یکپارچه)
    \item نیروی انسانی متخصص
    \item سیستم‌های نظارتی کارآمد
\end{itemize}

\textbf{۳. مشروعیت اجتماعی:}
\begin{itemize}
    \item اعتماد عمومی به نیات اصلاحات
    \item شفافیت در فرآیند تصمیم‌گیری
    \item حمایت از اقشار آسیب‌پذیر
\end{itemize}

\textbf{۴. ثبات نسبی:}
\begin{itemize}
    \item بهبود یا ثبات روابط خارجی
    \item کنترل انتظارات تورمی
    \item حداقل اجماع سیاسی داخلی
\end{itemize}
\end{tcolorbox}

%═══════════════════════════════════════════════════════════════════════════════
\section{جمع‌بندی}
\label{sec:corruption-conclusion}
%═══════════════════════════════════════════════════════════════════════════════

این فصل نشان داد که فساد و ناترازی‌های اقتصادی در ایران:

\begin{enumerate}
    \item \textbf{سیستمی هستند نه استثنایی:} ریشه در ساختارهای نهادی دارند
    
    \item \textbf{چندلایه و به‌هم‌پیوسته‌اند:} ناترازی ارزی، مالیاتی، بانکی، 
    قیمتی و نهادی یکدیگر را تقویت می‌کنند
    
    \item \textbf{ذی‌نفعان قدرتمند دارند:} گروه‌هایی که از وضع موجود سود 
    می‌برند مقاومت می‌کنند
    
    \item \textbf{قابل اصلاح هستند:} با اراده سیاسی و طراحی درست، امکان 
    اصلاح وجود دارد
    
    \item \textbf{اصلاح آن‌ها نیازمند رویکرد جامع است:} اصلاحات جزیره‌ای 
    ناکارآمد است
\end{enumerate}

\begin{tcolorbox}[colback=green!5!white,colframe=green!75!black,
    title=پیام کلیدی]
\textbf{فساد در ایران یک انتخاب سیاسی است، نه یک ضرورت اقتصادی.}

کشورهای مشابه (از نظر درآمد نفتی، موقعیت جغرافیایی، تحریم‌ها) عملکرد 
بهتری در کنترل فساد داشته‌اند. تفاوت در نهادها و انتخاب‌های سیاسی است.

اصلاحات ممکن است، اما نیازمند:
\begin{itemize}
    \item گذار از «دولت رانتیر» به «دولت مالیاتی»
    \item شفافیت به عنوان اصل نه استثنا
    \item پاسخگویی همه نهادها بدون استثنا
    \item توزیع قدرت و جلوگیری از تمرکز
\end{itemize}
\end{tcolorbox}

%═══════════════════════════════════════════════════════════════════════════════
% منابع
%═══════════════════════════════════════════════════════════════════════════════
\section*{منابع فصل}
\addcontentsline{toc}{section}{منابع فصل}

\begin{itemize}
    \item شفافیت بین‌الملل. (۲۰۲۳). شاخص ادراک فساد.
    \item مشارکت بودجه بین‌الملل (IBP). (۲۰۲۱). پیمایش بودجه باز ایران.
    \item بانک جهانی. (۲۰۲۳). شاخص‌های حکمرانی جهانی.
    \item صندوق بین‌المللی پول. (۲۰۲۳). گزارش ماده ۴ ایران.
    \item مرکز آمار ایران. آمارنامه‌های سالانه.
    \item بانک مرکزی جمهوری اسلامی ایران. گزارش‌های اقتصادی.
    \item Klitgaard, R. (1988). Controlling Corruption. UC Press.
    \item Rose-Ackerman, S. (1999). Corruption and Government. Cambridge.
\end{itemize}